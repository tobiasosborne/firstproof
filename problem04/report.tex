\documentclass[11pt,a4paper]{article}

% --- Packages ---
\usepackage[utf8]{inputenc}
\usepackage[T1]{fontenc}
\usepackage{lmodern}
\usepackage[margin=2.5cm]{geometry}
\usepackage{amsmath,amssymb,amsthm}
\usepackage{mathtools}
\usepackage{enumitem}
\usepackage{booktabs}
\usepackage{array}
\usepackage{hyperref}

% --- Theorem environments ---
\newtheorem{theorem}{Theorem}[section]
\newtheorem{lemma}[theorem]{Lemma}
\newtheorem{proposition}[theorem]{Proposition}
\newtheorem{corollary}[theorem]{Corollary}
\newtheorem{conjecture}[theorem]{Conjecture}
\theoremstyle{definition}
\newtheorem{definition}[theorem]{Definition}
\newtheorem{remark}[theorem]{Remark}

% --- Macros ---
\newcommand{\Pn}{\mathcal{P}_n}
\newcommand{\Phin}{\Phi_n}
\newcommand{\boxplusn}{\boxplus_n}
\newcommand{\R}{\mathbb{R}}
\newcommand{\C}{\mathbb{C}}
\newcommand{\N}{\mathbb{N}}
\newcommand{\ip}[2]{\langle #1, #2 \rangle}
\newcommand{\norm}[1]{\|#1\|}
\DeclareMathOperator{\Var}{Var}
\DeclareMathOperator{\Hess}{Hess}

% --- Title ---
\title{\textbf{Problem~4: The Finite Free Stam Inequality}\\[6pt]
\large Critical Comparison of the Automated \texttt{af} Proof Attempt\\
with the Official Solution of Garza Vargas--Srivastava--Stier}
\author{First Proof Project --- Adversarial Proof Framework Analysis}
\date{February 2026}

\begin{document}
\maketitle

\begin{abstract}
We present a critical comparison between the automated adversarial proof
framework (\texttt{af}) attempt and the official correct solution for
Problem~4 of the First Proof benchmark: the finite free Stam inequality,
asserting that $1/\Phin(p \boxplusn q) \ge 1/\Phin(p) + 1/\Phin(q)$ for
monic real-rooted polynomials $p,q$ of degree~$n$.  The \texttt{af}
system explored three proof paths (subordination, de~Bruijn/concavity,
and entropy power inequality), proved the result for $n \le 3$, and
identified the $n \ge 4$ barrier as genuine.  The official proof by
Garza Vargas, Srivastava, and Stier succeeds via a fourth approach not
attempted by \texttt{af}: exploiting the Jacobian of the convolution
map, variance additivity, and the convexity of partial sums of
eigenvalues of hyperbolic polynomials (Bauschke et al., 2001), combined
with Blachman's classical argument.  We analyze what the \texttt{af}
system got right, where it went wrong, and what lessons this comparison
yields for AI-assisted mathematical proof.
\end{abstract}

\tableofcontents
\newpage

%======================================================================
\section{Problem Statement}
\label{sec:problem}
%======================================================================

For monic polynomials $p(x) = \prod_{i=1}^n(x - \alpha_i)$ and
$q(x) = \prod_{i=1}^n(x - \beta_i)$ with all roots real, the
\emph{finite free additive convolution} $p \boxplusn q$ is defined via
the Marcus--Spielman--Srivastava (MSS) formula:
\[
  (p \boxplusn q)(x) = \sum_{k=0}^n c_k\, x^{n-k},
  \qquad
  c_k = \sum_{i+j=k} \frac{(n-i)!\,(n-j)!}{n!\,(n-k)!}\, a_i\, b_j,
\]
where $a_i,b_j$ are the coefficients of $p,q$.  Equivalently, via the
permanent-like formula:
\[
  (p \boxplusn q)(x) = \sum_{\pi \in S_n} \prod_{i=1}^n (x - \alpha_i - \beta_{\pi(i)}).
\]
Walsh (1922) showed that this preserves real-rootedness.

The \emph{finite free score vector} is
$\mathcal{J}_n(\alpha)_i = \sum_{j \neq i} \frac{1}{\alpha_i -
\alpha_j}$, and the \emph{finite free Fisher information} is
$\Phin(p) = \|\mathcal{J}_n(\alpha)\|^2$.  The conjecture, due to
D.~Shlyakhtenko, is:

\begin{conjecture}[Finite free Stam inequality]
\label{conj:main}
For all monic real-rooted polynomials $p, q$ of degree $n$,
\begin{equation}\label{eq:stam}
  \frac{1}{\Phin(p \boxplusn q)} \ge \frac{1}{\Phin(p)} + \frac{1}{\Phin(q)}.
\end{equation}
\end{conjecture}

This is the finite-$n$ analogue of Voiculescu's free Stam inequality
\cite{V98}, proved in the continuum by Shlyakhtenko--Tao \cite{ST20}.

%======================================================================
\section{The Official Solution}
\label{sec:official}
%======================================================================

The proof by Garza Vargas, Srivastava, and Stier \cite{GVSSproof}
proceeds in three steps.

\subsection{Step 1: Score vectors and the Jacobian}

Let $\gamma = \Omega_{\boxplusn}(\alpha,\beta)$ denote the ordered
vector of roots of $p \boxplusn q$, and let $J_{\boxplusn}$ be the
Jacobian of the map $\Omega_{\boxplusn}$ at $(\alpha,\beta)$.  The
\emph{heat flow} $p_t = \exp(-\frac{t}{2}\partial_x^2) p$ moves the
roots via $\alpha_i'(0) = \mathcal{J}_n(\alpha)_i$ (Lemma~1.1 of
\cite{GVSSproof}).  Because finite free convolution commutes with
differential operators, the identity $r_{(a+b)t} = p_{at} \boxplusn
q_{bt}$ holds, which upon differentiation at $t=0$ yields
\begin{equation}\label{eq:jacobian-score}
  J_{\boxplusn}(a\,\mathcal{J}_n(\alpha),\; b\,\mathcal{J}_n(\beta))
  = (a+b)\,\mathcal{J}_n(\gamma).
\end{equation}
This is the key structural identity relating the three score vectors
through the Jacobian.

\subsection{Step 2: The Jacobian is a contraction on mean-zero vectors}

The central technical result is:

\begin{proposition}[Proposition~2.1 of \cite{GVSSproof}]
\label{prop:contraction}
If $u, v \in \R^n$ are both orthogonal to $\mathbf{1}_n$, then
\[
  \|J_{\boxplusn}(u,v)\|^2 \le \|u\|^2 + \|v\|^2.
\]
\end{proposition}

The proof of this proposition uses two ingredients:

\paragraph{Variance additivity and the Hessian identity.}
The subspace $\mathcal{V} = \{(u,v) : u^\top \mathbf{1} = v^\top
\mathbf{1} = 0\}$ has the property that perturbations within it preserve
the means of $\alpha$ and $\beta$.  Differentiating the variance
additivity relation $\Var(\gamma(t)) = \Var(\alpha(t)) +
\Var(\beta(t))$ twice yields
\begin{equation}\label{eq:hessian-identity}
  w^\top J_{\boxplusn} J_{\boxplusn}^\top w
  = \|w\|^2 - \sum_{i=1}^n \gamma_i\, w^\top H^{(i)}_{\boxplusn} w,
  \qquad w \in \mathcal{V},
\end{equation}
where $H^{(i)}_{\boxplusn} = \Hess_{\Omega_{\boxplusn,i}}$ is the
Hessian of the $i$-th coordinate function.

\paragraph{Convexity from hyperbolic polynomials.}
The multivariate polynomial $f(x, a, b) = \sum_{\pi \in S_n}
\prod_{i=1}^n (x - a_i - b_{\pi(i)})$ is hyperbolic in the direction
$e_1 = (1, 0, \ldots, 0)$.  A theorem of Bauschke, G\"uler, Lewis, and
Sendov \cite{BGLS01} asserts that for hyperbolic polynomials, the
partial sums $\sigma_k(a) = \sum_{i=1}^k \lambda_i(a)$ of the ordered
roots are convex functions of the parameters.  Since the $\gamma_i$ are
ordered, any linear combination $\sum c_i \gamma_i$ with $c_1 \ge
\cdots \ge c_n$ can be written as a positive combination of the
$\sigma_k$, and is therefore convex.  In particular,
$\sum_i \gamma_i H^{(i)}_{\boxplusn}$ is positive semidefinite
(Corollary~2.1 of \cite{GVSSproof}).

\noindent
Combining with \eqref{eq:hessian-identity}:
$\|J_{\boxplusn}(u,v)\|^2 = \|w\|^2 - \sum_i \gamma_i
w^\top H^{(i)} w \le \|w\|^2 = \|u\|^2 + \|v\|^2$.

\subsection{Step 3: Blachman's argument}

Since $\mathcal{J}_n(\alpha)$ is always orthogonal to $\mathbf{1}_n$
(each pair $(\alpha_i - \alpha_j)^{-1}$ and $(\alpha_j -
\alpha_i)^{-1}$ cancels in the sum), Proposition~\ref{prop:contraction}
applies.  Setting $a = 1/\Phin(p)$ and $b = 1/\Phin(q)$ in
\eqref{eq:jacobian-score} and applying the contraction bound:
\[
  (a+b)^2 \Phin(r) \le a^2 \Phin(p) + b^2 \Phin(q)
  = \frac{1}{\Phin(p)} + \frac{1}{\Phin(q)},
\]
which rearranges to $1/\Phin(r) \ge 1/\Phin(p) + 1/\Phin(q)$.
\hfill$\square$

%======================================================================
\section{The \texttt{af} Automated Attempt}
\label{sec:af}
%======================================================================

The adversarial proof framework (\texttt{af}) system attacked this
problem across two independent campaigns (designated ``examples6'' and
``examples7'') involving over 25 AI prover and verifier agents, 63
Python scripts totalling approximately 28,000 lines of verification
code, and more than 2,000,000 random numerical trials.  A third
campaign (``examples8'') synthesized the results into a 24-node hybrid
proof tree with three independent proof paths.

\subsection{What the \texttt{af} proved}

\begin{enumerate}[label=(\roman*)]
\item \textbf{Correct conjecture:} The system correctly identified the
  answer as TRUE, with zero violations in $>2$M numerical trials.

\item \textbf{Base cases $n \le 3$:} Two independent algebraic proofs
  for each of $n=2$ (exact Pythagorean equality on root gaps) and $n=3$
  (via Cauchy--Schwarz on cumulants, and via subordination quadratic
  forms).

\item \textbf{$n=4$ symmetric case ($e_3=0$):} Proved via strict
  concavity of a rational function and independently via a key lemma
  with vanishing third cumulant.

\item \textbf{Structural identities:}
  $\Phin(p) = 2\sum_{i<j}(\lambda_i - \lambda_j)^{-2}$;
  $S_2$-additivity $S_2(p \boxplusn q) = S_2(p) + S_2(q)$; the
  Gaussian splitting identity; the chain rule $H_r(\nu_k) =
  H_p(\lambda_{\sigma(k)}) - \alpha_k$ conditional on subordination.
\end{enumerate}

\subsection{Path A: Subordination + $L^2$ Contraction}

This path adapted the Shlyakhtenko--Tao \cite{ST20} continuum proof.
The plan was:
\begin{enumerate}
\item Construct finite subordination functions $\omega_p, \omega_q$
  with $G_r(z) = G_p(\omega_p(z)) = G_q(\omega_q(z))$.
\item Use the chain rule to decompose $\Phin(p) = \Phin(r) + J_p$ where
  $J_p \ge 0$ is the Fisher decrease.
\item Prove the ``Herglotz coupling lemma'' $J_p \cdot J_q \ge
  \|\mathcal{J}_n(\gamma)\|^4$, from which Stam follows by AM--GM.
\end{enumerate}

\textbf{Where it stalled:} The existence of finite subordination
(Node~1.3.1) was verified numerically for $n=2,3,4,5$ but not proved.
More critically, the continuum identity $\omega_p(z) + \omega_q(z) =
z + 1/(nG_r(z))$ was shown to be \emph{false} at finite $n$ (verified
at $n=2$), blocking the coupling argument.  The Herglotz coupling lemma
(Node~1.3.4) remained open: expressing the Fisher decreases as
quadratic forms in Herglotz residues was attempted but the coupling
constraint between $\omega_p$ and $\omega_q$ could not be formulated
without the continuum summation identity.

\subsection{Path B: De Bruijn + $1/\Phin$ Concavity}

This path followed the Costa--Villani entropy power template:
\begin{enumerate}
\item Establish the de Bruijn identity $dS/dt = \Phin(p_t)$ along the
  heat flow $p_t = p \boxplusn G_t$.
\item Prove concavity of $1/\Phin(p_t)$ in $t$
  (equivalently $\Phin \cdot \Phin'' \ge 2(\Phin')^2$).
\item Derive Stam from concavity + Gaussian splitting.
\end{enumerate}

\textbf{Where it stalled:} The concavity claim (Node~1.4.2) was
numerically validated (0 violations in 590 trials) but the required
inequality $\Phin \cdot \Phin'' \ge 2(\Phin')^2$, when expanded in
terms of root dynamics, involved power sums $\sum_{i \neq j}(\lambda_i
- \lambda_j)^{-k}$ of arbitrarily high order.  A Cauchy--Schwarz
argument was proposed but never completed for general $n$.

\subsection{Path C: Finite Entropy Power Inequality}

The most speculative path aimed to prove $N(p \boxplusn q) \ge N(p) +
N(q)$ where $N(p) = \exp(2S(p)/\binom{n}{2})$ is the entropy power,
and derive Stam by differentiation.  Approaches via the
Harish-Chandra--Itzykson--Zuber formula and Brascamp--Lieb inequality
were proposed but not developed.  Numerical support was strong (0
violations in 13,770 trials).

\subsection{Exhausted approaches}

The \texttt{af} system systematically identified and refuted several
false leads:
\begin{itemize}
\item $\ip{\mathcal{J}_n(\gamma)}{\alpha} \ge 0$: FALSE (counterexamples
  at $n \ge 3$).
\item Partition of unity $\omega_p'(z) + \omega_q'(z) = 1$: FALSE
  (each derivative equals 1 independently at the roots, not summing to 1).
\item Continuum summation $\omega_p + \omega_q = z + 1/(nG_r)$: FALSE
  at finite $n$.
\item Shape factor monotonicity: FALSE (42.7\% violation rate).
\item Monotone gap along heat flow: FALSE (44\% violation rate).
\item Joint concavity of $-R_4$: FALSE (indefinite Hessian).
\item SOS decomposition for $n \ge 4$: FAILS (mixed-sign cross terms).
\end{itemize}

%======================================================================
\section{Critical Comparison}
\label{sec:comparison}
%======================================================================

\subsection{Did any \texttt{af} path overlap with the successful approach?}

The official proof is structurally closest to Path~A (subordination/
$L^2$ contraction), but uses fundamentally different tools at the
critical step.  We can align the proofs as follows:

\begin{center}
\begin{tabular}{@{}p{5.8cm}p{5.8cm}@{}}
\toprule
\textbf{Official proof} & \textbf{\texttt{af} Path A} \\
\midrule
Score vectors as heat flow derivatives & Chain rule at roots
  (Node~1.3.2) \\
Jacobian relates score vectors via \eqref{eq:jacobian-score} &
  Subordination $\omega_p, \omega_q$ relates score vectors
  (Node~1.3.2) \\
Jacobian contraction on $\mathcal{V}$ (Prop.~\ref{prop:contraction}) &
  Herglotz coupling lemma $J_p \cdot J_q \ge \norm{h}^4$
  (Node~1.3.4) \\
Blachman's choice of $a,b$ to conclude & Harmonic mean from coupling
  (Node~1.3.5) \\
\bottomrule
\end{tabular}
\end{center}

Both proofs express the Stam inequality through a contraction/monotonicity
property of the map relating input and output score vectors.  The
\texttt{af} formulated this as a multiplicative coupling between Fisher
decreases ($J_p \cdot J_q \ge \norm{h}^4$), while the official proof
formulates it as an $L^2$ contraction of the Jacobian.  Both ultimately
reduce to Blachman's classical trick of choosing parameters
$a = 1/\Phin(p)$, $b = 1/\Phin(q)$.

However, the technical mechanism for establishing the contraction is
entirely different:
\begin{itemize}
\item The \texttt{af} attempted to work through subordination functions
  and their Herglotz representations, trying to express everything in
  terms of the residues and poles of $\omega_p, \omega_q$.  This is a
  complex-analytic approach.
\item The official proof works entirely in real variables: the Jacobian
  $J_{\boxplusn}$ is a real matrix, and its contraction property follows
  from variance additivity (a second-moment identity) and convexity of
  eigenvalue partial sums for hyperbolic polynomials (a result from
  convex algebraic geometry).
\end{itemize}

\subsection{The key insight the \texttt{af} missed}

The crucial insight of the official proof is the \emph{Hessian identity}
\eqref{eq:hessian-identity}, which converts the contraction question
into a positive-semidefiniteness question about weighted Hessians.
This identity has two remarkable features:

\begin{enumerate}
\item \textbf{Variance additivity as the engine.}  The fact that
  $\Var(\gamma) = \Var(\alpha) + \Var(\beta)$ under finite free
  convolution is a well-known, elementary property.  The official proof
  differentiates this identity \emph{twice} along perturbations in the
  mean-zero subspace to extract a Hessian identity.  The \texttt{af}
  system \emph{did know} this identity---it proved $S_2$-additivity
  (which is equivalent to variance additivity up to centering) as
  Node~1.1(ii).  But it never considered differentiating it to obtain
  information about the Jacobian.

\item \textbf{Hyperbolic polynomial convexity.}  The positive
  semidefiniteness of $\sum_i \gamma_i H^{(i)}_{\boxplusn}$ follows from
  the theorem of Bauschke et al.\ \cite{BGLS01} on convexity of partial
  sums of roots of hyperbolic polynomials.  This is a result from convex
  algebraic geometry that is not standard in the free probability
  literature.  The connection arises because the MSS convolution formula
  $\sum_{\pi \in S_n} \prod_i(x - a_i - b_{\pi(i)})$ is manifestly a
  hyperbolic polynomial in $x$ (it has only real roots by Walsh's
  theorem).  The \texttt{af} system never considered the convolution
  polynomial as a hyperbolic polynomial or invoked results from that
  theory.
\end{enumerate}

In summary, the \texttt{af} missed the interaction between two ideas it
partially possessed: (a) variance additivity (which it proved) and (b)
the structure of the convolution as a sum over permutations (which it
used for computations).  The official proof's innovation is to connect
these through the Jacobian and Hessian, then resolve the sign question
via hyperbolic polynomial theory.

\subsection{The \texttt{af}'s three paths in light of the solution}

\paragraph{Path A (Subordination).} This is the closest to the official
approach at the strategic level (both use a contraction/decomposition of
score vectors), but the \texttt{af} chose to route through subordination
functions---a complex-analytic detour that introduces unnecessary
difficulties.  The official proof avoids subordination entirely.  The
existence of finite subordination functions remains unproved and appears
to be a genuinely hard problem in its own right.  By going through the
Jacobian of the root map directly, the official proof sidesteps this
entirely.

\paragraph{Path B (De Bruijn/Concavity).} The concavity of $1/\Phin$
along the heat flow is a stronger statement than the Stam inequality
itself.  While the numerical evidence is compelling, no proof for
general $n$ was found.  The official proof does not use heat flow
concavity at all; it is a direct algebraic/geometric argument.  Path~B
remains potentially viable as an alternative approach, but it requires
establishing a strictly stronger result than necessary.

\paragraph{Path C (EPI).}  The finite entropy power inequality $N(p
\boxplusn q) \ge N(p) + N(q)$ is also a strictly stronger conjecture
than the Stam inequality.  The official proof does not address it.
Whether the finite EPI is true remains open.  If true, it would be a
deeper result requiring different techniques.

%======================================================================
\section{Evaluation of Dead Ends}
\label{sec:deadends}
%======================================================================

\subsection{Genuine dead ends a mathematician would also encounter}

Several of the \texttt{af}'s dead ends reflect traps that a human
mathematician working from the free probability literature would also
naturally fall into:

\begin{enumerate}
\item \textbf{Continuum identities failing at finite $n$.}  The
  identities $\omega_p + \omega_q = z + 1/(nG_r)$ and $\omega_p'(z) +
  \omega_q'(z) = 1$ are \emph{true} in the large-$n$ limit (they are
  Voiculescu's subordination identities for free convolution).  A
  mathematician familiar with free probability would naturally try to
  use these and would discover they fail at finite $n$.  The
  \texttt{af}'s discovery that each $\omega_p'(\nu_k) = 1$
  independently (rather than a partition of unity) is a genuine
  mathematical observation that clarifies the finite-$n$ structure.

\item \textbf{SOS failure for $n \ge 4$.}  Attempting a sum-of-squares
  proof is a standard approach for polynomial inequalities.  The failure
  of SOS for $n \ge 4$ reflects the genuine algebraic complexity of the
  problem: the inequality is not a consequence of any simple polynomial
  identity.

\item \textbf{Direct polynomial manipulation at $n \ge 4$.}  The
  combinatorial explosion of the MSS formula at $n=4$ (24 terms in $S_4$)
  makes brute-force approaches genuinely intractable.  A human
  mathematician would also find this route prohibitive.

\item \textbf{Subordination existence.}  Constructing finite
  subordination functions is an open problem that several researchers
  have considered.  The \texttt{af}'s difficulty here is shared with the
  human mathematical community.
\end{enumerate}

\subsection{Dead ends specific to the AI approach}

\begin{enumerate}
\item \textbf{Excessive reliance on the continuum template.}  The
  \texttt{af} organized all three proof paths as finite analogues of
  known continuum results (Shlyakhtenko--Tao, Costa--Villani,
  Shannon EPI).  A human mathematician might also start here, but
  would more quickly recognize the need for genuinely finite-dimensional
  tools.  The official proof, while inspired by Blachman's argument,
  introduces fundamentally new ideas (the Hessian identity from variance
  additivity, hyperbolic polynomial convexity) that have no direct
  continuum analogue.

\item \textbf{Not exploiting the permutation formula.}  The \texttt{af}
  used the MSS coefficient formula for computations but never exploited
  the permutation formula $\sum_{\pi \in S_n} \prod_i(x - \alpha_i -
  \beta_{\pi(i)})$ structurally.  The official proof recognizes this as
  defining a hyperbolic polynomial in the $(x, \alpha, \beta)$
  variables, which is the entry point for the Bauschke et al.\ convexity
  theorem.

\item \textbf{Searching for coupling constraints instead of convexity.}
  In Path~A, the \texttt{af} tried to find algebraic constraints between
  $\omega_p$ and $\omega_q$ to force the coupling inequality.  The
  official proof obtains the needed bound not from coupling but from
  \emph{convexity}---specifically, convexity of eigenvalue sums, which
  is a property of the convolution map itself, not of any auxiliary
  objects.
\end{enumerate}

%======================================================================
\section{What the \texttt{af} Got Right}
\label{sec:right}
%======================================================================

Despite failing to find the complete proof, the \texttt{af} system
demonstrated several strengths:

\begin{enumerate}
\item \textbf{Correct answer.}  The conjecture is indeed true, and
  the system correctly identified this with overwhelming numerical
  evidence ($>2$M trials, zero violations).

\item \textbf{Valid base cases.}  The proofs for $n=2$ (Pythagorean
  identity) and $n=3$ (two independent proofs) are correct and
  non-trivial.  The symmetric $n=4$ case was also proved.

\item \textbf{Structural identities.}  The identities $\Phin =
  2\sum_{i<j}(\lambda_i - \lambda_j)^{-2}$ and $S_2$-additivity are
  both used (implicitly) in the official proof.  The official proof's
  variance additivity (Proposition~1.1(i)) is essentially the
  $S_2$-additivity that the \texttt{af} proved.

\item \textbf{Blachman's argument was anticipated.}  The \texttt{af}'s
  Path~A strategy of decomposing $\Phin(p) = \Phin(r) + J_p$ and then
  using $J_p \cdot J_q \ge \Phin(r)^2$ is algebraically equivalent to
  Blachman's argument.  The system recognized this structure as the
  classical template.

\item \textbf{Identifying the true difficulty.}  The system correctly
  identified that the hard step is establishing a contraction or
  coupling property for general $n \ge 4$.  All three paths isolated
  a single hard lemma whose resolution would complete the proof.

\item \textbf{Efficient falsification.}  Numerical testing quickly
  eliminated 6+ false conjectures (inner product non-negativity, shape
  factor monotonicity, partition of unity, etc.), preventing wasted
  effort.  This is a genuine strength of the computational approach.

\item \textbf{Score vectors and heat flow.}  The \texttt{af} identified
  the score vector $\mathcal{J}_n(\alpha)$ and its connection to heat
  flow derivatives.  This is precisely Lemma~1.1 of the official proof.
  The \texttt{af}'s de Bruijn identity $dS/dt = \Phin(p_t)$ is the
  integral form of the same observation.
\end{enumerate}

%======================================================================
\section{Lessons for AI Approaches to Inequality Proofs}
\label{sec:lessons}
%======================================================================

\subsection{The template trap}

The \texttt{af} system organized its attack entirely around three
templates from the continuum setting: the subordination proof
(Shlyakhtenko--Tao), the concavity proof (Costa--Villani), and the
entropy power proof (Shannon--Stam).  Each template guided the
discovery of useful partial results, but none yielded the final proof.
The official proof uses a template (Blachman's argument) but supplies a
genuinely new technical ingredient (hyperbolic polynomial convexity)
that does not appear in any continuum analogue.

\emph{Lesson:} Templates from analogous problems are valuable for
structure, but solving genuinely new problems requires connecting to
tools outside the template's domain.  An AI system needs the ability
to search across mathematical subfields for applicable results.

\subsection{The abstraction barrier at $n \ge 4$}

All three \texttt{af} paths proved $n \le 3$ using direct computation
and failed at $n \ge 4$ because the combinatorial complexity overwhelms
algebraic manipulation.  The official proof succeeds at general $n$
because it works at the level of the Jacobian and Hessian---objects that
exist for all $n$ and whose properties can be established uniformly.

\emph{Lesson:} For inequality proofs that must hold for all $n$, the key
is often to find the right \emph{abstraction} that makes the proof
$n$-independent.  Direct computation on specific-degree instances,
while useful for building intuition, cannot substitute for finding the
right structural viewpoint.

\subsection{Known results as hidden tools}

The \texttt{af} proved $S_2$-additivity and knew the permutation formula
for $\boxplusn$, but never combined them into the Hessian identity
\eqref{eq:hessian-identity} or recognized the connection to hyperbolic
polynomial theory.  The Bauschke et al.\ theorem \cite{BGLS01} is a
published result from 2001, but it lives in the convex analysis
literature rather than the free probability literature.

\emph{Lesson:} The bottleneck in mathematical proof is often not proving
new results but recognizing which \emph{existing} results are relevant.
AI systems would benefit from broader literature search capabilities
and mechanisms for recognizing structural analogies across subfields.

\subsection{The value of the Jacobian viewpoint}

The \texttt{af} system worked with subordination functions
$\omega_p, \omega_q$ as the intermediary between input and output score
vectors.  The official proof works directly with the Jacobian of the
root map $\Omega_{\boxplusn}$.  The Jacobian is a more elementary and
more powerful object: it is a real matrix that can be analyzed with
linear algebra, rather than a complex-analytic function requiring
Herglotz representations.

\emph{Lesson:} When a problem involves a smooth map between
finite-dimensional spaces, the Jacobian is often the most natural
analytical tool.  The \texttt{af} introduced auxiliary complex-analytic
objects (subordination functions) that, while natural from the free
probability perspective, added unnecessary complexity.

\subsection{Numerical evidence: strengths and limitations}

The \texttt{af}'s numerical testing was a clear strength---it
correctly predicted the truth of the conjecture and efficiently
eliminated false leads.  However, numerical evidence also gave
confidence in subsidiary conjectures (concavity of $1/\Phin$, finite
EPI) that turned out to be unnecessary for the proof.

\emph{Lesson:} Numerical validation is excellent for falsification
(disproving wrong conjectures) and for building confidence in the
main claim, but it can lead to overinvestment in strong-but-unnecessary
auxiliary conjectures.  The official proof succeeds with a weaker
intermediate statement (Jacobian contraction) than any of the three
intermediate conjectures pursued by \texttt{af}.

\subsection{Comparison with the ChatGPT Pro~5.2 attempt}

The problem authors' commentary \cite{GVSSproof} notes that
ChatGPT Pro~5.2 also attempted this problem and tried Blachman's
approach via the random matrix model $r(x) = \mathbb{E}\det(xI - A -
UBU^\top)$.  That attempt failed because: (a) it only considered the
score function of $r(x)$ without mentioning those of $p(x)$ and $q(x)$;
(b) it did not exploit the preservation of real roots; (c) it asserted
incorrect residue calculus facts.

The \texttt{af} system avoided these specific errors: it worked with
all three score vectors simultaneously, it knew real-rootedness was
essential, and its algebraic computations were verified.  However, like
ChatGPT Pro~5.2, the \texttt{af} was unable to identify the correct
technical tool (hyperbolic polynomial convexity) for the key step.
Both AI systems demonstrate that current LLMs can organize proof
strategies and verify computations, but struggle to make the creative
leap of connecting a problem to an unexpected area of mathematics.

%======================================================================
\section{Conclusion}
\label{sec:conclusion}
%======================================================================

The finite free Stam inequality admits a complete proof of approximately
four pages, using:
\begin{itemize}
\item the connection between score vectors and heat flow derivatives,
\item variance additivity for finite free convolution,
\item convexity of eigenvalue partial sums for hyperbolic polynomials
  \cite{BGLS01},
\item Blachman's classical optimization trick.
\end{itemize}

The \texttt{af} system's investigation, while not reaching a complete
proof, produced substantial partial results: correct base cases,
important structural identities, efficient elimination of false leads,
and a clear mapping of the proof landscape through three viable
strategies.  The system's principal failure was not recognizing that
variance additivity---a fact it had already proved---could be
differentiated to yield information about the Jacobian, and that the
permutation formula for $\boxplusn$ gives a hyperbolic polynomial whose
eigenvalue convexity properties close the argument.

This case study illustrates both the current capabilities and
limitations of AI-assisted mathematical reasoning: strong on
computation, verification, and systematic exploration of known
frameworks; weak on cross-domain insight and the creative recognition of
how elementary known facts can be combined in unexpected ways.

\begin{thebibliography}{99}

\bibitem{GVSSproof}
J.~Garza Vargas, N.~Srivastava, and Z.~Stier,
\emph{The finite free Stam inequality},
in: First Proof Solutions and Comments, February 2026.

\bibitem{BGLS01}
H.~H.~Bauschke, O.~G\"uler, A.~S.~Lewis, and H.~S.~Sendov,
\emph{Hyperbolic polynomials and convex analysis},
Canad.\ J.\ Math.\ \textbf{53} (2001), 470--488.

\bibitem{Bla65}
N.~M.~Blachman,
\emph{The convolution inequality for entropy powers},
IEEE Trans.\ Inform.\ Theory \textbf{11} (1965), 267--271.

\bibitem{MSS22}
A.~W.~Marcus, D.~A.~Spielman, and N.~Srivastava,
\emph{Finite free convolutions of polynomials},
Probab.\ Theory Related Fields \textbf{182} (2022), 807--848.

\bibitem{V98}
D.~Voiculescu,
\emph{The analogues of entropy and of Fisher's information measure in
free probability theory V},
Invent.\ Math.\ \textbf{132} (1998), 189--227.

\bibitem{ST20}
D.~Shlyakhtenko and T.~Tao,
\emph{A self-contained analytic proof of the free monotonicity of the
free Fisher information},
arXiv:2009.01882, 2020.

\bibitem{Costa85}
M.~H.~M.~Costa,
\emph{A new entropy power inequality},
IEEE Trans.\ Inform.\ Theory \textbf{31} (1985), 751--760.

\bibitem{Wal22}
J.~L.~Walsh,
\emph{On the location of the roots of certain types of polynomials},
Trans.\ Amer.\ Math.\ Soc.\ \textbf{24} (1922), 163--180.

\bibitem{Nui68}
W.~Nuij,
\emph{A note on hyperbolic polynomials},
Math.\ Scand.\ \textbf{23} (1968), 69--72.

\end{thebibliography}

\end{document}
