\documentclass[11pt,a4paper]{article}

% --- Packages ---
\usepackage[utf8]{inputenc}
\usepackage[T1]{fontenc}
\usepackage{lmodern}
\usepackage[margin=2.5cm]{geometry}
\usepackage{amsmath,amssymb,amsthm}
\usepackage{mathtools}
\usepackage{enumitem}
\usepackage{booktabs}
\usepackage{array}
\usepackage{xcolor}
\usepackage{hyperref}

% --- Theorem environments ---
\newtheorem{theorem}{Theorem}[section]
\newtheorem{lemma}[theorem]{Lemma}
\newtheorem{proposition}[theorem]{Proposition}
\newtheorem{corollary}[theorem]{Corollary}
\theoremstyle{definition}
\newtheorem{definition}[theorem]{Definition}
\newtheorem{remark}[theorem]{Remark}

% --- Macros ---
\newcommand{\R}{\mathbb{R}}
\newcommand{\Z}{\mathbb{Z}}
\DeclareMathOperator{\Sp}{Sp}
\DeclareMathOperator{\LG}{LG}

% --- Colors ---
\definecolor{proved}{RGB}{0,128,0}
\definecolor{pending}{RGB}{200,150,0}

% --- Title ---
\title{\textbf{Critical Comparison: Automated Proof Attempt\\vs.\ Official Solution for Problem~8}\\[6pt]
\large Lagrangian Smoothing of Polyhedral Lagrangian Surfaces\\[4pt]
\normalsize First Proof Benchmark}
\author{Comparative Analysis Report\\
First Proof Project}
\date{February 2026}

\begin{document}
\maketitle

\begin{abstract}
This report critically compares the automated adversarial proof (\texttt{af}) attempt for Problem~8 of the First Proof benchmark with the official correct solution by Mohammed Abouzaid.
The problem asks whether a polyhedral Lagrangian surface~$K$ in~$(\R^4, \omega_{\mathrm{std}})$ with exactly 4 faces at every vertex necessarily admits a Lagrangian smoothing.
The answer is \textbf{YES}.
Over seven sessions, the \texttt{af} system evolved through three distinct strategies---tropical resolution, cotangent generating functions, and a two-zone construction---arriving at a proof attempt with 2 of 9 nodes validated and a self-assessed correctness of 55--65\%.
Abouzaid's official solution takes a fundamentally different and more elegant approach: it constructs a global \emph{conormal fibration} dual to~$K$, identifies \emph{smoothing functions} as the key analytic objects, and uses contractibility arguments to solve the local-to-global problem.
We find that the \texttt{af} attempt correctly identified the local smoothing near vertices (essentially the same linear algebra as Abouzaid's Lemma~1), but failed to find the conceptual framework---smoothing functions and conormal fibrations---that makes the global argument tractable.
This gap is precisely the local-to-global failure mode identified in the authors' commentary on AI solutions.
\end{abstract}

\tableofcontents
\newpage

%======================================================================
\section{The Problem}
\label{sec:problem}
%======================================================================

We work in~$(\R^4, \omega_{\mathrm{std}})$ with coordinates $(q_1, q_2, p_1, p_2)$ and symplectic form $\omega = dp_1 \wedge dq_1 + dp_2 \wedge dq_2$.

\begin{definition}
A \emph{polyhedral Lagrangian surface}~$K$ in~$\R^4$ is a finite polyhedral complex all of whose faces are Lagrangian 2-planes, which is a topological submanifold of~$\R^4$.
\end{definition}

\begin{definition}
A \emph{Lagrangian smoothing} of~$K$ is a Hamiltonian isotopy~$K^t$ of smooth Lagrangian submanifolds, parametrized by $t \in (0,1]$, extending to a topological isotopy on~$[0,1]$ with $K^0 = K$.
\end{definition}

\begin{proposition}[Abouzaid]
If $K$ is a polyhedral Lagrangian surface with exactly 4 faces meeting at every vertex, then $K$ has a Lagrangian smoothing.
\end{proposition}

The condition of exactly 4 faces per vertex is a combinatorial constraint that restricts the local geometry.  The challenge is that the smoothing must be \emph{Hamiltonian}---not merely smooth or symplectic---and that local smoothings near individual vertices and edges must be assembled into a \emph{globally} smooth, embedded Lagrangian surface.


%======================================================================
\section{The Official Solution}
\label{sec:official}
%======================================================================

Abouzaid's proof proceeds in two main stages: a local analysis near vertices and edges, followed by a global assembly using the notion of a \emph{conormal fibration}.

\subsection{Local Analysis: Vertex Normal Form}

\begin{lemma}[Abouzaid, Lemma~1]
\label{lem:vertex-normal-form}
For each piecewise-linear Lagrangian embedding $\R^2 \to \R^4$ that is linear on the four quadrants with Lagrangian image~$\Sigma$ not contained in a plane, there is a linear symplectic transformation mapping $\Sigma$ to the product of the unions of the positive coordinate axes in the two symplectic factor planes.
\end{lemma}

The proof is a short linear algebra argument: the four edge tangent vectors form a symplectic basis for~$\R^4$ (the key point being that the symplectic pairings $\omega(v_i, u_i)$ cannot vanish, or else $\omega$ would vanish on a 3-dimensional subspace).

\begin{corollary}[Abouzaid, Corollary~1]
There exists a linear Lagrangian plane $L \subset \R^4$ such that the symplectic pairing $\R^4 \to L^\vee$ defines a homeomorphism $\Sigma \to L^\vee$.
\end{corollary}

This is the crucial conceptual step: the singular Lagrangian~$\Sigma$ near a vertex is \emph{homeomorphic to the cotangent fibre}~$L^\vee$ via the symplectic pairing.  This gives~$\Sigma$ a canonical smooth structure.

\subsection{Smoothing Functions}

Rather than constructing explicit smooth Lagrangians by hand, Abouzaid introduces:

\begin{definition}[Abouzaid, Definition~2]
The space $\mathcal{S}(\Sigma)$ of \emph{smoothing functions} for~$\Sigma$ is the space of $C^1$ functions $f: \Sigma \to \R$ such that $f + q^\Sigma$ is $C^\infty$, where $q^\Sigma$ is the piecewise-quadratic function on~$\Sigma$ whose restriction to each face is the quadratic form arising from the cotangent-bundle description.
\end{definition}

\begin{lemma}[Abouzaid, Lemma~3]
The assignment $f \mapsto \Lambda_{df}$ determines a bijection between graphical Lagrangians (with respect to~$L$) and smoothing functions on~$\Sigma$ up to addition of constants.
\end{lemma}

This is the key local result: smooth Lagrangians near a vertex correspond bijectively to smoothing functions, which are analytically tractable objects.

\subsection{Local Analysis: Edges}

\begin{lemma}[Abouzaid, Lemma~4]
If $\Sigma$ consists of a pair of linear Lagrangian half-planes meeting along a line~$\ell$, then the space of Lagrangian subspaces~$L$ such that the symplectic pairing $\Sigma \to L^\vee$ is a homeomorphism is \emph{contractible}.
\end{lemma}

The contractibility of the space of admissible choices near edges is what makes the local-to-global assembly possible.

\subsection{The Global Argument: Conormal Fibrations}

This is the heart of the proof and the part that has no counterpart in the \texttt{af} attempt.

\begin{definition}[Abouzaid, Definition~3]
A \emph{conormal fibration} dual to~$K$ is a smoothly varying family $\{L_z\}_{z \in K}$ of affine-linear Lagrangian planes, satisfying:
\begin{enumerate}
\item Near each vertex, $L_z$ consists of translates of a single Lagrangian (from Corollary~1).
\item $L_z$ varies smoothly along edges.
\item $L_z$ varies smoothly along faces.
\item In a collar of each edge, the Lagrangians in the normal direction are translates of those along the edge.
\end{enumerate}
\end{definition}

\begin{lemma}[Abouzaid, Lemma~8]
$K$ admits a dual conormal fibration which, near each vertex, agrees with the choice given by Corollary~1.
\end{lemma}

The proof uses Lemma~4 (contractibility near edges) to extend the vertex choices to edges, then a standard contractibility argument for the faces.

With the conormal fibration in hand, Abouzaid defines \emph{global smoothing functions} on~$K$ and proves:

\begin{lemma}[Abouzaid, Lemma~9]
There exist smoothing functions for~$K$ of arbitrarily small $C^1$-norm.
\end{lemma}

The proof uses a partition of unity indexed by strata of~$K$, composed with dilations, together with the observation that the $C^1$ contribution of the partition's gradient is controlled by the $O(\epsilon^2)$ agreement of local smoothing functions along shared strata.

\subsection{Proof Assembly}

The final proof is short: the conormal fibration gives a neighbourhood~$\nu K$ of~$K$ in~$\R^4$ with a projection $\nu K \to K$.  Global smoothing functions of small $C^1$-norm produce smooth embedded graphical Lagrangians close to~$K$.  A sequence of such Lagrangians, connected by Hamiltonian paths (they are graphs of differentials over each other), concatenates into the desired isotopy.

\medskip

\noindent\textbf{Key features of the official solution:}
\begin{itemize}
\item The proof is \emph{coordinate-free} (after the initial normal form) and works with intrinsic objects (smoothing functions, conormal fibrations).
\item The local-to-global problem is solved by \emph{contractibility}---the space of admissible choices is contractible, so choices are automatically compatible.
\item No explicit Hamiltonian vector fields or flow equations are needed.
\item The proof generalizes to other symplectic manifolds in any dimension.
\end{itemize}


%======================================================================
\section{The \texttt{af} Automated Attempt}
\label{sec:af-attempt}
%======================================================================

\subsection{Overview and Evolution}

The \texttt{af} system worked on this problem over 7 adversarial sessions, evolving through three major strategic phases.  The proof tree had 9 nodes (root + 8 children), of which 2 were validated, 7 remained pending, and 84 challenges were resolved out of 90 total.

\subsection{Session 1--2: Tropical Resolution (Abandoned)}

The initial strategy attempted a hybrid approach combining Polterovich surgery with tropical resolution and Matessi--Mikhalkin pair-of-pants smoothing.  The first verification wave (Session~2) raised 53 challenges and revealed fundamental problems:
\begin{itemize}
\item The tropical resolution approach has issues for the 4-face vertex geometry.
\item The angular partition of unity is not $C^\infty$ at the origin.
\item The Matessi--Mikhalkin smoothing machinery does not directly apply.
\end{itemize}
This approach was abandoned.

\subsection{Sessions 3--4: Cotangent Generating Functions}

After resolving the 53 challenges from Session~2, the \texttt{af} system rewrote the vertex smoothing node (1.3) using \emph{cotangent generating functions}: choose a reference Lagrangian plane~$\Lambda$ transverse to all four face planes, represent each face as the graph of a symmetric matrix~$A_i$ in cotangent coordinates, and construct a smooth generating function by angular interpolation $F(X) = \sum_i \rho_i(\theta) \cdot \tfrac{1}{2} X^T A_i X$.

Session~4's verification wave found a critical flaw: the angular partition of unity at the origin produces a generating function $F(X) = r^2 g(\theta)$ whose smoothness at the origin requires $g(\theta)$ to have only Fourier modes $|n| \le 2$, but generic partitions $\rho_i$ introduce all even modes.  The ``Hessian'' $\sum_i \rho_i(\theta) A_i$ depends on $\theta$ and is not a well-defined bilinear form.  The function $F$ is not $C^2$ at the origin.

\subsection{Sessions 5--7: Two-Zone Construction (Final Strategy)}

The solution to the Fourier mode obstruction was the \textbf{two-zone construction}:
\begin{enumerate}
\item \textbf{Inner zone} ($|X| < \varepsilon$): Set $F_{\mathrm{smooth}} = 0$.  Trivially $C^\infty$.
\item \textbf{Outer zone} ($|X| > 2\varepsilon$): Use angular interpolation $F_{\mathrm{angular}}$.
\item \textbf{Transition} ($\varepsilon \le |X| \le 2\varepsilon$): Multiply by a flat cutoff $\chi(|X|/\varepsilon)$.
\end{enumerate}
The origin is never reached by the angular interpolation, so the Fourier mode obstruction is irrelevant.  The Hamiltonian isotopy is a ``shrinking inner-zone family'' expanding $\varepsilon$ from 0 to $\delta/8$.

For edge smoothing (Node~1.4), the \texttt{af} system found an elegant product construction: in edge-adapted Darboux coordinates, both faces lie in $\{y_1 = 0\}$, so replacing the V-shaped transverse profile with a smooth curve gives an \emph{automatically} Lagrangian surface.

For global assembly (Node~1.5), the strategy used overlapping domains (vertex balls and edge middles), overlap matching via the $\{y_1 = 0\}$ compatibility, and a two-phase Hamiltonian (Phase~A for vertices, Phase~B for edges with identity on $\gamma_1$).

\subsection{Final Status}

\begin{center}
\begin{tabular}{@{}llcc@{}}
\toprule
\textbf{Node} & \textbf{Content} & \textbf{Status} & \textbf{Challenges (open/total)} \\
\midrule
1.2 & Vertex classification & \textcolor{proved}{VALIDATED} & 3/11 \\
1.4 & Edge smoothing & \textcolor{proved}{VALIDATED} & 3/11 \\
1.1 & Strategy overview & \textcolor{pending}{Pending} & 0/0 \\
1.3 & Vertex smoothing (two-zone) & \textcolor{pending}{Pending} & 0/26 \\
1.5 & Global assembly & \textcolor{pending}{Pending} & 0/23 \\
1.6 & Hamiltonian isotopy & \textcolor{pending}{Pending} & 0/10 \\
1.7 & Topological extension & \textcolor{pending}{Pending} & 0/0 \\
1.8 & Obstruction analysis & \textcolor{pending}{Pending} & 0/11 \\
\bottomrule
\end{tabular}
\end{center}

Self-assessed confidence: 55--65\% that the proof is correct, 70--75\% that the answer is YES.


%======================================================================
\section{Critical Comparison}
\label{sec:comparison}
%======================================================================

\subsection{Where the Approaches Overlap}

\paragraph{Vertex normal form (strong overlap).}
Both the official solution and the \texttt{af} attempt begin with essentially the same linear algebra observation.  Abouzaid's Lemma~1 shows that any 4-face vertex configuration is symplectically equivalent to a standard normal form (product of positive coordinate axes).  The \texttt{af} Node~1.2 proves the same result in different language: all Type~A configurations are $\Sp(4,\R)$-equivalent to a normal form $\Pi_1 = \langle e_1, e_2\rangle$, $\Pi_2 = \langle e_1, e_4\rangle$, $\Pi_3 = \langle e_3, e_4\rangle$, $\Pi_4 = \langle e_2, e_3\rangle$.  The \texttt{af} version is more explicit (specifying coordinates) but contains the same essential content.  The \texttt{af} system additionally proves that ``Type~B'' (opposite sectors coplanar) is impossible for topological submanifolds, which Abouzaid handles implicitly through the assumption that $\Sigma$ is not contained in a plane.

\paragraph{Cotangent bundle perspective (partial overlap).}
Both approaches use the cotangent bundle structure: Abouzaid identifies $\Sigma$ with $L^\vee$ via the symplectic pairing and works with graphs of differentials; the \texttt{af} system works in cotangent coordinates $(X,Y) \in T^*\R^2$ with generating functions.  The mathematical content is closely related---smooth Lagrangians near a vertex correspond to smooth functions---but the conceptual framing differs significantly.  Abouzaid's \emph{smoothing functions} (Definition~2) are defined intrinsically on $\Sigma$ and are invariant under change of polarization (Lemma~2), while the \texttt{af} generating functions are tied to a specific coordinate system.

\paragraph{Edge smoothing (moderate overlap).}
The \texttt{af} Node~1.4 and Abouzaid's Lemma~4 both recognize that the edge geometry is simpler than the vertex geometry.  The \texttt{af} product construction ($L_e = \{(x_1, \gamma(s), 0, \gamma_2(s))\}$) is a concrete version of Abouzaid's observation that $\Sigma$ near an edge splits as a symplectic product of the real axis with a piecewise-linear curve.  However, Abouzaid's Lemma~4 extracts the essential abstract content---\emph{contractibility} of the space of admissible Lagrangian planes---rather than performing an explicit construction.

\subsection{Where the Approaches Diverge}

\paragraph{Smoothing functions vs.\ explicit generating functions.}
This is the most significant conceptual difference.  Abouzaid defines the space $\mathcal{S}(\Sigma)$ of smoothing functions as an \emph{intrinsic} object on $\Sigma$, independent of the choice of Lagrangian splitting (Lemma~2).  This abstraction is what makes the global argument work: smoothing functions can be manipulated using partitions of unity, and the $C^1$-norm can be controlled using the $O(\epsilon^2)$ first-order agreement of local smoothing functions along shared strata (Lemma~7).

The \texttt{af} system never discovers this concept.  Its generating functions are coordinate-dependent objects, and the smoothing is performed by explicit construction (inner zone $F = 0$, outer zone angular interpolation, cutoff transition).  This approach \emph{works locally} but creates enormous difficulties for the global assembly.

\paragraph{Conormal fibration vs.\ Hamiltonian flow composition.}
This is the decisive divergence.  Abouzaid's conormal fibration $\{L_z\}_{z \in K}$ is a global geometric object that simultaneously parametrizes all the local smoothings.  The key insight is that choosing $L_z$ is an \emph{obstruction-free} problem (the space of choices is contractible at every stage), so the global object always exists (Lemma~8).

The \texttt{af} system, by contrast, constructs local smoothings in coordinate patches and then attempts to glue them using a two-phase Hamiltonian composition: Phase~A smooths vertices, Phase~B smooths edges, and the overlap is handled by explicit compatibility arguments ($\{y_1 = 0\}$ compatibility, $x_1$-independence of angular interpolation, Phase~B identity on $\gamma_1$).  This is a \emph{brute-force} approach that works in principle but requires detailed coordinate computations and careful support control.

\paragraph{Partition of unity: intrinsic vs.\ extrinsic.}
Both approaches use partitions of unity, but in fundamentally different ways.  Abouzaid's partition (Lemma~7, Lemma~9) operates on \emph{smoothing functions on $\Sigma$}, which are scalar-valued functions.  The convex combination of smoothing functions is again a smoothing function (by linearity of the smoothing condition), and the $C^1$-norm is controlled by a clean estimate.

The \texttt{af} partition of unity operates on \emph{generating functions in ambient coordinates}, and the compatibility at overlaps requires showing that two different coordinate descriptions agree on their overlap.  This is where the \texttt{af} attempt is most vulnerable.

\subsection{Were the Dead Ends Genuine?}

The \texttt{af} system went through two major dead ends before arriving at the two-zone construction.

\paragraph{Dead End 1: Tropical resolution (Sessions 1--2).}
This was a genuine dead end.  The tropical-to-Lagrangian correspondence (Matessi, Mikhalkin, Hicks) is a powerful tool for constructing Lagrangians from tropical curves, but it operates in a different regime: it produces Lagrangians from tropical \emph{hypersurfaces}, not from polyhedral complexes with specified vertex conditions.  The \texttt{af} system correctly identified that this machinery does not directly apply.  The official solution makes no reference to tropical geometry.

\paragraph{Dead End 2: Angular partition of unity at the origin (Sessions 3--4).}
This was a genuine and instructive dead end.  The observation that $F(X) = r^2 g(\theta)$ requires $g$ to have only Fourier modes $|n| \le 2$ for $C^2$-smoothness at the origin is a real analytic obstruction.  The \texttt{af} adversarial process correctly identified this flaw.  Interestingly, Abouzaid's approach \emph{also} confronts this issue but solves it differently: the space of smoothing functions $\mathcal{S}(\Sigma)$ is defined to consist of functions $f$ such that $f + q^\Sigma$ is $C^\infty$, and the existence of such functions with small $C^1$-norm is proved by the dilation argument in Lemma~7.  The \texttt{af}'s two-zone construction (setting $F = 0$ in the inner zone) is a cruder but workable solution to the same underlying problem.


%======================================================================
\section{Evaluation of the Validated Nodes}
\label{sec:validated}
%======================================================================

\subsection{Node 1.2: Vertex Classification}

The \texttt{af} Node~1.2 proves:
\begin{enumerate}[label=(\alph*)]
\item The tangent cone at each vertex consists of 4 Lagrangian sectors in distinct planes.
\item Type~B (opposite sectors coplanar) is impossible for topological submanifolds.
\item All configurations are $\Sp(4,\R)$-equivalent to a unique normal form.
\end{enumerate}

\paragraph{Consistency with the official solution.}
This is \textbf{fully consistent} with and essentially equivalent to Abouzaid's Lemma~1.  The main difference is presentation:
\begin{itemize}
\item Abouzaid works with edge tangent vectors $v_1, v_2, u_1, u_2$ and shows they form a symplectic basis.
\item The \texttt{af} system works with face planes $\Pi_i$ and their intersections, classifying via the Lagrangian Grassmannian.
\end{itemize}
The mathematical content is the same.  The \texttt{af} version is more detailed (it explicitly addresses the Type~B impossibility and the 1-parameter family reduction), which is appropriate for an adversarial framework but unnecessary for the proof itself.

\paragraph{Assessment:} The 3 remaining open challenges (minor/note) concern expository issues (arc length justification, parabolic subgroup mention, $\pi_1(\LG(2,4))$ short exact sequence).  None affect the mathematical validity.  \textbf{This node is sound.}

\subsection{Node 1.4: Edge Smoothing}

The \texttt{af} Node~1.4 proves that in edge-adapted Darboux coordinates, both faces lie in $\{y_1 = 0\}$, and replacing the V-shaped transverse profile with a smooth curve gives an automatically Lagrangian surface.

\paragraph{Consistency with the official solution.}
This is \textbf{consistent} with Abouzaid's treatment of edges.  Abouzaid's Lemma~4 observes that $\Sigma$ near an edge splits as a product of the real axis with piecewise-linear Lagrangian in the transverse $\R^2$, which is precisely the content of the \texttt{af}'s $\{y_1 = 0\}$ observation.

However, the two approaches extract different conclusions:
\begin{itemize}
\item The \texttt{af} performs an \emph{explicit} smoothing (replace the V-profile with a smooth curve) and verifies the Lagrangian condition by computation.
\item Abouzaid proves that the space of admissible Lagrangian planes~$L$ near an edge is \emph{contractible} (Lemma~4), which feeds into the global conormal fibration argument.
\end{itemize}

The \texttt{af}'s explicit construction is correct as a local result.  The difficulty is that it does not naturally compose with the vertex smoothing into a global object---precisely the local-to-global gap.

\paragraph{Assessment:} The 3 remaining open challenges (minor) concern compact support, the Hamiltonian property, and terminology.  \textbf{This node is sound as a local result.}


%======================================================================
\section{The Local-to-Global Gap}
\label{sec:gap}
%======================================================================

The authors' commentary on AI-generated solutions identifies the local-to-global gluing argument as the key failure point:

\begin{quote}
``The proof then proceeds to perform a local-to-global gluing argument.  It was a priori clear that there must be a gap in this argument because the LLM solution refers to the existence of a linear symplectic transformation that brings a neighbourhood of each vertex and each edge into a standard position, but fails to discuss the compatibility between these choices.''
\end{quote}

Two specific errors were identified in the best AI solutions:
\begin{enumerate}
\item One solution ``asserted that one can choose disjoint neighbourhoods of the edges and of the vertices.''
\item Another ``performs a local move near vertices, which changes the local geometry near the edges, invalidating the application of the edge move.''
\end{enumerate}

\subsection{How the \texttt{af} Attempt Addresses This}

The \texttt{af} system's Node~1.5 (global assembly) directly confronts the local-to-global problem.  Its strategy is:
\begin{enumerate}
\item Use overlapping (not disjoint) domains: vertex balls $V_i = B_{\delta_i + \eta}(v_i)$ overlap edge middles $E_k$ by width $\eta$.
\item Argue that the vertex smoothing (angular interpolation) and edge smoothing (constant profile) agree in the overlap, because both lie in $\{y_1 = 0\}$.
\item Use a two-phase sequential Hamiltonian: Phase~A (vertices) then Phase~B (edges), with Phase~B designed to be the identity on the already-smoothed profile $\gamma_1$.
\end{enumerate}

This is a more sophisticated attempt than the naive ``choose disjoint neighbourhoods'' approach identified as Error~1 in the commentary.  The \texttt{af} system explicitly uses overlapping domains and argues for compatibility.  It also avoids Error~2 (local vertex move invalidating edge geometry) by designing Phase~B to be the identity where Phase~A has already acted.

\subsection{Does the \texttt{af} Approach Actually Work?}

The \texttt{af} approach has a plausible structure, but several issues remain:

\paragraph{Coordinate compatibility.}
The overlap matching requires that the vertex construction's angular interpolation, expressed in vertex cotangent coordinates, agrees with the edge construction's product Lagrangian, expressed in edge-adapted Darboux coordinates.  The \texttt{af} claims this follows from $\{y_1 = 0\}$ compatibility, but the coordinate change between vertex and edge coordinates may mix $y_1$ with other coordinates.  The \texttt{af} report itself identifies this as the primary risk (Node~1.5, Step~5).

\paragraph{Phase~B support control.}
The claim that Phase~B's Hamiltonian can be supported away from $\gamma_1$ (the already-smoothed profile) requires careful analysis of the geometry.  If $\gamma_1$ passes through the region where the V-shape needs to be smoothed, the support cannot be made disjoint.

\paragraph{Contrast with the official solution.}
Abouzaid's proof avoids all of these issues by construction.  The conormal fibration $\{L_z\}$ is a \emph{global} object whose existence is guaranteed by contractibility arguments.  There is no need to check compatibility of coordinate changes, because the smoothing functions are defined intrinsically on $K$.  The partition of unity operates on scalar functions, not on ambient Lagrangians.  The authors' commentary notes:

\begin{quote}
``The errors in these solutions can be repaired at the cost of significant computations of changes of coordinates, which would become extremely burdensome in any generalisation.  The point of the solution we provide is to obtain a proof which avoids (most of) the hard work, and which experts can readily generalise to other symplectic manifolds (in any dimension).''
\end{quote}

This strongly suggests that the \texttt{af} approach \emph{can} be made to work in principle (``the errors can be repaired''), but requires substantial additional computation that the \texttt{af} system did not carry out.  The coordinate compatibility checks at overlaps would require explicit formulas for the coordinate changes, which grow in complexity with the number of faces and edges.

\subsection{The Fundamental Conceptual Gap}

The deepest insight of Abouzaid's proof is the \emph{smoothing function} concept: the condition for a Lagrangian to smooth~$\Sigma$ is encoded in a \emph{linear} condition on scalar functions ($f + q^\Sigma \in C^\infty$), and this linear space is invariant under addition of smooth functions.  This linearity is what makes the partition of unity argument work cleanly.

The \texttt{af} system never discovers this linearization.  Its generating functions are nonlinear objects (quadratic forms patched together), and the smoothing is performed by explicit geometric construction.  The absence of the smoothing function concept forces the \texttt{af} system into coordinate-dependent arguments that become increasingly difficult to manage globally.

This is not merely a matter of elegance.  The smoothing function framework provides:
\begin{enumerate}
\item \textbf{Polarization independence} (Lemma~2): the space $\mathcal{S}(\Sigma)$ depends only on $L$, not on the splitting.
\item \textbf{Partition of unity compatibility}: smoothing functions can be combined via partitions of unity because the smoothing condition is affine.
\item \textbf{$C^1$-norm control} (Lemma~7): the dilation trick gives smoothing functions of arbitrarily small norm.
\item \textbf{Dimension-free generalization}: the argument extends to higher dimensions without additional computation.
\end{enumerate}

None of these properties are available to the \texttt{af} approach.


%======================================================================
\section{Lessons Learned}
\label{sec:lessons}
%======================================================================

\subsection{What the \texttt{af} System Did Well}

\begin{enumerate}
\item \textbf{Correct local analysis.}  The vertex normal form (Node~1.2) and edge product structure (Node~1.4) are mathematically correct and consistent with the official solution.  The linear algebra underlying the local smoothing is essentially the same.

\item \textbf{Effective adversarial process.}  The adversarial framework successfully identified genuine errors: the Fourier mode obstruction at the origin (Sessions~3--4) and the tropical resolution dead end (Sessions~1--2).  The system's ability to detect and repair flaws led to progressively stronger proof attempts.

\item \textbf{Reasonable strategic pivots.}  The evolution from tropical resolution to cotangent generating functions to the two-zone construction shows adaptive problem-solving.  Each pivot was motivated by a concrete failure identified by the adversarial process.

\item \textbf{Honest self-assessment.}  The final confidence of 55--65\% for proof correctness (70--75\% for the answer) is realistic.  The system correctly identified the global assembly as the weakest point.
\end{enumerate}

\subsection{What the \texttt{af} System Missed}

\begin{enumerate}
\item \textbf{The smoothing function concept.}  The key abstraction---that smooth Lagrangians near a singular vertex correspond to elements of a \emph{linear} space of scalar functions---was not discovered.  This is arguably the central mathematical insight of the official solution.

\item \textbf{Contractibility as a gluing tool.}  Abouzaid's proof repeatedly uses contractibility of spaces of choices (Lagrangian planes near edges in Lemma~4, Lagrangian lifts in the proof of Lemma~4, extensions to face interiors in Lemma~8) to solve the compatibility problem.  The \texttt{af} system uses explicit constructions and support arguments instead, which is harder and less general.

\item \textbf{The conormal fibration.}  The global geometric object $\{L_z\}_{z \in K}$ that simultaneously parametrizes all local smoothings was not conceived.  This is the device that transforms the local-to-global problem from a coordinate-compatibility nightmare into a clean existence argument.

\item \textbf{Intrinsic vs.\ extrinsic perspective.}  The \texttt{af} system works entirely in ambient coordinates, while Abouzaid works with intrinsic objects on $K$ (smoothing functions, conormal fibrations, the smooth structure inherited from $L^\vee$).  The intrinsic perspective is what makes the global argument tractable.
\end{enumerate}

\subsection{Broader Implications}

\paragraph{Local arguments are within reach of AI.}
The \texttt{af} system's success on the local analysis (vertex normal form, edge product structure) suggests that AI systems can handle problems that reduce to linear algebra and explicit computation in coordinates.  These are essentially ``finite-dimensional'' arguments.

\paragraph{Global geometric arguments remain difficult.}
The failure on the global assembly reflects a deeper challenge: the official solution requires inventing a \emph{new concept} (smoothing functions, conormal fibrations) that reframes the problem.  AI systems tend to work within existing frameworks rather than creating new abstractions.  The conormal fibration is not a standard textbook construction---it is a new idea tailored to this specific problem.

\paragraph{The adversarial process has value but cannot substitute for insight.}
The adversarial framework was effective at detecting errors and forcing repairs, but it could not generate the key conceptual innovation.  The progression tropical resolution $\to$ generating functions $\to$ two-zone construction represents increasingly sophisticated \emph{implementations} of the same basic strategy (local smoothing + Hamiltonian gluing), not a fundamental rethinking of the approach.

\paragraph{Coordinate-free thinking matters.}
The contrast between the \texttt{af} approach (coordinate-dependent, explicit) and Abouzaid's approach (coordinate-free, conceptual) illustrates a broader point: mathematical proofs that work in coordinates often cannot scale (to higher dimensions, more general settings, or more complex configurations), while proofs based on structural properties (contractibility, linearity, functoriality) tend to generalize.  AI systems may benefit from being steered toward coordinate-free formulations.


%======================================================================
\section{Summary}
\label{sec:summary}
%======================================================================

\begin{center}
\begin{tabular}{@{}p{4.5cm}p{5cm}p{5cm}@{}}
\toprule
\textbf{Aspect} & \textbf{\texttt{af} Attempt} & \textbf{Official Solution} \\
\midrule
Vertex normal form & $\Sp(4,\R)$ classification (correct) & Linear symplectic basis (equivalent) \\[4pt]
Local smoothing concept & Generating functions, two-zone & Smoothing functions $\mathcal{S}(\Sigma)$ \\[4pt]
Edge treatment & Product Lagrangian (correct) & Contractibility of choices \\[4pt]
Local-to-global strategy & Two-phase Hamiltonian, overlap matching & Conormal fibration \\[4pt]
Key tool for gluing & Coordinate compatibility & Contractibility \\[4pt]
Partition of unity target & Generating functions (nonlinear) & Smoothing functions (linear) \\[4pt]
Coordinate dependence & Heavy & Minimal (intrinsic) \\[4pt]
Generalizability & Limited to $\R^4$ & Any symplectic manifold \\[4pt]
Status & 2/9 nodes validated, 55--65\% & Complete proof \\
\bottomrule
\end{tabular}
\end{center}

\medskip

The \texttt{af} automated attempt correctly solved the local part of the problem (vertex classification and edge smoothing) using essentially the same linear algebra as the official solution.  It also correctly identified the answer as YES and made reasonable strategic adjustments when dead ends were encountered.  However, it did not discover the key conceptual innovations---smoothing functions and conormal fibrations---that make Abouzaid's global argument work.  The local-to-global gap identified by the authors' commentary is precisely the point where the \texttt{af} system's coordinate-dependent approach becomes inadequate, and where Abouzaid's intrinsic, contractibility-based approach succeeds.

The authors' observation that ``the errors in these solutions can be repaired at the cost of significant computations of changes of coordinates'' applies directly to the \texttt{af} attempt: the two-zone construction and overlap matching strategy is not fundamentally wrong, but completing it would require extensive coordinate calculations that the system did not (and perhaps could not) perform.  The official solution shows that the right abstraction can eliminate this computational burden entirely.


\begin{thebibliography}{99}

\bibitem{Abouzaid26}
M.~Abouzaid,
\emph{First Proof: Solutions and Comments} (Problem~8),
February 2026.

\bibitem{MS17}
D.~McDuff and D.~Salamon,
\emph{Introduction to Symplectic Topology},
3rd ed., Oxford University Press, 2017.

\bibitem{Hirsch76}
M.~W.~Hirsch,
\emph{Differential Topology},
Graduate Texts in Mathematics, Springer, 1976.

\bibitem{Matessi19}
D.~Matessi,
\emph{Lagrangian pairs of pants},
Int.\ Math.\ Res.\ Not.\ \textbf{2019}, no.~15, 4683--4723.

\bibitem{Mikhalkin19}
G.~Mikhalkin,
\emph{Examples of tropical-to-Lagrangian correspondence},
European J.\ Math.\ \textbf{5} (2019), 1033--1066.

\bibitem{Hicks20}
J.~Hicks,
\emph{Tropical Lagrangian hypersurfaces are unobstructed},
J.\ Topol.\ \textbf{13} (2020), 1409--1454.

\bibitem{Pol91}
L.~Polterovich,
\emph{The surgery of Lagrange submanifolds},
Geom.\ Funct.\ Anal.\ \textbf{1} (1991), 198--210.

\end{thebibliography}

\end{document}
