\documentclass[11pt,a4paper]{article}

% --- Packages ---
\usepackage[utf8]{inputenc}
\usepackage[T1]{fontenc}
\usepackage{lmodern}
\usepackage[margin=2.5cm]{geometry}
\usepackage{amsmath,amssymb,amsthm}
\usepackage{mathtools}
\usepackage{enumitem}
\usepackage{booktabs}
\usepackage{array}
\usepackage{xcolor}
\usepackage{hyperref}

% --- Theorem environments ---
\newtheorem{theorem}{Theorem}[section]
\newtheorem{lemma}[theorem]{Lemma}
\newtheorem{proposition}[theorem]{Proposition}
\newtheorem{corollary}[theorem]{Corollary}
\theoremstyle{definition}
\newtheorem{definition}[theorem]{Definition}
\newtheorem{remark}[theorem]{Remark}

% --- Macros ---
\newcommand{\Z}{\mathbb{Z}}
\newcommand{\C}{\mathbb{C}}
\newcommand{\fro}{\mathfrak{o}}
\newcommand{\frp}{\mathfrak{p}}
\newcommand{\frq}{\mathfrak{q}}
\newcommand{\GL}{\mathrm{GL}}
\newcommand{\Ind}{\mathrm{Ind}}
\DeclareMathOperator{\vol}{vol}
\DeclareMathOperator{\Wh}{\mathcal{W}}
\DeclareMathOperator{\trace}{trace}
\newcommand{\eps}{\varepsilon}

% --- Title ---
\title{\textbf{Comparative Analysis: Official Solution vs.\ Adversarial Proof Attempt\\for Problem~2 (Whittaker Functions for Rankin--Selberg Integrals)}\\[6pt]
\large First Proof Benchmark}
\author{Automated Analysis\\First Proof Project}
\date{February 2026}

\begin{document}
\maketitle

\begin{abstract}
We critically compare the official solution to Problem~2 of the First Proof benchmark
(posed by Paul D.\ Nelson) with an automated adversarial proof (\texttt{af}) attempt.
The problem asks whether, for a generic irreducible admissible representation $\Pi$ of
$\GL_{n+1}(F)$ over a non-archimedean local field, there exists a single Whittaker
function $W$ such that for every generic $\pi$ of $\GL_n(F)$, the local Rankin--Selberg
integral is finite and nonzero for all $s \in \C$.  Both approaches answer \textbf{YES}.
The official solution uses the Godement--Jacquet functional equation to relate the
unipotent-shifted Rankin--Selberg integral to a $K_1(\frq)$-integral of the newvector,
obtaining an explicit constant in two pages.  The \texttt{af} attempt instead pursued a
three-case strategy (unramified, supercuspidal, non-supercuspidal ramified) via Iwasawa
unfolding and Kirillov model support arguments, accumulating 57 open challenges over 17
nodes without reaching a complete proof.  We identify which of the five systematic
issues found by \texttt{af} were genuine mathematical obstacles versus artifacts of the
wrong proof strategy, and extract lessons for future automated theorem proving in
representation theory.
\end{abstract}

\tableofcontents
\newpage

%======================================================================
\section{Problem Statement}
\label{sec:problem}
%======================================================================

Let $F$ be a non-archimedean local field with ring of integers $\fro$, maximal ideal
$\frp$, and residue field cardinality $q$.  Let $\psi : F \to \C^\times$ be a nontrivial
additive character of conductor $\fro$.  Set $G_r := \GL_r(F)$, $K_r := \GL_r(\fro)$,
and let $N_r < G_r$ denote the upper-triangular unipotent subgroup.

Let $\Pi$ be a generic irreducible admissible representation of $G_{n+1}$, realized in
its $\psi^{-1}$-Whittaker model $\Wh(\Pi, \psi^{-1})$.

\begin{theorem}[Nelson]
\label{thm:main}
There exists $W \in \Wh(\Pi, \psi^{-1})$ with the following property.  For every generic
irreducible admissible $\pi$ of $G_n$ with conductor ideal $\frq$ and generator
$Q \in F^\times$ of $\frq^{-1}$, setting $u_Q := I_{n+1} + Q\, E_{n,n+1}$, there exists
$V \in \Wh(\pi, \psi)$ such that the local Rankin--Selberg integral
\[
  \ell_{\mathrm{RS}}(s, W, V) :=
  \int_{N_n \backslash G_n} W\bigl(\mathrm{diag}(g,1) \cdot u_Q\bigr)\, V(g)\,
  |\det g|^{s - 1/2}\, dg
\]
is finite and nonzero for all $s \in \C$.
\end{theorem}

The critical subtlety, emphasized in the authors' commentary, is that $W$ must be
\emph{independent of} $\pi$.  Without this universality requirement, the problem would be
much easier and well-known.

%======================================================================
\section{The Official Solution}
\label{sec:official}
%======================================================================

\subsection{Overview}

The official solution (Nelson, with Jana) is remarkably compact---roughly two pages of
mathematics---and proceeds via a single unified argument that avoids any case splitting by
the type of $\pi$.  The key ideas are:

\begin{enumerate}[label=(\roman*)]
\item \textbf{Choice of $W$:} Define $W_0 \in \Wh(\Pi, \psi^{-1})$ via the Kirillov
  model extension of the function
  \[
    W_0(g) := \int_{N_n} \mathbf{1}_{K_n}(xg)\, \psi(x)\, dx
  \]
  on $G_n$, extended to $G_{n+1}$ by the theory of the Kirillov model (Bernstein).
  The actual test vector for $\Pi$ is $u_Q W_0$ (a unipotent translate of $W_0$).

\item \textbf{Choice of $V$:} Take $V$ to be the normalized newvector of $\pi$---the
  unique $K_1(\frq)$-invariant vector with $V(1) = 1$.

\item \textbf{Godement--Jacquet bridge:} Apply the Godement--Jacquet local functional
  equation (Lemma~2 in the solution) with a carefully chosen Schwartz function
  $\phi \in \mathcal{S}(M_n(F))$ and a function $\beta \in \mathcal{S}^e(F^\times)$
  (depending on $\pi$ only through its conductor) to transform the Rankin--Selberg
  integral into a $K_1(\frq)$-integral.

\item \textbf{Reduction to newvector theory:} The transformed integral evaluates to
  $|Q|^n \cdot \vol(K_1(\frq))$ by $K_1(\frq)$-invariance of $V$ and the normalization
  $V(1) = 1$.
\end{enumerate}

\subsection{Key Technical Ingredients}

\paragraph{The Schwartz function $\phi$ and the weight $\beta$.}
The solution defines $\phi_0 = \mathbf{1}_{M_n(\fro)}$ and
\[
  \phi(x) := \psi(-x_{nn})\, \phi_0(x\, d_Q^{-1}),
\]
where $d_Q = \mathrm{diag}(1, \ldots, 1, Q)$.  The function $\beta$ is defined via
Mellin inversion:
\[
  \widetilde{\beta}(s) := \frac{\eps(\tfrac{1}{2}+s, \pi, \psi)}{L(\tfrac{1}{2}+s, \pi)},
\]
and has the crucial properties that $\beta$ is supported on
$\{y : |Q| \leq |y| \leq |Q| q^n\}$ while $\beta^\sharp$ (the transform of $\beta$
through the gamma factor) is supported on $\{y : 1 \leq |y| \leq q^n\}$ and takes the
value $1$ on $\fro^\times$.

\paragraph{The Godement--Jacquet identity (Lemma~4).}
For unitary generic $\pi$ and $f = V$ (a Whittaker function treated as a generalized
matrix coefficient), the identity
\[
  \int_{G_n} \phi(g)\, f(g)\, \beta(\det g)\, |\det g|^{n/2}\, dg
  =
  \int_{G_n} \phi^\wedge(g)\, f^\vee(g)\, \beta^\sharp(\det g)\, |\det g|^{n/2}\, dg
\]
holds with absolute convergence on both sides.  This is established by inserting the
Mellin expansion of $\beta$, recognizing the Godement--Jacquet zeta integral, applying
the local functional equation, and re-expanding.

\paragraph{The factorization identity (Lemma~5).}
Two identities link the Schwartz data $(\phi, \beta)$ to the Rankin--Selberg integral on
the left side and to the congruence subgroup on the right:
\begin{align}
  \text{(Left):} &\quad \int_{N_n} \beta(\det xg)\, \phi(xg)\, \psi(x)\, dx
    = \eps(\tfrac{1}{2}, \pi, \psi)\, W_0(g\, t_Q), \label{eq:left} \\
  \text{(Right):} &\quad \beta^\sharp(\det g)\, \phi^\wedge(g)
    = |Q|^n\, \mathbf{1}_{K_1(\frq)}(g). \label{eq:right}
\end{align}

\paragraph{The final computation.}
Combining these, the Rankin--Selberg integral evaluates to:
\[
  \ell_{\mathrm{RS}}(s, u_Q W_0, d_Q V)
  = \eps(\tfrac{1}{2}, \pi, \psi)^{-1}\, |Q|^{n - n/2}\,
    \vol(K_1(\frq)),
\]
which is a \emph{nonzero constant} (independent of $s$).

\subsection{Why the Solution Is Elegant}

Several features are worth highlighting:
\begin{itemize}
\item \textbf{No case splitting:} The argument works uniformly for all $\pi$---unramified,
  supercuspidal, and non-supercuspidal ramified---because the Godement--Jacquet functional
  equation is universal.
\item \textbf{The $s$-independence is immediate:} Since $W_0$ is supported on
  $\det^{-1}(\fro^\times)$, the translate $t_Q W_0$ is supported on
  $\det^{-1}(Q\fro^\times)$, so the torus sum collapses to a single determinant value
  automatically.  This makes $\ell_{\mathrm{RS}}(s)$ a constant times $|Q|^s$, and the
  homogeneity property (equation (19) in the solution) absorbs this into a constant.
\item \textbf{Newvector theory does all the work:} The right-hand side of the
  Godement--Jacquet identity (after applying \eqref{eq:right}) reduces to an integral
  over $K_1(\frq)$, where $K_1(\frq)$-invariance and the normalization $V(1) = 1$
  immediately give a computable nonzero answer.
\item \textbf{The choice of $W_0$ is explicit and canonical:} It is the Whittaker
  function corresponding to $\mathbf{1}_{K_n}$ in the Kirillov model.  This is
  \emph{not} the newvector of $\Pi$ (unless $\Pi$ is unramified), contrary to what the
  \texttt{af} attempt assumed.
\end{itemize}


%======================================================================
\section{The \texttt{af} Automated Attempt}
\label{sec:af}
%======================================================================

\subsection{Answer and Strategy}

The \texttt{af} attempt correctly identified the answer as \textbf{YES} and proposed that
the essential Whittaker function (newvector) $W^\circ$ of $\Pi$ works universally.  The
proof was organized as a 17-node tree with the following high-level structure:

\begin{enumerate}
\item \textbf{Algebraic reduction} (Nodes 1.1, 1.2): Establish the commutation identity
  $W(\mathrm{diag}(g,1)\, u_Q) = \psi^{-1}(Q g_{nn})\, W(\mathrm{diag}(g,1))$ and the
  monomial characterization (a rational function in $q^{-s}$ with no zeros or poles on
  $\C^\times$ is a monomial).

\item \textbf{Three-case strategy:} The choice of test vector $V$ depends on $\pi$:
  \begin{itemize}
  \item \emph{Unramified $\pi$:} Construct $V_0$ with compact Kirillov model support
    $\phi_0 = \mathbf{1}_{(\fro^\times)^{n-1}}$, aiming to collapse the torus sum to a
    single term.
  \item \emph{Supercuspidal ramified $\pi$:} Use the newvector $V^\circ$; argue that
    compact Kirillov support of supercuspidal representations forces the torus sum to
    collapse.
  \item \emph{Non-supercuspidal ramified $\pi$:} Use $V^\circ$ and attempt to identify
    the integral with the Rankin--Selberg epsilon factor $\eps(s, \Pi \times \pi, \psi)$.
  \end{itemize}

\item \textbf{Iwasawa unfolding} (Node 1.4): For ramified $\pi$, unfold via $g = nak$
  and exploit $W^\circ$-factorization and conductor analysis to show $J_K(m) = 0$ unless
  $m_n = 0$.

\item \textbf{Nonvanishing} (Node 1.5): Argue that the surviving terms are nonzero.
\end{enumerate}

\subsection{Status at Termination}

After three sessions, the proof had:
\begin{itemize}
\item 5 validated nodes (1.1, 1.2, 1.4.1, 1.4.2, plus the root by propagation),
\item 12 pending nodes with 57 open challenges (18 critical, 19 major, 15 minor),
\item 0 refutations.
\end{itemize}

The five systematic issues identified were:
\begin{enumerate}
\item Kirillov model evaluation for $n \geq 2$ (pointwise evaluation off the mirabolic),
\item Casselman--Shalika formula scope (unramified $\Pi$ only),
\item $K$-projection dilemma (unramified case),
\item Test vector theory for general $\GL_{n+1} \times \GL_n$,
\item $J_K(0)$ nonvanishing for $n \geq 2$.
\end{enumerate}


%======================================================================
\section{Critical Comparison}
\label{sec:comparison}
%======================================================================

\subsection{The Fundamental Strategic Divergence}

The most important difference between the two approaches is \textbf{structural}: the
official solution avoids case splitting entirely, while the \texttt{af} attempt committed
to a three-case strategy that created three independent proof obligations, each of which
proved intractable.

The official solution achieves this by using the Godement--Jacquet functional equation as
a \emph{bridge} between the Rankin--Selberg world (where the integral lives) and the
congruence subgroup world (where the newvector is well-understood).  The key insight is
that one can choose Schwartz data $(\phi, \beta)$ that simultaneously:
\begin{itemize}
\item match the Whittaker function $W_0(g\, t_Q)$ on the Rankin--Selberg side
  (via identity \eqref{eq:left}), and
\item reduce to the indicator function $\mathbf{1}_{K_1(\frq)}$ on the dual side
  (via identity \eqref{eq:right}).
\end{itemize}

The \texttt{af} attempt, by contrast, tried to analyze the Rankin--Selberg integral
\emph{directly} via the Iwasawa decomposition and Kirillov/Whittaker model support.  This
``direct analysis'' approach works for $n = 1$ (where it reduces to Gauss sums) but
becomes increasingly unwieldy for $n \geq 2$ because:
\begin{itemize}
\item The torus sum has $n$ indices, and collapsing it to a single term requires
  controlling the support of both $W^\circ$ and $V$ simultaneously.
\item The Kirillov model, while powerful for $P_n$-orbits, does not give pointwise
  evaluation for elements outside the mirabolic subgroup.
\item Different types of $\pi$ (unramified, supercuspidal, non-supercuspidal ramified)
  have qualitatively different Kirillov/Whittaker support, necessitating case splits.
\end{itemize}

\subsection{Choice of $W$: Newvector vs.\ Kirillov Extension of $\mathbf{1}_{K_n}$}

A second major divergence concerns the choice of $W$ itself:

\begin{center}
\begin{tabular}{@{}lp{5.5cm}p{5.5cm}@{}}
\toprule
& \textbf{Official solution} & \textbf{\texttt{af} attempt} \\
\midrule
$W$ & $W_0$ defined via the Kirillov model extension of $g \mapsto \int_{N_n} \mathbf{1}_{K_n}(xg)\, \psi(x)\, dx$ & $W^\circ$, the newvector (essential Whittaker function) of $\Pi$ \\[4pt]
Dependence on $\Pi$ & Only through the Kirillov model embedding $\Wh(\pi_n, \psi) \hookrightarrow \Wh(\Pi, \psi^{-1})$ & Through the conductor $c(\Pi)$ and the $K_1(\frp^{c(\Pi)})$-invariance \\[4pt]
Key property & Supported on $\det^{-1}(\fro^\times)$ when restricted to $G_n$ & $K_1(\frp^{c(\Pi)})$-invariant \\
\bottomrule
\end{tabular}
\end{center}

The \texttt{af} choice of $W = W^\circ$ is not \emph{wrong}---both choices produce valid
Whittaker functions.  However, $W^\circ$ lacks the determinantal support property that
makes the official argument work: since $W^\circ$ is not necessarily supported on
$\det^{-1}(\fro^\times)$ when restricted to $G_n$, the torus sum does not automatically
collapse, and one is forced into the case-by-case analysis that generated all 57
challenges.

The official choice of $W_0$ is specifically engineered so that its translate $t_Q W_0$ is
supported on $\det^{-1}(Q\fro^\times)$, which immediately pins the determinant and
eliminates all but one term in the torus sum.  This is the single most important technical
idea that the \texttt{af} attempt missed.

\subsection{Choice of $V$: Universal Newvector vs.\ Case-Dependent Construction}

\begin{center}
\begin{tabular}{@{}lp{5.5cm}p{5.5cm}@{}}
\toprule
& \textbf{Official solution} & \textbf{\texttt{af} attempt} \\
\midrule
$V$ & Always $d_Q V^\circ$, where $V^\circ$ is the normalized newvector of $\pi$ &
  $V^\circ$ for ramified $\pi$; a compact-Kirillov-support $V_0$ for unramified $\pi$ \\[4pt]
Universality & Uniform across all $\pi$ (with the natural $d_Q$-scaling) &
  Two different constructions depending on $c(\pi)$ \\
\bottomrule
\end{tabular}
\end{center}

The official solution's choice of $V = d_Q V^\circ$ is arguably more natural: the
$d_Q$-twist accounts for the conductor in the same way that the $u_Q$-shift accounts for
it on the $\Pi$-side.  The \texttt{af} attempt's construction of a special $V_0$ for the
unramified case was motivated by the need to avoid $L$-function poles in the torus sum---a
problem that simply does not arise in the official approach because the torus sum is
already collapsed by the support of $W_0$.

\subsection{Role of the Godement--Jacquet Functional Equation}

The Godement--Jacquet theory plays no role in the \texttt{af} attempt.  This is the
single biggest missed opportunity.  The official solution uses it as the central engine:
\begin{enumerate}
\item The functional equation relates $Z(\phi, f, s)$ to $Z(\phi^\wedge, f^\vee, 1-s)$.
\item With the choice $f = V$ (a Whittaker function, not just a matrix coefficient), and
  the Schwartz data $(\phi, \beta)$ from Lemma~5, the left side becomes
  $\eps(\frac{1}{2}, \pi, \psi) \cdot \ell_{\mathrm{RS}}(\frac{n+1}{2}, t_Q W_0, V)$
  (after unfolding).
\item The right side becomes $|Q|^n \int_{K_1(\frq)} V(g^{-1})\, dg =
  |Q|^n \vol(K_1(\frq))$ (by \eqref{eq:right} and newvector properties).
\end{enumerate}

The \texttt{af} attempt's Node~1.5.2 \emph{gestured} toward using the JPSS functional
equation to establish the monomial property, but this was framed as an identification
$I(s) = C \cdot \eps(s, \Pi \times \pi, \psi)$ via test vector theory---a much harder
claim that requires knowing the precise proportionality constant, which the
\texttt{af} attempt could not establish.  The official solution sidesteps this entirely
by working with the Godement--Jacquet functional equation (for $\pi$ alone, not for the
pair $\Pi \times \pi$), which is a simpler and more classical tool.


%======================================================================
\section{Assessment of the Five Systematic Issues}
\label{sec:issues}
%======================================================================

We now evaluate each of the five systematic issues identified by the \texttt{af} attempt,
distinguishing between genuine mathematical obstacles and artifacts of the proof strategy.

\subsection{Issue 1: Kirillov Model Evaluation for $n \geq 2$}

\textbf{Verdict: Artifact of the wrong approach.}

The \texttt{af} attempt needed pointwise evaluation of $V_0(ak)$ for $k \in K_n$ not in
the mirabolic subgroup $P_n$, which the Kirillov model does not provide.  The official
solution avoids this entirely: $V$ is always the newvector $V^\circ$, which is defined
globally on $G_n$ (not just on $P_n$), and the Godement--Jacquet approach reduces
everything to an integral over $K_1(\frq)$ where $V^\circ$ is simply a constant.

However, the underlying mathematical issue---that the Kirillov model is a
$P_n$-representation, not a $G_n$-representation---is a genuine fact that any proof using
Kirillov-model support arguments for $G_n$-integrals must contend with.  The
\texttt{af} attempt correctly identified this as a problem; it simply should not have been
using the Kirillov model in this way.

\subsection{Issue 2: Casselman--Shalika Formula Scope}

\textbf{Verdict: Artifact of the wrong approach.}

The \texttt{af} attempt invoked the Casselman--Shalika formula to control the torus
support of $W^\circ$ (the newvector of $\Pi$), but this formula applies only when $\Pi$
is unramified.  The official solution uses $W_0$ (not $W^\circ$), whose torus support is
controlled not by Casselman--Shalika but by the elementary fact that
$\mathbf{1}_{K_n}(xg)$ forces $g \in K_n$ (after integrating over $N_n$), giving
$\det g \in \fro^\times$.

\subsection{Issue 3: $K$-Projection Dilemma}

\textbf{Verdict: Genuine obstacle for the Iwasawa-unfolding approach, but irrelevant to
the correct proof.}

For unramified $\pi$, the $K_n$-average of any Whittaker function is either proportional
to the spherical vector (reproducing $L$-function poles) or zero.  This is a real
phenomenon---it reflects the fact that the spherical vector generates the unique
$K_n$-fixed line in $\pi$.  The \texttt{af} attempt correctly identified this as a
dilemma for the unramified case.

The official solution resolves this ``dilemma'' implicitly: by using $V = d_Q V^\circ$
and working through the Godement--Jacquet functional equation rather than through a $K_n$-average, the issue simply does not arise.  The integral over $K_1(\frq)$
(which equals $K_n$ when $\pi$ is unramified, since $\frq = \fro$) yields
$\vol(K_n) \cdot V^\circ(1) = 1$---there is no projection dilemma because $V^\circ$
\emph{is} $K_n$-fixed.

\subsection{Issue 4: Test Vector Theory for General $\GL_{n+1} \times \GL_n$}

\textbf{Verdict: Genuine gap in the literature, but irrelevant to the correct proof.}

The \texttt{af} attempt tried to identify the Rankin--Selberg integral with
$C \cdot \eps(s, \Pi \times \pi, \psi)$, invoking Humphries (2021, $\GL_2$-specific) and
Assing--Blomer (2024, does not cover the general case).  This is indeed an open problem in
the test vector literature: the precise proportionality constant for new vectors in
$\GL_{n+1} \times \GL_n$ Rankin--Selberg integrals is not fully established for
general~$n$.

However, the official solution does not need this.  It computes the integral
\emph{directly} as $c \cdot |Q|^{-n/2}$ using the Godement--Jacquet bridge, without ever
identifying it with an epsilon factor of the pair $\Pi \times \pi$.  The epsilon factor of
$\pi$ alone (not of the pair) appears in the Schwartz function $\beta$, but this is
elementary (Godement--Jacquet theory for a single representation).

\subsection{Issue 5: $J_K(0)$ Nonvanishing for $n \geq 2$}

\textbf{Verdict: Genuine mathematical question, partially relevant.}

The \texttt{af} attempt needed to show that
$\sum_{x \in (\fro/\frq)^\times} \psi^{-1}(Qx)\, \Phi(x) \neq 0$, where $\Phi$ is a
fiber sum of the newvector over $\GL_n(\fro/\frq)$.  This is related to the nonvanishing
of certain character sums on finite groups of Lie type, which is a non-trivial question.

The official solution encounters a related but much simpler nonvanishing: it reduces to
showing that $\int_{K_1(\frq)} V^\circ(g^{-1})\, dg = \vol(K_1(\frq)) \cdot V^\circ(1)
= \vol(K_1(\frq)) \neq 0$, which is immediate from the normalization $V^\circ(1) = 1$ and
$K_1(\frq)$-invariance of $V^\circ$.  The ``hard'' nonvanishing problem that the
\texttt{af} attempt struggled with is thus an artifact of not having the
Godement--Jacquet reduction.

\subsection{Summary Table}

\begin{center}
\renewcommand{\arraystretch}{1.3}
\begin{tabular}{@{}clcc@{}}
\toprule
\textbf{\#} & \textbf{Issue} & \textbf{Genuine?} & \textbf{Relevant to official proof?} \\
\midrule
1 & Kirillov model evaluation & Genuine fact & No \\
2 & Casselman--Shalika scope & Genuine fact & No \\
3 & $K$-projection dilemma & Genuine for Iwasawa approach & No \\
4 & Test vector theory for $\GL_{n+1} \times \GL_n$ & Open problem & No \\
5 & $J_K(0)$ nonvanishing & Non-trivial question & No (trivially resolved) \\
\bottomrule
\end{tabular}
\end{center}

All five issues are real mathematical phenomena, but \emph{none} of them are relevant to
the official proof.  They are all artifacts of the \texttt{af} attempt's strategic choice
to analyze the Rankin--Selberg integral directly via Iwasawa unfolding and Kirillov model
support, rather than using the Godement--Jacquet functional equation as a bridge.


%======================================================================
\section{What the \texttt{af} Attempt Got Right}
\label{sec:right}
%======================================================================

Despite the strategic divergence, the \texttt{af} attempt produced several correct and
valuable contributions:

\begin{enumerate}
\item \textbf{Correct answer (YES):} The \texttt{af} attempt correctly identified that
  the answer is affirmative, which is non-trivial---the authors' commentary notes that
  some LLM attempts constructed $W$ depending on $\pi$, solving a weaker problem.  The
  \texttt{af} attempt understood the universality requirement from the start.

\item \textbf{Node 1.1 (Commutation identity):} The identity
  $W(\mathrm{diag}(g,1)\, u_Q) = \psi^{-1}(Q g_{nn})\, W(\mathrm{diag}(g,1))$ is
  correct, useful, and appears implicitly in the official solution (in the computation
  $W_0(gt_Q) = \psi(-g_{nn}) W_0(g d_Q^{-1})$, equation (12)).

\item \textbf{Node 1.2 (Monomial characterization):} The fact that a rational function
  in $q^{-s}$ with no zeros or poles on $\C^\times$ is a monomial is used in both
  approaches.  The official solution obtains an explicit constant directly, but the
  underlying algebraic fact is the same.

\item \textbf{Node 1.4.1 ($W^\circ$ factorization):} The factorization of $W^\circ$ out
  of the $K$-integral via $K_1(\frp^{c(\Pi)})$-invariance is a valid and standard
  argument.  While the official solution uses a different $W$ (not $W^\circ$), the
  factorization principle is the same.

\item \textbf{Node 1.4.2 (Vanishing for $m_n > c(\pi)$):} The argument that
  $\pi^{K_n} = 0$ for ramified $\pi$ is correct.  The official solution does not need
  this step (the torus collapse is automatic), but the underlying representation-theoretic
  fact is valid.

\item \textbf{Correct identification of difficulty:} The \texttt{af} attempt's systematic
  issues (Section~\ref{sec:issues}) accurately reflect the genuine obstacles to a
  ``direct analysis'' proof.  The self-diagnosis was remarkably accurate---the attempt
  knew where it was stuck and why.

\item \textbf{Correct use of references:} The references to JPSS, Matringe, Miyauchi,
  Casselman--Shalika, and Godement--Jacquet are all appropriate and correctly cited (even
  if some were applied outside their valid scope).
\end{enumerate}


%======================================================================
\section{What Went Wrong: Root Cause Analysis}
\label{sec:wrong}
%======================================================================

\subsection{The Core Error: Missing the Godement--Jacquet Bridge}

The \texttt{af} attempt's fundamental error was not a mathematical mistake but a
\emph{strategic} one: it tried to analyze the Rankin--Selberg integral directly, without
using the Godement--Jacquet functional equation as an intermediary.

This is understandable: the Rankin--Selberg integral is defined directly in terms of
Whittaker functions, and the most natural approach is to unfold it via the Iwasawa
decomposition and analyze the resulting torus sum.  This is, in fact, the standard
approach in much of the literature on Rankin--Selberg integrals.  The problem is that this
approach requires controlling the torus support of both $W$ and $V$ simultaneously, which
(as the \texttt{af} attempt discovered) is extremely difficult for general $\pi$.

The official solution's insight is that the Godement--Jacquet theory for the
\emph{smaller} group $G_n$ provides a way to ``dualize'' the integral, replacing the
Whittaker function $V$ (which has complicated torus support) with the indicator function
$\mathbf{1}_{K_1(\frq)}$ (which has trivial support properties).  This duality is
achieved by the identity $\beta^\sharp(\det g)\, \phi^\wedge(g) = |Q|^n\,
\mathbf{1}_{K_1(\frq)}(g)$, which is the key computation in Lemma~5.

\subsection{The Secondary Error: Wrong Choice of $W$}

The \texttt{af} attempt chose $W = W^\circ$ (the newvector of $\Pi$), while the official
solution uses $W_0$ (the Kirillov extension of $\mathbf{1}_{K_n}$).  These are different
functions (unless $\Pi$ is unramified, in which case $W_0$ may coincide with $W^\circ$ up
to normalization).

The newvector $W^\circ$ is a natural first guess---it is the most canonical element of
$\Wh(\Pi, \psi^{-1})$---but it does not have the determinantal support property that makes
the official argument work.  The official choice of $W_0$ is engineered precisely to have
compact determinantal support ($\det^{-1}(\fro^\times)$ on $G_n$), which collapses the
torus sum to a single term without any case analysis.

Interestingly, the authors' commentary notes that the best LLM attempt (ChatGPT 5.2 Pro)
\emph{did} identify a suitable $W$ and reduced to the nonvanishing of
$\int_{K_n} V(g)\, \psi(-Q g_{nn})\, dg$.  This suggests that the choice of $W$ is the
easier part; the harder part is the nonvanishing argument, which is where the
Godement--Jacquet bridge is essential.

\subsection{The Cascade of Complications}

The wrong strategic choices cascaded into increasingly complex proof obligations:

\begin{enumerate}
\item Wrong $W$ $\Rightarrow$ torus sum does not collapse $\Rightarrow$ need to analyze
  multi-dimensional torus sums.
\item Multi-dimensional torus sums $\Rightarrow$ need to control support of both $W^\circ$
  and $V$ $\Rightarrow$ case splitting by type of $\pi$.
\item Case splitting $\Rightarrow$ three independent proof branches, each with its own
  difficulties.
\item Unramified branch $\Rightarrow$ need Kirillov model for non-mirabolic elements
  $\Rightarrow$ Issue~1.
\item Supercuspidal branch $\Rightarrow$ need $J_K(0) \neq 0$ for $n \geq 2$
  $\Rightarrow$ Issue~5.
\item Non-supercuspidal branch $\Rightarrow$ need test vector theory for general $n$
  $\Rightarrow$ Issue~4.
\end{enumerate}

Each issue, in turn, generated multiple challenges in the adversarial framework, leading to
the 57 open challenges.


%======================================================================
\section{Lessons Learned}
\label{sec:lessons}
%======================================================================

\subsection{For Automated Theorem Proving}

\begin{enumerate}
\item \textbf{Strategy selection matters more than local correctness.}
  The \texttt{af} attempt produced many locally correct arguments (5 validated nodes), but
  the global strategy was suboptimal.  An automated prover that can search over proof
  strategies---not just refine a fixed strategy---would have been far more effective.

\item \textbf{Functional equations are not just for computing $L$-values.}
  The Godement--Jacquet functional equation is usually presented as a tool for establishing
  analytic properties of $L$-functions.  The official solution uses it as a \emph{duality
  transform}---a way to move from the Whittaker model (complicated) to the congruence
  subgroup (simple).  Automated provers should be trained to recognize functional equations
  as potential ``simplification bridges,'' not just as identities to be verified.

\item \textbf{Case splitting is a warning sign.}
  When a proof of a universal statement ($\forall \pi$) requires case splitting by the
  type of $\pi$, this often indicates that the proof is not using the right level of
  abstraction.  The official solution's universality (no case splitting) is a hallmark of
  the ``right'' proof.

\item \textbf{The adversarial framework correctly identified failure modes.}
  The 5 systematic issues were accurate diagnoses of why the proof was stuck.
  The \texttt{af} framework's strength is in \emph{verification} (catching errors) rather
  than \emph{synthesis} (finding the right approach).  This suggests a complementary
  architecture: use a creative ``strategy search'' module to propose proof outlines, and
  use the adversarial framework to verify them.

\item \textbf{Beware of ``natural'' choices.}
  The newvector $W^\circ$ is the most natural element of $\Wh(\Pi, \psi^{-1})$, and it was
  the \texttt{af} attempt's first (and only) choice.  The official solution's $W_0$ is
  less obvious but more effective.  Automated provers should be encouraged to explore
  multiple candidate constructions before committing to a proof strategy.
\end{enumerate}

\subsection{For the Representation Theory of $p$-Adic Groups}

\begin{enumerate}
\item \textbf{The Godement--Jacquet theory for a single group is underappreciated as a
  tool for Rankin--Selberg problems.}  The standard approach to Rankin--Selberg integrals
  uses the JPSS theory (zeta integrals for the \emph{pair} $\Pi \times \pi$).  The
  official solution instead uses the Godement--Jacquet theory for $\pi$ alone, combined
  with a clever choice of Schwartz data.  This is a genuinely novel technique.

\item \textbf{Determinantal support is a powerful tool.}  The collapse of the torus sum in
  the official solution comes from the fact that $W_0$ (restricted to $G_n$) is supported
  on $\det^{-1}(\fro^\times)$.  This is a much simpler condition than the detailed torus
  support analysis (Casselman--Shalika, Matringe formulas) that the \texttt{af} attempt
  attempted.

\item \textbf{The test vector problem for $\GL_{n+1} \times \GL_n$ remains open for
  general $n$.}  The \texttt{af} attempt's Issue~4 is a real gap in the literature.  The
  official solution circumvents it, but does not solve it.
\end{enumerate}

\subsection{Comparison with the Authors' Commentary on LLM Attempts}

The authors' commentary identifies three failure modes in LLM-generated solutions:
\begin{enumerate}[label=(\alph*)]
\item Constructing $W$ depending on $\pi$ (solving a weaker problem).
\item Choosing $V$ so the integrand is constant on its support (unviable due to central
  characters).
\item Asserting false support properties of $V$ via a ``standard Howe-vector existence
  result'' (contradicts central character).
\end{enumerate}

The \texttt{af} attempt avoided failure mode (a)---it correctly understood the
universality requirement.  It partially fell into failure mode (c) via the Kirillov model
evaluation issue (treating $V_0$ as having simpler support than it actually does).  It did
not exhibit failure mode (b).  Overall, the \texttt{af} attempt was more sophisticated
than the typical LLM attempt described in the commentary, but still failed to find the
correct proof strategy.


%======================================================================
\section{Conclusion}
\label{sec:conclusion}
%======================================================================

The comparison between the official solution and the \texttt{af} attempt illustrates a
fundamental challenge in automated theorem proving for research mathematics: \emph{finding
the right proof strategy is often harder than executing it}.

The official solution is elegant precisely because it reduces a difficult
representation-theoretic problem to a routine computation via a single well-chosen
duality transform (the Godement--Jacquet functional equation).  The \texttt{af} attempt,
lacking this key insight, was forced into an increasingly complex case analysis that
generated genuine mathematical difficulties---all of which dissolve when the right
approach is used.

The \texttt{af} attempt's strengths---correct answer identification, accurate error
diagnosis, rigorous local verification---are valuable but insufficient without the
strategic insight that makes the proof work.  Future automated provers would benefit from
a ``strategy search'' capability that explores multiple high-level proof outlines before
committing to detailed verification.


%======================================================================
% References
%======================================================================

\begin{thebibliography}{99}

\bibitem{JPSS83}
H.~Jacquet, I.~I.~Piatetski-Shapiro, and J.~A.~Shalika,
\emph{Rankin--Selberg convolutions},
Amer.\ J.\ Math.\ \textbf{105} (1983), 367--464.

\bibitem{GJ72}
R.~Godement and H.~Jacquet,
\emph{Zeta Functions of Simple Algebras},
Lecture Notes in Mathematics \textbf{260}, Springer, 1972.

\bibitem{Matringe13}
N.~Matringe,
\emph{Essential Whittaker functions for $\GL(n)$},
Doc.\ Math.\ \textbf{18} (2013), 1191--1214.

\bibitem{JPSS81}
H.~Jacquet, I.~I.~Piatetski-Shapiro, and J.~A.~Shalika,
\emph{Conducteur des repr\'esentations du groupe lin\'eaire},
Math.\ Ann.\ \textbf{256} (1981), 199--214.

\bibitem{CS80}
W.~Casselman and J.~Shalika,
\emph{The unramified principal series of $p$-adic groups II: The Whittaker function},
Compositio Math.\ \textbf{41} (1980), 207--231.

\bibitem{Bernstein84}
J.~N.~Bernstein,
\emph{$P$-invariant distributions on $\GL(N)$ and the classification of unitary
representations of $\GL(N)$ (non-Archimedean case)},
in \emph{Lie Group Representations, II}, Lecture Notes in Math.\ \textbf{1041},
Springer, 1984, 50--102.

\bibitem{BKL20}
A.~R.~Booker, M.~Krishnamurthy, and M.~Lee,
\emph{Test vectors for Rankin--Selberg $L$-functions},
J.\ Number Theory \textbf{209} (2020), 37--48.

\bibitem{Humphries21}
P.~Humphries,
\emph{Test vectors for Rankin--Selberg $L$-functions},
J.\ Number Theory \textbf{220} (2021), 259--279.

\bibitem{AB24}
E.~Assing and V.~Blomer,
\emph{The density conjecture for principal congruence subgroups},
Duke Math.\ J.\ \textbf{173} (2024), 1949--2008.

\bibitem{Nelson26}
P.~D.~Nelson,
Problem~2 in \emph{First Proof: Solutions and Comments},
Abouzaid, Blumberg, Hairer, Kileel, Kolda, Nelson, Spielman, Srivastava, Ward,
Weinberger, Williams, February 2026.

\end{thebibliography}


\end{document}
