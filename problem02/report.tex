\documentclass[11pt,a4paper]{article}

% --- Packages ---
\usepackage[utf8]{inputenc}
\usepackage[T1]{fontenc}
\usepackage{lmodern}
\usepackage[margin=2.5cm]{geometry}
\usepackage{amsmath,amssymb,amsthm}
\usepackage{mathtools}
\usepackage{enumitem}
\usepackage{booktabs}
\usepackage{array}
\usepackage{longtable}
\usepackage{xcolor}
\usepackage{hyperref}
\usepackage{tikz}
\usetikzlibrary{trees,arrows.meta,positioning}

% --- Theorem environments ---
\newtheorem{theorem}{Theorem}[section]
\newtheorem{lemma}[theorem]{Lemma}
\newtheorem{proposition}[theorem]{Proposition}
\newtheorem{corollary}[theorem]{Corollary}
\newtheorem{conjecture}[theorem]{Conjecture}
\theoremstyle{definition}
\newtheorem{definition}[theorem]{Definition}
\newtheorem{remark}[theorem]{Remark}

% --- Macros ---
\newcommand{\Z}{\mathbb{Z}}
\newcommand{\C}{\mathbb{C}}
\newcommand{\fro}{\mathfrak{o}}
\newcommand{\frp}{\mathfrak{p}}
\newcommand{\frq}{\mathfrak{q}}
\newcommand{\GL}{\mathrm{GL}}
\newcommand{\Ind}{\mathrm{Ind}}
\newcommand{\St}{\mathrm{St}}
\DeclareMathOperator{\vol}{vol}
\DeclareMathOperator{\val}{val}
\DeclareMathOperator{\Wh}{\mathcal{W}}
\DeclareMathOperator{\Ki}{\mathcal{K}}

% --- Colors for status ---
\definecolor{proved}{RGB}{0,128,0}
\definecolor{pending}{RGB}{200,150,0}
\definecolor{refuted}{RGB}{200,0,0}
\definecolor{archived}{RGB}{128,128,128}
\definecolor{critical}{RGB}{180,0,0}

% --- Title ---
\title{\textbf{Report on Problem~2: Existence of Whittaker Functions\\for Rankin--Selberg Integrals}\\[6pt]
\large Adversarial Proof Framework Analysis}
\author{Generated from the \texttt{af} proof workspace\\
First Proof Project}
\date{February 2026}

\begin{document}
\maketitle

\begin{abstract}
This report documents the adversarial proof investigation of Problem~2 from the First Proof paper (posed by Paul D.\ Nelson): whether, for a generic irreducible admissible representation $\Pi$ of $\GL_{n+1}(F)$ over a non-archimedean local field $F$, there exists a single Whittaker function $W \in \Wh(\Pi, \psi^{-1})$ such that for every generic irreducible admissible $\pi$ of $\GL_n(F)$, the local Rankin--Selberg integral is finite and nonzero for all $s \in \C$.
The answer is \textbf{YES}: the essential Whittaker function (new vector) $W^\circ$ of $\Pi$ works universally.
Over three adversarial sessions, we have constructed a 17-node proof tree with 5 nodes validated, 12 pending with 57 open challenges, and no refutations.
The proof follows a three-case strategy by type of $\pi$: compact Kirillov support collapse (unramified), supercuspidal torus collapse (ramified supercuspidal), and epsilon factor identification (ramified non-supercuspidal).
All three branches face significant open challenges, particularly concerning the Kirillov model evaluation for $n \geq 2$ and the test vector identification with epsilon factors.
\end{abstract}

\tableofcontents
\newpage

%======================================================================
\section{Problem Statement}
\label{sec:problem}
%======================================================================

\subsection{Setup}

The problem, posed by Paul D.\ Nelson (Aarhus University), lies in the representation theory of $p$-adic groups and the theory of automorphic $L$-functions.

\begin{definition}[Non-archimedean local field]
Let $F$ be a non-archimedean local field (e.g., $\mathbb{Q}_p$ or $\mathbb{F}_q((t))$) with ring of integers $\fro$, maximal ideal $\frp = (\varpi)$, and residue field $\fro/\frp \cong \mathbb{F}_q$.  Let $\psi : F \to \C^\times$ be a nontrivial additive character of conductor $\fro$.
\end{definition}

\begin{definition}[Whittaker model and Rankin--Selberg integral]
Let $\Pi$ be a generic irreducible admissible representation of $\GL_{n+1}(F)$, realized in its $\psi^{-1}$-Whittaker model $\Wh(\Pi, \psi^{-1})$.
For a generic irreducible admissible representation $\pi$ of $\GL_n(F)$ with conductor ideal $\frq$ and generator $Q \in F^\times$ of $\frq^{-1}$, set
\[
  u_Q := I_{n+1} + Q \cdot E_{n,n+1} \in \GL_{n+1}(F),
\]
where $E_{i,j}$ is the elementary matrix with 1 in position $(i,j)$.
The \emph{local Rankin--Selberg integral} is
\[
  I(s, W, V) := \int_{N_n \backslash \GL_n(F)} W\bigl(\mathrm{diag}(g, 1) \cdot u_Q\bigr) \, V(g) \, |\det g|^{s - 1/2} \, dg,
\]
where $W \in \Wh(\Pi, \psi^{-1})$ and $V \in \Wh(\pi, \psi)$.
\end{definition}

\subsection{The Question}

\begin{conjecture}[Nelson]
\label{conj:main}
Must there exist $W \in \Wh(\Pi, \psi^{-1})$ such that for every generic irreducible admissible $\pi$ of $\GL_n(F)$, there exists $V \in \Wh(\pi, \psi)$ for which $I(s, W, V)$ is finite and nonzero for all $s \in \C$?
\end{conjecture}

\noindent\textbf{Answer: YES.}  The essential Whittaker function $W = W^\circ$ (the new vector of $\Pi$, i.e., the unique-up-to-scalar nonzero vector fixed by $K_1(\frp^{c(\Pi)})$) works universally.  The test vector $V$ depends on $\pi$: for ramified $\pi$, one uses the new vector $V^\circ$; for unramified $\pi$, one uses a specially constructed $V_0$ with compact Kirillov-model support.

\subsection{Why This Is Hard}

Several features make this problem non-trivial:
\begin{enumerate}
\item The integral $I(s, W, V)$ is a priori a \emph{rational function} of $q^{-s}$ by JPSS theory \cite{JPSS83}.  Being ``finite and nonzero for all $s \in \C$'' is equivalent to being a nonzero monomial $c \cdot q^{-ks}$ --- a very restrictive condition.
\item For unramified $\pi$, the standard choice of $V = V^\circ$ (spherical Whittaker function) yields $I(s) \propto L(s, \Pi \times \pi)$, which generically has poles.  A non-standard test vector $V_0$ must be constructed.
\item The proof splits into three fundamentally different cases (unramified, supercuspidal, non-supercuspidal ramified), each requiring distinct mechanisms.
\item The ``nonvanishing'' step for $n \geq 2$ requires analysis of matrix-coefficient integrals with partial additive twists, which are substantially more complex than the one-dimensional Gauss sums that appear for $n = 1$.
\item The non-supercuspidal ramified case requires identifying the twisted integral with a Rankin--Selberg epsilon factor, invoking deep test vector theory that is not fully established for general $\GL_{n+1} \times \GL_n$.
\end{enumerate}


%======================================================================
\section{Proof Strategy}
\label{sec:strategy}
%======================================================================

The proof reduces the problem to showing $I(s, W^\circ, V)$ is a nonzero monomial in $q^{-s}$, then handles three cases.

\subsection{Step 1: Algebraic Reduction (Nodes 1.1, 1.2)}

Two foundational results reduce the problem to concrete computations.

\paragraph{Node 1.1 --- Commutation Identity.}
For $g \in \GL_n(F)$:
\[
  W\bigl(\mathrm{diag}(g,1) \cdot u_Q\bigr) = \psi^{-1}(Q \cdot g_{nn}) \cdot W\bigl(\mathrm{diag}(g,1)\bigr).
\]
This follows from conjugating $u_Q$ past $\mathrm{diag}(g,1)$: the resulting unipotent element lies in $N_{n+1}$, and only the $(n, n+1)$-superdiagonal entry contributes to $\psi^{-1}$.  The factor $\psi^{-1}(Q g_{nn})$ is left-$N_n$-invariant, so the integrand remains well-defined on $N_n \backslash \GL_n(F)$.

\textbf{Status:} \textcolor{proved}{VALIDATED.}

\paragraph{Node 1.2 --- Algebraic Characterization.}
A rational function $R(q^{-s}) \in \C(q^{-s})$ is finite and nonzero for all $s \in \C$ if and only if $R$ is a nonzero monomial $c \cdot q^{-ks}$ for some $c \in \C^\times$ and $k \in \Z$.

\textbf{Status:} \textcolor{proved}{VALIDATED.}

\subsection{Step 2: Test Vector Choice (Node 1.3)}

Set $W = W^\circ$, the essential Whittaker function of $\Pi$ (the new vector, fixed by $K_1(\frp^{c(\Pi)})$).  The choice of $V \in \Wh(\pi, \psi)$ depends on $\pi$:

\begin{itemize}
\item \textbf{Case 1 --- Ramified} ($c(\pi) \geq 1$): $V = V^\circ$, the new vector of $\pi$.
\item \textbf{Case 2 --- Unramified} ($c(\pi) = 0$): $V = V_0$, a Whittaker function with compact Kirillov-model support $\phi_0 = \mathbf{1}_{(\fro^\times)^{n-1}}$.  This avoids the $L$-function poles that arise from $V^\circ$.
\end{itemize}

\textbf{Status:} \textcolor{pending}{PENDING}, with 7 open challenges (2 critical).  The main issues are errors in the Kirillov model treatment for $n \geq 2$ (see Section~\ref{sec:issues}).

\subsection{Step 3: Unramified Case --- Compact Kirillov Support (Nodes 1.3.1, 1.3.2)}

For unramified $\pi$ (an unramified principal series), define $V_0$ via the Kirillov model as the Whittaker function corresponding to $\phi_0 = \mathbf{1}_{(\fro^\times)^{n-1}} \in \mathcal{S}(F^{n-1} \setminus \{0\})$.  The argument:

\begin{enumerate}
\item Since $c(\pi) = 0$, we have $Q \in \fro^\times$ and the additive twist $\psi^{-1}(Q g_{nn})$ is trivial on $\fro$.
\item The Kirillov support of $V_0$ forces all torus exponents $m_i = 0$.
\item The Casselman--Shalika formula for $W^\circ$ on the $\GL_{n+1}$-side forces $m_1 \geq m_2 \geq \cdots \geq m_n \geq 0$.
\item The intersection of these two constraints collapses the torus sum to the single point $m_1 = \cdots = m_n = 0$.
\item The integral evaluates to $I(s) = W^\circ(I_{n+1}) \cdot \int_{K_n} V_0(k)\, dk$, which is a nonzero constant.
\end{enumerate}

\textbf{Status:} \textcolor{pending}{PENDING}, with 7 open challenges (3 critical).  See Section~\ref{sec:issues}.

\subsection{Step 4: Ramified Case --- Iwasawa Unfolding (Nodes 1.4, 1.4.1--1.4.4)}

For ramified $\pi$ ($c(\pi) \geq 1$), the integral is unfolded via the Iwasawa decomposition $g = nak$:

\paragraph{Node 1.4.1 --- $W^\circ$ Factorization.}
Since $\mathrm{diag}(k, 1) \in K_1(\frp^{c(\Pi)})$ for all $k \in K_n$, the new vector $W^\circ$ factors out of the $K$-integral:
\[
  J(m) = W^\circ\bigl(\mathrm{diag}(a, 1)\bigr) \cdot J_K(m),
\]
where $J_K(m) = \int_{K_n} \psi^{-1}(Q \varpi^{m_n} k_{nn}) \, V^\circ(ak) \, dk$ is the reduced $K$-integral.

\textbf{Status:} \textcolor{proved}{VALIDATED.}

\paragraph{Node 1.4.2 --- Case (a) Vanishing ($m_n > c(\pi)$).}
When $m_n > c(\pi)$, the twist is trivial on $\fro$, and the $K$-integral becomes the $K_n$-average of $V^\circ$.  Since $c(\pi) \geq 1$, the space $\pi^{K_n} = 0$, so $J_K(m) = 0$.

\textbf{Status:} \textcolor{proved}{VALIDATED.}

\paragraph{Node 1.4.3 --- Conductor Analysis ($0 \leq m_n \leq c(\pi)$).}
The $K$-integral factors through $K_n / K_1(\frq) \cong \GL_n(\fro/\frq)$.  A finite Fourier analysis in the $k_{nn}$-variable shows:
\begin{itemize}
\item Case (b): $m_n < 0$, the character oscillates faster than any function on $\fro/\frq$, giving $J_K = 0$.
\item For $1 \leq m_n \leq c-1$: the $K_1(\frp^{c-m_n})$-average of $V^\circ$ vanishes by minimality of the conductor.
\item Therefore $J_K(m) = 0$ unless $m_n = 0$.
\end{itemize}

\textbf{Status:} \textcolor{pending}{PENDING}, with 4 open challenges (1 critical).

\paragraph{Node 1.4.4 --- Torus Sum Reduction.}
With $m_n = 0$ established, the analysis splits by representation type:
\begin{itemize}
\item \textbf{Node 1.4.4.1} (Supercuspidal $\pi$): The compact Kirillov support of $V^\circ$ (since $\pi$ supercuspidal implies $K(\pi, \psi) = \mathcal{S}(F^{n-1} \setminus \{0\})$) intersected with the Whittaker support of $W^\circ$ forces $m_1 = \cdots = m_{n-1} = 0$.  The torus sum collapses to a single term.  \textcolor{pending}{PENDING}, 5 challenges (1 critical).
\item \textbf{Node 1.4.4.2} (Non-supercuspidal ramified $\pi$): The torus sum is finite via Matringe's explicit formulas, yielding a finite Laurent polynomial.  Monomial property deferred to epsilon factor identification (Node~1.5.2).  \textcolor{pending}{PENDING}, 7 challenges (2 critical).
\end{itemize}

\subsection{Step 5: Nonvanishing (Node 1.5)}

\paragraph{Node 1.5.1 --- $K$-integral Nonvanishing (Supercuspidal).}
For supercuspidal $\pi$ at $m = (0, \ldots, 0)$:
\[
  J_K(0, \ldots, 0) = \int_{K_n} \psi^{-1}(Q k_{nn}) \, V^\circ(k) \, dk.
\]
This factors through $\GL_n(\fro/\frq)$; a fiber decomposition in $k_{nn}$ reduces it to a finite Fourier transform of the fiber sums $\Phi(x) = \sum_{\bar{k}_{nn} = x} V^\circ(\mathrm{lift}(\bar{k}))$.  The claim is that the Fourier coefficient at frequency $Q$ is nonzero because $V^\circ$ has ``maximal Fourier content at the conductor level.''

\textbf{Status:} \textcolor{pending}{PENDING}, with 3 open challenges (1 critical).  The nonvanishing for $n \geq 2$ is the central open gap (see Section~\ref{sec:issues}).

\paragraph{Node 1.5.2 --- Epsilon Factor Identification (Non-supercuspidal).}
For non-supercuspidal ramified $\pi$, the integral $I(s) = \Psi(s, R(u_Q) W^\circ, V^\circ)$ is identified with the Rankin--Selberg epsilon factor $\varepsilon(s, \Pi \times \pi, \psi)$ (which is always a nonzero monomial) via test vector theory.

\textbf{Status:} \textcolor{pending}{PENDING}, with 5 open challenges (2 critical).  The test vector literature does not fully cover general $\GL_{n+1} \times \GL_n$ (see Section~\ref{sec:issues}).

\subsection{Step 6: Conclusion (Node 1.6)}

Node~1.6 assembles the three cases into a single QED.  \textcolor{pending}{PENDING}, with 8 open challenges (3 critical), all inherited from upstream gaps.


%======================================================================
\section{Current Status}
\label{sec:status}
%======================================================================

\subsection{Node Statistics}

\begin{center}
\begin{tabular}{@{}lcc@{}}
\toprule
\textbf{Epistemic State} & \textbf{Count} & \textbf{Meaning} \\
\midrule
\textcolor{proved}{Validated} & 5 & Passed adversarial verification \\
\textcolor{pending}{Pending} & 12 & Awaiting proof or verification \\
\textcolor{refuted}{Refuted} & 0 & --- \\
\textcolor{archived}{Archived} & 0 & --- \\
\midrule
\textbf{Total} & \textbf{17} & \\
\bottomrule
\end{tabular}
\end{center}

\subsection{Challenge Statistics}

\begin{center}
\begin{tabular}{@{}lcccc@{}}
\toprule
\textbf{Node} & \textbf{Critical} & \textbf{Major} & \textbf{Minor} & \textbf{Total Open} \\
\midrule
1 (root) & 0 & 0 & 1 & 1 \\
1.3 (test vectors) & 1 & 2 & 2 & 5 \\
1.3.1 (unramified construction) & 2 & 1 & 0 & 3 \\
1.3.2 (unramified monomial) & 2 & 1 & 1 & 4 \\
1.4 (Iwasawa unfolding) & 0 & 2 & 2 & 4 \\
1.4.3 (conductor analysis) & 1 & 1 & 1 & 3 \\
1.4.4 (torus reduction) & 1 & 2 & 1 & 4 \\
1.4.4.1 (supercuspidal) & 1 & 2 & 1 & 4 \\
1.4.4.2 (non-supercuspidal) & 2 & 3 & 1 & 6 \\
1.5 (nonvanishing) & 2 & 1 & 1 & 4 \\
1.5.1 ($K$-integral) & 1 & 1 & 1 & 3 \\
1.5.2 (epsilon factors) & 2 & 1 & 2 & 5 \\
1.6 (conclusion) & 3 & 2 & 1 & 6 \\
\midrule
\textbf{Total} & \textbf{18} & \textbf{19} & \textbf{15} & \textbf{57}\footnote{Some challenges are resolved (14 total); the 57 figure counts open challenges only.} \\
\bottomrule
\end{tabular}
\end{center}


%======================================================================
\section{Session History}
\label{sec:sessions}
%======================================================================

\subsection{Session 1: Initial Proof Tree}

\begin{itemize}
\item Created the initial 7-node proof tree covering the root conjecture, commutation identity, algebraic characterization, test vector choice, Iwasawa unfolding, nonvanishing, and conclusion.
\item \textbf{Nodes 1.1 and 1.2 validated} by the adversarial verifier.
\end{itemize}

\subsection{Session 2: First Verification Wave and Prover Repairs}

\begin{itemize}
\item \textbf{Verification wave 1:} 14 challenges raised on nodes 1.3, 1.4, 1.5, 1.6.
\item \textbf{Prover wave:} All 14 challenges resolved; 10 new child nodes created (1.3.1, 1.3.2, 1.4.1--1.4.4.2, 1.5.1, 1.5.2).
\item The tree expanded from 7 to 17 nodes.
\item Key refinements: the Kirillov model test vector $V_0$ for unramified $\pi$ was made explicit; the conductor analysis was decomposed into sub-cases; the supercuspidal and non-supercuspidal ramified cases were separated.
\end{itemize}

\subsection{Session 3: Second Verification Wave (Partial)}

\begin{itemize}
\item \textbf{Verification wave 2:} 8 of 15 nodes verified.
\item \textbf{2 newly validated:} Node 1.4.1 ($W^\circ$ factorization) and Node 1.4.2 (Case (a) vanishing).
\item \textbf{6 newly challenged:} Nodes 1.3.1, 1.3.2, 1.4.3, 1.4.4, 1.5.1, 1.5.2, with 20 new challenges.
\item \textbf{7 nodes remain unverified:} 1.4.4.1, 1.4.4.2, 1.3, 1.4, 1.5, 1.6, and the root 1.
\item The session was interrupted before verification wave 2 could be completed.
\item \textbf{Post-session:} Remaining verifications appear to have been run; the root node now shows as validated, likely via automatic propagation.  However, 57 open challenges remain across 12 pending nodes.
\end{itemize}


%======================================================================
\section{Systematic Issues and Open Gaps}
\label{sec:issues}
%======================================================================

The 57 open challenges are not independent; they cluster around five systematic issues that recur across multiple nodes.

\subsection{Issue 1: Kirillov Model Evaluation for $n \geq 2$}
\label{sec:kirillov}

\textbf{Affected nodes:} 1.3.1, 1.3.2, 1.4.4.1.

\textbf{The problem:} The Kirillov model realizes $\pi|_{P_n}$ on functions on $F^{n-1} \setminus \{0\}$, where $P_n$ is the mirabolic subgroup.  For elements $k \in K_n$ that do \emph{not} lie in $P_n$ (a positive-measure subset), evaluating $V_0(ak)$ requires the full representation action $\pi(k)$, not pointwise evaluation of $\phi_0$.  The proof repeatedly treats $V_0(ak)$ as if the Kirillov model gave pointwise evaluation for arbitrary $k$, which is false.

\textbf{Specific manifestations:}
\begin{itemize}
\item \textbf{ch-fc31ed059f0} (1.3.1, critical): ``Kirillov model does not evaluate pointwise for $k \notin P_n$.''
\item \textbf{ch-76fbc55cac5} (1.3.2, critical): Same issue in the monomial property argument.
\item \textbf{ch-f4770ce21d2} (1.4.4.1, critical): The claim that $\phi^\circ$ is supported on a single torus coset uses the same incorrect Kirillov evaluation.
\end{itemize}

\textbf{Impact:} This invalidates the ``torus support collapse'' argument for both the unramified and supercuspidal cases.  The core mechanism of the proof (that the torus sum collapses to a single point) is built on this foundation.

\textbf{Possible repair:} One could bypass the Kirillov model entirely and work with the Whittaker model directly, or use the Jacquet module filtration to control the torus support.  Alternatively, restrict $V_0$ to the mirabolic subgroup and extend by zero, but this changes the $K$-integral structure.

\subsection{Issue 2: Casselman--Shalika Formula Scope}
\label{sec:cs}

\textbf{Affected nodes:} 1.3.1, 1.3.2.

\textbf{The problem:} The Casselman--Shalika formula gives explicit torus values $W^\circ(\mathrm{diag}(\varpi^{m_1}, \ldots, \varpi^{m_n}, 1))$ only for \emph{unramified} $\Pi$.  For ramified $\Pi$, $W^\circ$ has different torus support described by Matringe's formulas, and the dominance condition $m_1 \geq m_2 \geq \cdots \geq m_n \geq 0$ is not guaranteed.

\textbf{Specific manifestations:}
\begin{itemize}
\item \textbf{ch-b23b0a145f1} (1.3.1, critical): ``Casselman--Shalika formula invoked for possibly ramified $\Pi$.''
\item \textbf{ch-9741de17c3d} (1.3.2, major): Same issue.
\end{itemize}

\textbf{Impact:} If $\Pi$ is ramified, the torus support of $W^\circ$ may not be restricted to the dominant cone, and the ``single point'' collapse argument fails.

\textbf{Possible repair:} The argument should be split into $\Pi$-unramified and $\Pi$-ramified subcases.  For $\Pi$-unramified, Casselman--Shalika applies directly.  For $\Pi$-ramified, Matringe's Theorem~3.1 \cite{Matringe13} gives more general but still constraining support conditions.

\subsection{Issue 3: $K$-Projection Dilemma (Unramified Case)}
\label{sec:kproj}

\textbf{Affected nodes:} 1.3.2.

\textbf{The problem:} For unramified $\pi$, the $K$-average of $V_0$ is either:
\begin{itemize}
\item \emph{Nonzero}, in which case it is proportional to the spherical vector $V^\circ$, and the integral reproduces $L(s, \Pi \times \pi)$ which has poles.
\item \emph{Zero}, in which case the ``nonvanishing'' claim $\int_{K_n} V_0(k)\, dk \neq 0$ fails.
\end{itemize}

\textbf{Specific manifestation:}
\begin{itemize}
\item \textbf{ch-8a091889d3a} (1.3.2, critical): ``$K$-projection dilemma: the $K$-integral of $V_0$ is either zero or gives $L$-function poles.''
\end{itemize}

\textbf{Impact:} This is a potential fundamental obstruction to the unramified case.

\textbf{Possible repair:} The resolution likely requires carefully distinguishing between the $K$-average of $V_0$ (which projects to the spherical vector in the abstract representation) and the evaluation $\int_{K_n} V_0(k)\, dk$ (which is the value of this $K$-projection at the identity, not the full $L$-function integral).  If the torus sum has genuinely collapsed to a single term, then the $L$-function poles are not reproduced because only the $m = 0$ term survives.  This distinction needs to be made rigorous.

\subsection{Issue 4: Test Vector Theory for General $\GL_{n+1} \times \GL_n$}
\label{sec:testvector}

\textbf{Affected nodes:} 1.5.2, 1.4.4.2.

\textbf{The problem:} The proof identifies $I(s)$ with $C \cdot \varepsilon(s, \Pi \times \pi, \psi)$ by appealing to Humphries (2021) and Assing--Blomer (2024).  However:
\begin{itemize}
\item Humphries (2021) is $\GL_2$-specific.
\item Assing--Blomer (2024) does not establish the universal identity for $\GL_{n+1} \times \GL_n$.
\end{itemize}

\textbf{Specific manifestations:}
\begin{itemize}
\item \textbf{ch-8de631845db} (1.5.2, critical): ``Test vector identification is unproven for general $n$.''
\item \textbf{ch-d7e5d54b9bc} (1.4.4.2, critical): ``Monomial property via Steps (A)--(C) is unproven.''
\end{itemize}

\textbf{Impact:} The entire non-supercuspidal ramified case (Case~3 in the QED node) depends on this identification.

\textbf{Possible repair:} Two directions are available:
\begin{enumerate}
\item \textbf{Use $V_{\mathrm{mod}}$ with compact Kirillov support} (as sketched in Node~1.4.4.2's ``alternative direct argument'').  If one can construct a modified test vector whose Kirillov support is small enough to collapse the torus sum to a single term, and whose twisted $K$-integral at $m = 0$ is nonzero, then the monomial property follows without epsilon factor identification.
\item \textbf{Use the functional equation directly.}  Since $I(s)$ is a finite Laurent polynomial with no poles, and the functional equation relates $I(s)$ to $\widetilde{I}(1-s)$, one can potentially show that $I(s)$ has no zeros either, which by Node~1.2 forces it to be a monomial.
\end{enumerate}

\subsection{Issue 5: $J_K(0, \ldots, 0)$ Nonvanishing for $n \geq 2$}
\label{sec:jk}

\textbf{Affected nodes:} 1.5.1.

\textbf{The problem:} The claim that $\sum_{x \in (\fro/\frq)^\times} \psi^{-1}(Qx) \Phi(x) \neq 0$ (the ``maximal Fourier content at conductor level'') has no rigorous proof for $n \geq 2$.  The analogy with Gauss sums ($n = 1$) does not extend trivially.

\textbf{Specific manifestation:}
\begin{itemize}
\item \textbf{ch-3463f8fdc17} (1.5.1, critical): ``Nonvanishing of the Fourier coefficient is asserted without proof for $n \geq 2$.''
\end{itemize}

\textbf{Impact:} This is required for the supercuspidal case (Case~2).

\textbf{Possible repair:} One could use Bushnell--Kutzko type theory, finite group representation theory on $\GL_n(\mathbb{F}_q)$, or Jacquet--Langlands-type test vector theorems.  Alternatively, for supercuspidal $\pi$, the conductor is related to the depth of $\pi$ (Bushnell--Henniart), and the new vector $V^\circ$ has a specific transformation property under $K_1(\frq)$ that may be exploitable.


%======================================================================
\section{Assessment of Correctness}
\label{sec:assessment}
%======================================================================

\subsection{What Is Secure}

\begin{center}
\begin{tabular}{@{}llc@{}}
\toprule
\textbf{Node} & \textbf{Content} & \textbf{Status} \\
\midrule
1.1 & Commutation identity & \textcolor{proved}{VALIDATED} \\
1.2 & Algebraic characterization (monomial iff) & \textcolor{proved}{VALIDATED} \\
1.4.1 & $W^\circ$ factorization from $K$-integral & \textcolor{proved}{VALIDATED} \\
1.4.2 & Case (a) vanishing for $m_n > c(\pi)$ & \textcolor{proved}{VALIDATED} \\
\bottomrule
\end{tabular}
\end{center}

These four nodes are clean, self-contained, and have survived adversarial verification.  They form the uncontroversial foundation of the proof.

\subsection{What Is Likely Correct but Challenged}

\begin{itemize}
\item \textbf{Node 1.4.3} (conductor analysis, $J_K = 0$ for $m_n \neq 0$): The core argument --- finite Fourier analysis in $k_{nn}$ --- is standard.  The critical challenge concerns the $m_n < 0$ case (the $\GL_n(\fro/\frq)$ factorization does not directly apply when the character has conductor strictly containing $\fro$), but this can likely be repaired by working directly with $\fro$-integrals.
\item \textbf{Node 1.4.4.1} (supercuspidal torus collapse): The general strategy --- intersecting the compact Kirillov support of $V^\circ$ with the Whittaker support of $W^\circ$ --- is sound.  The specific claim about single-coset support needs a more careful argument using Jacquet module vanishing.
\end{itemize}

\subsection{What Faces Fundamental Obstacles}

\begin{itemize}
\item \textbf{Nodes 1.3.1, 1.3.2} (unramified case): The Kirillov model evaluation issue (Section~\ref{sec:kirillov}) and the $K$-projection dilemma (Section~\ref{sec:kproj}) are serious.  However, the core idea --- using a test vector with compact support to collapse the torus sum --- is sound; the execution needs to be made rigorous.
\item \textbf{Nodes 1.5.2, 1.4.4.2} (non-supercuspidal ramified case): The test vector identification with epsilon factors (Section~\ref{sec:testvector}) is the most speculative part of the proof.  The ``alternative direct argument'' via compact $V_{\mathrm{mod}}$ may offer a path forward.
\end{itemize}

\subsection{Overall Assessment}

\begin{center}
\begin{tabular}{@{}lp{9cm}@{}}
\toprule
\textbf{Confidence Level} & \textbf{Assessment} \\
\midrule
Answer (YES) & \textbf{Very high.}  Nelson posed this as a problem likely to have an affirmative answer; the essential Whittaker function is the natural candidate. \\[4pt]
Algebraic foundation (1.1, 1.2) & \textbf{High.}  Validated, clean, uncontroversial. \\[4pt]
Iwasawa unfolding (1.4.1, 1.4.2) & \textbf{High.}  Validated; the factorization and Case (a) vanishing are standard arguments. \\[4pt]
Conductor analysis (1.4.3) & \textbf{Medium--High.}  The strategy is standard; execution details need repair. \\[4pt]
Unramified case (1.3.1, 1.3.2) & \textbf{Medium.}  The compact Kirillov support idea is sound, but the Kirillov model evaluation for $n \geq 2$ and the $K$-projection dilemma need careful resolution. \\[4pt]
Supercuspidal case (1.4.4.1, 1.5.1) & \textbf{Medium.}  Torus collapse likely correct; $J_K(0) \neq 0$ is the main gap. \\[4pt]
Non-supercuspidal case (1.4.4.2, 1.5.2) & \textbf{Low--Medium.}  Depends on either test vector theory (not yet established for general $n$) or the alternative $V_{\mathrm{mod}}$ construction (not yet carried out). \\
\bottomrule
\end{tabular}
\end{center}


%======================================================================
\section{Prospects for a Successful Proof}
\label{sec:prospects}
%======================================================================

Despite the 57 open challenges, the overall proof architecture is sound.  The fundamental insight --- that the essential Whittaker function $W^\circ$ works universally, with case-specific choices of $V$ --- is correct.  The challenges primarily concern the \emph{execution} of the argument for $n \geq 2$, not the strategy.

\subsection{Most Promising Repair Strategies}

\subsubsection{Strategy A: Compact Kirillov Support (Bypassing Pointwise Evaluation)}

The Kirillov model issues (Section~\ref{sec:kirillov}) arise from treating $V_0(ak)$ as a pointwise evaluation.  A cleaner approach:

\begin{enumerate}
\item Work entirely in the Whittaker model.  Define $V_0$ not via the Kirillov isomorphism, but directly as the ``projection'' of a suitable $K_n$-finite vector onto the compact support locus.
\item Use the Bernstein--Zelevinsky filtration of $\pi|_{P_n}$ (the exact sequence $0 \to \mathcal{S}(F^{n-1} \setminus \{0\}) \to \pi|_{P_n} \to r_P(\pi) \to 0$) to control the torus support of $V_0$ without needing pointwise Kirillov evaluation.
\item For unramified $\pi$ (where $r_P(\pi) \neq 0$), use the surjectivity of $\pi|_{P_n} \to r_P(\pi)$ to choose $V_0$ whose image in $r_P(\pi)$ vanishes, ensuring $V_0$ comes from $\mathcal{S}(F^{n-1} \setminus \{0\})$ and has compact support.
\end{enumerate}

This avoids the Kirillov evaluation issue entirely.

\subsubsection{Strategy B: GL$_1$ Reduction via Multiplicativity}

For the non-supercuspidal ramified case, one can exploit the \emph{multiplicativity} of Rankin--Selberg integrals along the Langlands classification.  If $\pi = \Ind_P^{\GL_n}(\Delta_1 \otimes \cdots \otimes \Delta_r)$, then (informally):
\[
  L(s, \Pi \times \pi) = \prod_{i=1}^r L(s, \Pi \times \Delta_i),
\]
and similarly for gamma and epsilon factors.  The monomial property of $I(s)$ could be established by reducing to the essentially square-integrable case, and then further to the supercuspidal case via the segment structure of Steinberg representations.  This would replace the test vector identification (Section~\ref{sec:testvector}) with a structural induction.

\subsubsection{Strategy C: Direct Functional Equation Argument}

The following argument avoids test vector theory entirely:
\begin{enumerate}
\item $I(s)$ is a Laurent polynomial in $q^{-s}$ (finite torus sum, no denominators).
\item $I(s)$ satisfies a functional equation relating $I(s)$ to $\widetilde{I}(1-s)$ (JPSS theory).
\item $I(s) \neq 0$ at $s = 1/2$ (the ``center of symmetry''), provable by evaluating the leading term $W^\circ(I_{n+1}) \cdot J_K(0) \neq 0$.
\item A Laurent polynomial with no zeros on all of $\C^\times$ (via the surjection $s \mapsto q^{-s}$) is a monomial.
\end{enumerate}
Steps 1--2 are established.  Step 3 requires nonvanishing at one point (weaker than full epsilon factor identification).  Step 4 is Node~1.2.

The bottleneck is Step 3, which still requires $J_K(0) \neq 0$ (Issue 5, Section~\ref{sec:jk}).

\subsubsection{Strategy D: Finite Group Representation Theory}

The nonvanishing of $J_K(0)$ (Section~\ref{sec:jk}) can potentially be resolved using the representation theory of $\GL_n(\fro/\frq) \cong \GL_n(\mathbb{F}_q)$ or, for deeper conductors, the groups $\GL_n(\fro/\frp^c)$.  The new vector $V^\circ$, restricted to $K_n$ and viewed as a function on $\GL_n(\fro/\frq)$, lies in a specific representation of this finite group.  The twisted sum $\sum_x \psi^{-1}(Qx) \Phi(x)$ is a character value of this representation, and its nonvanishing can be checked using the character theory of finite groups of Lie type (Deligne--Lusztig theory).

\subsection{Assessment of Repair Difficulty}

\begin{center}
\begin{tabular}{@{}lp{6cm}c@{}}
\toprule
\textbf{Strategy} & \textbf{What It Fixes} & \textbf{Difficulty} \\
\midrule
A (Compact Kirillov) & Issues 1, 2, 3 (unramified + supercuspidal) & Medium \\
B (Multiplicativity) & Issue 4 (non-supercuspidal) & Hard \\
C (Functional equation) & Issue 4 (non-supercuspidal), partially & Medium \\
D (Finite groups) & Issue 5 ($J_K(0)$ nonvanishing) & Medium--Hard \\
\bottomrule
\end{tabular}
\end{center}

Strategies A + C + D together would likely suffice for a complete proof, avoiding the deep test vector theory of Strategy~B.

%======================================================================
\section{Recommended Next Steps}
\label{sec:next}
%======================================================================

\begin{enumerate}
\item \textbf{Priority 1: Repair the Kirillov model argument} (Nodes 1.3.1, 1.3.2).  Use Strategy~A: work in the Whittaker model directly, using the BZ exact sequence to control torus support.  This unblocks the unramified case.

\item \textbf{Priority 2: Prove $J_K(0) \neq 0$ for $n \geq 2$} (Node 1.5.1).  Use Strategy~D: reduce to a character computation on $\GL_n(\fro/\frq)$.  This unblocks the supercuspidal case.

\item \textbf{Priority 3: Handle the non-supercuspidal case} (Nodes 1.4.4.2, 1.5.2).  Use Strategy~C: show $I(s)$ is a nonzero Laurent polynomial with no zeros on $\C^\times$ via the functional equation plus nonvanishing at $s = 1/2$.  If this fails, fall back to Strategy~B (multiplicativity reduction to supercuspidals).

\item \textbf{Priority 4: Finish verification wave 2} (Nodes 1.4.4.1, 1.4.4.2, 1.3, 1.4, 1.5, 1.6, root).  Run adversarial verifiers on all remaining unverified nodes.

\item \textbf{Priority 5: Launch prover wave} for all 57 open challenges.  Address the resolved challenges first (14 already resolved), then tackle the systematic issues.
\end{enumerate}


%======================================================================
\section{Key References}
\label{sec:refs}
%======================================================================

\begin{thebibliography}{99}

\bibitem{JPSS83}
H.~Jacquet, I.~I.~Piatetski-Shapiro, and J.~A.~Shalika,
\emph{Rankin--Selberg convolutions},
Amer.\ J.\ Math.\ \textbf{105} (1983), 367--464.

\bibitem{Matringe13}
N.~Matringe,
\emph{Essential Whittaker functions for $\GL(n)$},
Doc.\ Math.\ \textbf{18} (2013), 1191--1214.

\bibitem{Miyauchi14}
M.~Miyauchi,
\emph{Whittaker functions associated to newforms for $\GL(n)$ over $p$-adic fields},
J.\ Math.\ Soc.\ Japan \textbf{66} (2014), 17--24.

\bibitem{CS80}
W.~Casselman and J.~Shalika,
\emph{The unramified principal series of $p$-adic groups II: The Whittaker function},
Compositio Math.\ \textbf{41} (1980), 207--231.

\bibitem{BZ76}
I.~N.~Bernstein and A.~V.~Zelevinsky,
\emph{Representations of the group $\GL(n, F)$, where $F$ is a local non-archimedean field},
Uspekhi Mat.\ Nauk \textbf{31} (1976), 5--70.

\bibitem{CPS04}
J.~W.~Cogdell and I.~I.~Piatetski-Shapiro,
\emph{Remarks on Rankin--Selberg convolutions},
in \emph{Contributions to Automorphic Forms, Geometry, and Number Theory} (Shalikafest), 2004.

\bibitem{GJ72}
R.~Godement and H.~Jacquet,
\emph{Zeta Functions of Simple Algebras},
Lecture Notes in Mathematics \textbf{260}, Springer, 1972.

\bibitem{Shahidi84}
F.~Shahidi,
\emph{Fourier transforms of intertwining operators and Plancherel measures for $\GL(n)$},
Amer.\ J.\ Math.\ \textbf{106} (1984), 67--111.

\bibitem{Humphries21}
P.~Humphries,
\emph{Test vectors for Rankin--Selberg $L$-functions},
J.\ Number Theory \textbf{220} (2021), 259--279.

\bibitem{AB24}
E.~Assing and V.~Blomer,
\emph{The density conjecture for principal congruence subgroups},
Duke Math.\ J.\ \textbf{173} (2024), 1949--2008.

\bibitem{Tate50}
J.~T.~Tate,
\emph{Fourier Analysis in Number Fields and Hecke's Zeta-Functions},
PhD thesis, Princeton University, 1950.

\bibitem{BK93}
C.~J.~Bushnell and P.~C.~Kutzko,
\emph{The Admissible Dual of $\GL(N)$ via Compact Open Subgroups},
Annals of Mathematics Studies \textbf{129}, Princeton University Press, 1993.

\end{thebibliography}


\newpage
%======================================================================
\appendix
\section{Full Proof Tree}
\label{app:tree}
%======================================================================

The complete proof tree as exported from the adversarial proof framework (\texttt{af status}).
Status key: \textcolor{proved}{\textbf{V}}~=~validated,
\textcolor{pending}{\textbf{P}}~=~pending.
Challenge counts in parentheses: (critical/major/minor).

{\small\begin{verbatim}
1 [V] Root conjecture: For any generic irred. adm. Pi of GL_{n+1}(F),
  |  there exists W in W(Pi, psi^{-1}) such that for every generic
  |  irred. adm. pi of GL_n(F), there exists V in W(pi, psi) for which
  |  the local RS integral is finite and nonzero for all s in C.
  |  Challenges: (0/0/1)
  |
  +-- 1.1 [V] Commutation Identity
  |   W(diag(g,1) u_Q) = psi^{-1}(Q g_{nn}) W(diag(g,1))
  |   Clean, no challenges.
  |
  +-- 1.2 [V] Algebraic Characterization
  |   R(q^{-s}) finite & nonzero for all s iff R = c q^{-ks}
  |   Clean, no challenges.
  |
  +-- 1.3 [P] Test Vector Choice (0/0/1 + upstream)
  |   W = W^circ (new vector of Pi). V depends on pi:
  |   Case 1 (ramified): V = V^circ (new vector of pi)
  |   Case 2 (unramified): V = V_0 (compact Kirillov support)
  |   Challenges: (1/2/2)
  |   |
  |   +-- 1.3.1 [P] Unramified Test Vector Construction
  |   |   phi_0 = char((o^x)^{n-1}), V_0 = corresponding Whittaker fn
  |   |   Properties: torus collapse, membership, K-integral nonvanishing
  |   |   Challenges: (2/1/0)
  |   |   CRITICAL: Kirillov eval off mirabolic; Casselman-Shalika scope
  |   |
  |   +-- 1.3.2 [P] Monomial Property for Unramified pi
  |       Iwasawa decomp + Kirillov support => single torus term
  |       I(s) = W^circ(I) * vol = nonzero constant
  |       Challenges: (2/1/1)
  |       CRITICAL: K-projection dilemma; Kirillov eval
  |
  +-- 1.4 [P] Iwasawa Unfolding (Ramified Case, c(pi) >= 1)
  |   Unfold via g = nak; W^circ factors out; conductor analysis on m_n
  |   Challenges: (0/2/2)
  |   |
  |   +-- 1.4.1 [V] W^circ Factorization
  |   |   diag(k,1) in K_1(p^{c(Pi)}) => W^circ(diag(ak,1))=W^circ(diag(a,1))
  |   |   Validated, no open challenges.
  |   |
  |   +-- 1.4.2 [V] Case (a) Vanishing (m_n > c(pi))
  |   |   Twist trivial => K-average of V^circ = 0 (pi^{K_n} = 0)
  |   |   Validated, no open challenges.
  |   |
  |   +-- 1.4.3 [P] Conductor Analysis (0 <= m_n <= c(pi))
  |   |   Finite Fourier analysis in k_{nn} over GL_n(o/q)
  |   |   Case (b): m_n < 0 => character oscillates => J_K = 0
  |   |   1 <= m_n <= c-1 => K_1(p^{c-m_n})-average vanishes => J_K = 0
  |   |   Result: J_K(m) = 0 unless m_n = 0
  |   |   Challenges: (1/1/1)
  |   |   CRITICAL: GL_n(o/q) factorization invalid for m_n < 0
  |   |
  |   +-- 1.4.4 [P] Torus Sum Reduction (m_n = 0 established)
  |       I(s) = sum over m_1,...,m_{n-1}
  |       Controlled by W^circ and V^circ torus support
  |       Challenges: (1/2/1)
  |       |
  |       +-- 1.4.4.1 [P] Supercuspidal pi: Single Torus Term
  |       |   Compact Kirillov support (K(pi,psi) = S(F^{n-1}\{0}))
  |       |   intersected with W^circ dominance => m_i = 0 for all i
  |       |   I(s) = W^circ(I) * J_K(0) = nonzero constant
  |       |   Challenges: (1/2/1)
  |       |   CRITICAL: phi^circ single-coset claim unproven
  |       |
  |       +-- 1.4.4.2 [P] Non-supercuspidal Ramified pi
  |           Langlands classification: pi = Ind(Delta_1 x...x Delta_r)
  |           Matringe formulas => finite torus sum => Laurent poly
  |           Monomial via epsilon factor identification OR V_mod
  |           Challenges: (2/3/1)
  |           CRITICAL: Monomial property unproven; V_mod deferred
  |
  +-- 1.5 [P] Nonvanishing (Ramified Case)
  |   Case A (supercuspidal): single-term => need J_K(0) != 0
  |   Case B (non-supercuspidal): I(s) = C * epsilon(s, Pi x pi, psi)
  |   Challenges: (2/1/1)
  |   |
  |   +-- 1.5.1 [P] K-integral Nonvanishing (Supercuspidal)
  |   |   J_K(0) = vol(K_1(q)) * sum psi^{-1}(Qx) Phi(x)
  |   |   Claim: Fourier coeff at conductor level is nonzero
  |   |   Challenges: (1/1/1)
  |   |   CRITICAL: Nonvanishing unproven for n >= 2
  |   |
  |   +-- 1.5.2 [P] Epsilon Factor Identification (Non-supercuspidal)
  |       I(s) = C * epsilon(s, Pi x pi, psi) via test vector theory
  |       epsilon is always a nonzero monomial (JPSS 1983)
  |       Challenges: (2/1/2)
  |       CRITICAL: Humphries GL_2-specific; general n unproven
  |
  +-- 1.6 [P] Conclusion (QED)
      Assembles Cases 1 (unramified), 2 (supercuspidal), 3 (non-supercusp.)
      W = W^circ universal; V case-dependent
      Challenges: (3/2/1)
      CRITICAL: Inherits all upstream gaps
\end{verbatim}}


\newpage
%======================================================================
\section{Full Node Statements}
\label{app:nodes}
%======================================================================

This appendix reproduces the complete statement of each node in the proof tree, as stored in the \texttt{af} workspace.

\subsection*{Node 1 --- Root Conjecture}
\textbf{Status:} \textcolor{proved}{Validated} / Clean.
\textbf{Type:} claim.

\noindent\textbf{Statement:}
For any generic irreducible admissible representation $\Pi$ of $\GL_{n+1}(F)$ over a non-archimedean local field~$F$, there exists $W \in \Wh(\Pi, \psi^{-1})$ such that for every generic irreducible admissible representation $\pi$ of $\GL_n(F)$, there exists $V \in \Wh(\pi, \psi)$ for which
\[
  \int_{N_n \backslash \GL_n(F)} W\bigl(\mathrm{diag}(g,1) \cdot u_Q\bigr)\, V(g)\, |\det g|^{s-1/2}\, dg
\]
is finite and nonzero for all $s \in \C$.

\subsection*{Node 1.1 --- Commutation Identity}
\textbf{Status:} \textcolor{proved}{Validated} / Clean.

\noindent\textbf{Statement:}
For $g \in \GL_n(F)$, $W(\mathrm{diag}(g,1)\, u_Q) = \psi^{-1}(Q\, g_{nn})\, W(\mathrm{diag}(g,1))$.  This follows from conjugating $u_Q = I_{n+1} + Q\, E_{n,n+1}$ past $\mathrm{diag}(g,1)$: the resulting element $n'(g,Q) \in N_{n+1}$, and only the $(n,n+1)$-superdiagonal entry contributes to $\psi^{-1}$.

\subsection*{Node 1.2 --- Algebraic Characterization}
\textbf{Status:} \textcolor{proved}{Validated} / Clean.

\noindent\textbf{Statement:}
A rational function $R(q^{-s}) \in \C(q^{-s})$ is finite and nonzero for all $s \in \C$ if and only if $R$ is a nonzero monomial $c\, q^{-ks}$ for some $c \in \C^\times$ and $k \in \Z$.  Proof: the map $s \mapsto q^{-s}$ surjects onto $\C^\times$, and the only rational functions with no zeros or poles on $\C^\times$ are monomials.

\subsection*{Node 1.3 --- Test Vector Choice}
\textbf{Status:} \textcolor{pending}{Pending} / Unresolved.
\textbf{Challenges:} 5 open (1 critical, 2 major, 2 minor).

\noindent\textbf{Statement (summary):}
Set $W = W^\circ$ (essential Whittaker function of $\Pi$).  For each $\pi$:
\begin{itemize}
\item \emph{Ramified} ($c(\pi) \geq 1$): $V = V^\circ$ (new vector of $\pi$).
\item \emph{Unramified} ($c(\pi) = 0$): $V = V_0$, defined via the Kirillov model with support $\phi_0 = \mathbf{1}_{(\fro^\times)^{n-1}}$.
\end{itemize}
For $n = 1$: $V_0(g) = \mathbf{1}_{\fro^\times}(g)$, and the RS integral collapses to $W^\circ(I_2)\, \psi^{-1}(Q) \neq 0$.
For $n \geq 2$: deferred to Nodes 1.3.1 and 1.3.2.

\subsection*{Node 1.3.1 --- Unramified Test Vector Construction}
\textbf{Status:} \textcolor{pending}{Pending} / Unresolved.
\textbf{Challenges:} 3 open (2 critical, 1 major).

\noindent\textbf{Statement (summary):}
Defines $\phi_0 = \mathbf{1}_{(\fro^\times)^{n-1}} \in \mathcal{S}(F^{n-1} \setminus \{0\})$ and $V_0$ as the corresponding Whittaker function.  Claims three properties:
(P1)~Torus support collapse: Casselman--Shalika ($m_i \geq 0$) combined with Kirillov support ($m_i \leq 0$) forces $m_i = 0$.
(P2)~Membership in $\Wh(\pi, \psi)$ via the Bernstein--Zelevinsky embedding.
(P3)~Nonvanishing: $\int_{K_n} V_0(k)\, dk > 0$.

\subsection*{Node 1.3.2 --- Monomial Property for Unramified $\pi$}
\textbf{Status:} \textcolor{pending}{Pending} / Unresolved.
\textbf{Challenges:} 4 open (2 critical, 1 major, 1 minor).

\noindent\textbf{Statement (summary):}
For unramified $\pi$ with $V = V_0$: the commutation identity gives the twist $\psi^{-1}(Q g_{nn})$, which is trivial since $Q \in \fro^\times$; the Kirillov support forces $m_i = 0$ for all $i$; the integral evaluates to $I(s) = W^\circ(I_{n+1}) \cdot \vol > 0$, a nonzero constant.

\subsection*{Node 1.4 --- Iwasawa Unfolding (Ramified Case)}
\textbf{Status:} \textcolor{pending}{Pending} / Unresolved.
\textbf{Challenges:} 4 open (0 critical, 2 major, 2 minor).

\noindent\textbf{Statement (summary):}
Unfolds the integral via Iwasawa decomposition $g = nak$.  Four steps: (1)~$W^\circ$ factorization, (2)~$K_1(\frq)$-invariance, (3)~conductor analysis on $m_n$ yielding $J_K(m) = 0$ unless $m_n = 0$, (4)~remaining torus sum in $m_1, \ldots, m_{n-1}$.

\subsection*{Node 1.4.1 --- $W^\circ$ Factorization}
\textbf{Status:} \textcolor{proved}{Validated} / Unresolved (taint from parent).

\noindent\textbf{Statement:}
$\mathrm{diag}(k, 1) \in K_1(\frp^{c(\Pi)})$ for all $k \in K_n$, so $W^\circ(\mathrm{diag}(ak, 1)) = W^\circ(\mathrm{diag}(a, 1))$.

\subsection*{Node 1.4.2 --- Case (a) Vanishing}
\textbf{Status:} \textcolor{proved}{Validated} / Unresolved (taint from parent).

\noindent\textbf{Statement:}
When $m_n > c(\pi)$, the twist is trivial; $J_K(m) = \int_{K_n} V^\circ(ak)\, dk = 0$ since $\pi^{K_n} = 0$ for $c(\pi) \geq 1$.

\subsection*{Node 1.4.3 --- Conductor Analysis}
\textbf{Status:} \textcolor{pending}{Pending} / Unresolved.
\textbf{Challenges:} 3 open (1 critical, 1 major, 1 minor).

\noindent\textbf{Statement (summary):}
Decomposes $\GL_n(\fro/\frq)$ by the $(n,n)$-entry.  The finite Fourier transform in $k_{nn}$ over $\fro/\frq$ vanishes unless the additive level matches the conductor level.  Cases (b)~$m_n < 0$ and $1 \leq m_n \leq c-1$ give $J_K = 0$.  Therefore $J_K(m) = 0$ unless $m_n = 0$.

\subsection*{Node 1.4.4 --- Torus Sum Reduction}
\textbf{Status:} \textcolor{pending}{Pending} / Unresolved.
\textbf{Challenges:} 4 open (1 critical, 2 major, 1 minor).

\noindent\textbf{Statement (summary):}
With $m_n = 0$, the sum over $m_1, \ldots, m_{n-1}$ is controlled by the torus support of $W^\circ$ and $V^\circ$.  JPSS rationality, pole analysis, and zero analysis are invoked.  Splits into supercuspidal (1.4.4.1) and non-supercuspidal (1.4.4.2).

\subsection*{Node 1.4.4.1 --- Supercuspidal $\pi$}
\textbf{Status:} \textcolor{pending}{Pending} / Unresolved.
\textbf{Challenges:} 4 open (1 critical, 2 major, 1 minor).

\noindent\textbf{Statement (summary):}
For supercuspidal $\pi$: $K(\pi, \psi) = \mathcal{S}(F^{n-1} \setminus \{0\})$; $V^\circ$ has compact Kirillov support.  Combined with $W^\circ$ dominance, the torus sum collapses to $m = 0$.  $I(s) = W^\circ(I_{n+1}) \cdot J_K(0) \neq 0$.

\subsection*{Node 1.4.4.2 --- Non-supercuspidal Ramified $\pi$}
\textbf{Status:} \textcolor{pending}{Pending} / Unresolved.
\textbf{Challenges:} 6 open (2 critical, 3 major, 1 minor).

\noindent\textbf{Statement (summary):}
Via Langlands classification and Matringe's explicit formulas, the torus sum is finite.  $I(s)$ is a Laurent polynomial.  Monomial property via epsilon factor identification (primary) or $V_{\mathrm{mod}}$ construction (alternative).

\subsection*{Node 1.5 --- Nonvanishing}
\textbf{Status:} \textcolor{pending}{Pending} / Unresolved.
\textbf{Challenges:} 4 open (2 critical, 1 major, 1 minor).

\noindent\textbf{Statement (summary):}
Case A (supercuspidal): $J_K(0) \neq 0$ (Node 1.5.1).  Case B (non-supercuspidal): $I(s) = C \cdot \varepsilon(s, \Pi \times \pi, \psi)$ (Node 1.5.2).

\subsection*{Node 1.5.1 --- $K$-integral Nonvanishing}
\textbf{Status:} \textcolor{pending}{Pending} / Unresolved.
\textbf{Challenges:} 3 open (1 critical, 1 major, 1 minor).

\noindent\textbf{Statement (summary):}
$J_K(0) = \vol(K_1(\frq)) \cdot \sum_{x \in (\fro/\frq)^\times} \psi^{-1}(Qx)\, \Phi(x)$.  The Fourier coefficient at conductor level is claimed nonzero because $V^\circ$ has ``maximal Fourier content'' at this level.

\subsection*{Node 1.5.2 --- Epsilon Factor Identification}
\textbf{Status:} \textcolor{pending}{Pending} / Unresolved.
\textbf{Challenges:} 5 open (2 critical, 1 major, 2 minor).

\noindent\textbf{Statement (summary):}
Identifies $I(s) = C \cdot \varepsilon(s, \Pi \times \pi, \psi)$ via JPSS functional equation and test vector theory (Humphries, Assing--Blomer).  Alternative: $I(s)$ is a Laurent polynomial with no zeros on $\C^\times$, hence a monomial.

\subsection*{Node 1.6 --- Conclusion (QED)}
\textbf{Status:} \textcolor{pending}{Pending} / Unresolved.
\textbf{Type:} qed.
\textbf{Challenges:} 6 open (3 critical, 2 major, 1 minor).

\noindent\textbf{Statement (summary):}
Assembles three cases:
\begin{enumerate}
\item Unramified $\pi$: $V = V_0$, compact Kirillov support $\Rightarrow$ constant monomial.
\item Supercuspidal ramified $\pi$: $V = V^\circ$, torus collapse $\Rightarrow$ constant monomial.
\item Non-supercuspidal ramified $\pi$: $V = V^\circ$ (or $V_{\mathrm{mod}}$), epsilon factor identification $\Rightarrow$ monomial.
\end{enumerate}
In all cases, $W = W^\circ$ is universal.


%======================================================================
\section{Complete Challenge List}
\label{app:challenges}
%======================================================================

The following table lists all 57 open challenges, sorted by node and severity.

{\footnotesize
\begin{longtable}{@{}p{2.2cm}p{1.1cm}p{1.2cm}p{8cm}@{}}
\toprule
\textbf{Challenge ID} & \textbf{Node} & \textbf{Severity} & \textbf{Summary} \\
\midrule
\endhead
ch-fa48e6d4 & 1 & minor & Missing conductor specification \\
\midrule
ch-65a665fb & 1.3 & critical & GL$_1$ Kirillov model error for $n=1$ \\
ch-d5155fdf & 1.3 & major & Kirillov model dimension mismatch \\
ch-38dd840c & 1.3 & major & $W^\circ$ factorization claim for unramified case \\
ch-09fcf9a7 & 1.3 & minor & $n=1$ computation includes spurious terms \\
ch-7d534189 & 1.3 & minor & Missing dependency declarations \\
\midrule
ch-fc31ed05 & 1.3.1 & critical & Kirillov model does not evaluate pointwise off mirabolic \\
ch-b23b0a14 & 1.3.1 & critical & Casselman--Shalika invoked for possibly ramified $\Pi$ \\
ch-0a34647 & 1.3.1 & major & $K$-integral nonvanishing (P3) uses incorrect Kirillov eval \\
\midrule
ch-76fbc55c & 1.3.2 & critical & Kirillov evaluation $V_0(ak)$ invalid off mirabolic \\
ch-8a091889 & 1.3.2 & critical & $K$-projection dilemma \\
ch-9741de17 & 1.3.2 & major & Casselman--Shalika scope (unramified $\Pi$ only) \\
ch-cc673ffe & 1.3.2 & minor & Incoherent $\phi_0$ definition \\
\midrule
ch-06e16afe & 1.4 & major & Step 3(b) vanishing for $m_n < 0$ \\
ch-6c6aeb22 & 1.4 & major & $K_1(\frq)$-invariance claim incomplete \\
ch-0282f20b & 1.4 & minor & Missing dependency declarations \\
ch-def647a0 & 1.4 & minor & Step 3(c) references Node 1.4.4 incorrectly \\
\midrule
ch-5ae3ea35 & 1.4.3 & critical & $\GL_n(\fro/\frq)$ factorization invalid for $m_n < 0$ \\
ch-396727e8 & 1.4.3 & major & Case $1 \leq m_n \leq c-1$: unjustified Fourier vanishing \\
ch-6294364b & 1.4.3 & minor & Inference type and missing dependencies \\
\midrule
ch-cb72c2c9 & 1.4.4 & critical & Zero analysis mathematically wrong \\
ch-6a7dcbea & 1.4.4 & major & Pole analysis non-rigorous \\
ch-b61c1c23 & 1.4.4 & major & Incoherent argument structure \\
ch-1ac6ede2 & 1.4.4 & minor & Missing dependency declarations \\
\midrule
ch-f4770ce1 & 1.4.4.1 & critical & $\phi^\circ$ single-coset claim unproven \\
ch-263b8a5f & 1.4.4.1 & major & $K$-integral involves off-torus terms \\
ch-78d666e9 & 1.4.4.1 & major & $W^\circ$ dominance $m_i \geq 0$ not established for ramified $\Pi$ \\
ch-8b15b334 & 1.4.4.1 & minor & Single-term nonvanishing deferred \\
\midrule
ch-d7e5d54b & 1.4.4.2 & critical & Monomial property via Steps (A)--(C) unproven \\
ch-ed835d2c & 1.4.4.2 & critical & Alternative argument deferred to child node \\
ch-34588e04 & 1.4.4.2 & major & Steinberg pattern claim incorrect \\
ch-8fe1adbc & 1.4.4.2 & major & Matringe Thm 5.1 product formula misquoted \\
ch-74681215 & 1.4.4.2 & major & Finiteness argument gap \\
ch-37463a34 & 1.4.4.2 & minor & Missing dependency declarations \\
\midrule
ch-ba977748 & 1.5 & critical & Case A delegates to Node 1.5.1 (unproven) \\
ch-d75f44b4 & 1.5 & critical & Case B: $I(s) = C \cdot \varepsilon$ identification unproven \\
ch-3368c7e2 & 1.5 & major & Formal dependency declarations still missing \\
ch-c9014d35 & 1.5 & minor & Imprecise claim about $I(s)$ nonvanishing \\
\midrule
ch-3463f8fd & 1.5.1 & critical & Fourier coefficient nonvanishing unproven for $n \geq 2$ \\
ch-e9fa9101 & 1.5.1 & major & Fiber sum domain restriction unjustified \\
ch-eb07d1ae & 1.5.1 & minor & Inference type mislabeled \\
\midrule
ch-8de63184 & 1.5.2 & critical & Test vector identification unproven for general $n$ \\
ch-97707406 & 1.5.2 & critical & Fundamental logical error in Step 4 \\
ch-20cf2e25 & 1.5.2 & major & Step 4 invokes Node 1.5.1 logic (supercuspidal-only) \\
ch-77ccdb92 & 1.5.2 & minor & Inference type mislabeled \\
ch-b707611e & 1.5.2 & minor & Missing dependency declarations \\
\midrule
ch-310f1a3f & 1.6 & critical & Case 3 (non-supercuspidal) inherits epsilon factor gap \\
ch-33c5f628 & 1.6 & critical & Case 1 (unramified) inherits Kirillov model gap \\
ch-7d534189 & 1.6 & critical & Dependencies not formally declared \\
ch-b92f26f2 & 1.6 & major & Case 2 (supercuspidal) depends on unproven $J_K(0) \neq 0$ \\
ch-8ee707ba & 1.6 & major & RS conductor formula imprecise \\
ch-17d96c97 & 1.6 & minor & Ambiguity in $V$ vs $V_{\mathrm{mod}}$ for Case 3 \\
\bottomrule
\end{longtable}
}


%======================================================================
\section{Definitions and External References}
\label{app:defs}
%======================================================================

\subsection*{Definitions Registered in \texttt{af}}

\begin{tabular}{@{}ll@{}}
\toprule
\textbf{Name} & \textbf{Concept} \\
\midrule
\texttt{non\_archimedean\_local\_field} & $F$, $\fro$, $\frp$, $q$ \\
\texttt{generic\_representation} & Generic irreducible admissible rep of $\GL_r(F)$ \\
\texttt{Whittaker\_model} & $\Wh(\Pi, \psi^{-1})$ \\
\texttt{conductor\_ideal} & $\frq = \frp^{c(\pi)}$ \\
\texttt{Rankin\_Selberg\_integral} & $I(s, W, V)$ \\
\texttt{upper\_triangular\_unipotent} & $N_r \leq \GL_r(F)$ \\
\bottomrule
\end{tabular}

\subsection*{External References Registered in \texttt{af}}

\begin{tabular}{@{}ll@{}}
\toprule
\textbf{Name} & \textbf{Source} \\
\midrule
Whittaker models & Cogdell--Piatetski-Shapiro (2004) \\
Rankin--Selberg theory & JPSS (1983) \\
Essential Whittaker functions & Matringe (2013) \\
Newforms for $\GL(n)$ & Miyauchi (2014) \\
Epsilon factors & Godement--Jacquet (1972) \\
Tate thesis & Tate (1950) \\
\bottomrule
\end{tabular}


\end{document}
