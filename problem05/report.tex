\documentclass[11pt,a4paper]{article}

\usepackage[margin=2.5cm]{geometry}
\usepackage{amsmath,amssymb,amsthm}
\usepackage{hyperref}
\usepackage{enumitem}

\hypersetup{
  colorlinks=true,
  linkcolor=blue!70!black,
  citecolor=blue!70!black,
  urlcolor=blue!70!black
}

\newtheorem{theorem}{Theorem}[section]
\newtheorem{lemma}[theorem]{Lemma}
\newtheorem{proposition}[theorem]{Proposition}
\newtheorem{corollary}[theorem]{Corollary}
\newtheorem{definition}[theorem]{Definition}
\newtheorem{remark}[theorem]{Remark}

\title{Critical Comparison: Automated vs.\ Official Solution\\
for the Slice Filtration for Incomplete Transfer Systems\\[6pt]
\large Problem~5 of the First Proof Benchmark}
\author{Comparative Analysis Report}
\date{February 14, 2026}

\begin{document}
\maketitle

\tableofcontents
\bigskip

%======================================================================
\section{Introduction and Problem Statement}
%======================================================================

Problem~5 of the First Proof benchmark~\cite{FirstProof2026} asks:

\begin{quote}
Fix a finite group~$G$ and let $\mathcal{O}$ denote an incomplete transfer system associated to an $N_\infty$ operad.
\begin{enumerate}
\item \textbf{Define} the slice filtration on the $G$-equivariant stable category adapted to~$\mathcal{O}$.
\item \textbf{State and prove} a characterization of the $\mathcal{O}$-slice connectivity of a connective $G$-spectrum in terms of the geometric fixed points.
\end{enumerate}
\end{quote}

The problem sits at the intersection of equivariant stable homotopy theory and the combinatorics of transfer systems.  The standard (complete) slice filtration, introduced by Hill--Hopkins--Ravenel~\cite{HHR2016} in their resolution of the Kervaire invariant one problem, is one of the most powerful tools in equivariant homotopy theory.  Generalizing it to incomplete transfer systems---where only a subset of norm maps are available---requires careful attention to which ``slice cells'' generate the filtration.

This report compares the official solution by Blumberg, Hill, and Lawson (extracted from~\cite{FirstProof2026}) with the automated adversarial-framework (\texttt{af}) proof attempt, which achieved a quality score of 30/100 with 0\% node completion across 11 nodes and 45 open challenges.

%======================================================================
\section{The Official Solution}
%======================================================================

\subsection{Transfer systems and indexing systems}

The official solution begins with a clean formulation of the combinatorial input.

\begin{definition}[Transfer system, {\cite[Def.~1.1]{FirstProof2026}}]
A \emph{transfer system} on~$G$ is a partial order~$\to$ on $\mathrm{Sub}(G)$ satisfying:
\begin{enumerate}
\item it refines subgroup inclusion: $H \to K$ implies $H \subseteq K$;
\item it is conjugation invariant;
\item it is closed under restriction: if $H \to K$ and $J \subseteq K$, then $H \cap J \to J$.
\end{enumerate}
\end{definition}

\begin{definition}[Admissible sets and indexing systems, {\cite[Def.~1.2]{FirstProof2026}}]
Let $\mathcal{O}$ be a transfer system.  A finite $H$-set
$T = \coprod_i H/K_i$
is \emph{admissible} for~$\mathcal{O}$ if $K_i \to H$ for all~$i$.  The collection $\mathcal{O}(H)$ of admissible $H$-sets, as $H$ varies, gives an \emph{indexing system}.
\end{definition}

The key conceptual step is that the correct input to the slice filtration is not a ``family of subgroups'' extracted from the transfer system, but rather the \emph{indexing system}~$\mathcal{O}(H)$ of admissible finite $H$-sets for each subgroup~$H$.  This indexing system encodes which norm maps $N_K^H$ are available and, critically, which \emph{composite} norms $N^T$ can be formed.

\subsection{The $\mathcal{O}$-slice filtration via nullification}

The official solution defines the filtration using equivariant nullification (a form of Bousfield localization), not cellularization.

\begin{definition}[$\mathcal{O}$-slice connectivity, {\cite[Def.~1.4]{FirstProof2026}}]
If $\mathcal{O}$ is an indexing system, let $\tau_{\geq n}^{\mathcal{O}}$ be the equivariant localizing subcategory of $\mathrm{Sp}^G$ generated by
\[
\bigl\{G_+ \underset{H}{\otimes} N^T S^1 \;\big|\; T \in \mathcal{O}(H),\; |T| \geq n \bigr\}.
\]
\end{definition}

\begin{remark}
By the canonical identification $N^T S^1 \cong S^{\mathbb{R}\cdot T}$, the generators are representation spheres of permutation representations for admissible sets.  This gives a clean, representation-theoretic description of the generating cells.
\end{remark}

The truncation functors $P_{\mathcal{O}}^n$ are defined as nullifications killing $\tau_{\geq (n+1)}^{\mathcal{O}}$, and the $n$th $\mathcal{O}$-slice is the fiber of $P_{\mathcal{O}}^n \to P_{\mathcal{O}}^{n-1}$.  Proposition~1.7 of the official solution establishes that the tower is exhaustive (the inverse limit recovers the spectrum) and Corollary~1.10 shows that the category of $\mathcal{O}$-$k$-slices is discrete.

\subsection{The characteristic function and connectivity characterization}

The official solution introduces a crucial combinatorial invariant:

\begin{definition}[Characteristic function, {\cite[Def.~2.1]{FirstProof2026}}]
For a transfer system~$\mathcal{O}$, define
\[
\chi^{\mathcal{O}}(H) = \min\{K : K \to H\} = \bigcap_{K \to H} K.
\]
\end{definition}

This function captures the ``most efficient norm'' available at each level of the subgroup lattice.  The main theorem is:

\begin{theorem}[{\cite[Thm.~2.7]{FirstProof2026}}]
A $G$-spectrum $E$ is in $\tau_{\geq n}^{\mathcal{O}}$ if and only if for all $H \subseteq G$,
\[
[H : \chi^{\mathcal{O}}(H)] \cdot \mathrm{gconn}(E)(H) \geq n,
\]
where $\mathrm{gconn}(E)(H) = \mathrm{conn}(\Phi^H(E))$ is the connectivity of the geometric fixed points.
\end{theorem}

The proof strategy follows Hill--Yarnall~\cite{HillYarnall2018}:
\begin{itemize}
\item \textbf{Forward direction} (Lemma~2.3): Reduce to generators $N^T S^1$ via the fact that geometric fixed points preserve colimits. Decompose $T = \sum_H n_H \, G/H$ and use the inequality $[G:\chi^{\mathcal{O}}(G)] \cdot |T/G| \geq |T| \geq n$.
\item \textbf{Reverse direction} (Theorem~2.7): Use the isotropy separation sequence $E\mathcal{P}_+ \otimes E \to E \to \widetilde{E}\mathcal{P} \otimes E$ and downward induction on the subgroup lattice.  Lemma~2.6 handles the ``geometric part'' $\widetilde{E}\mathcal{P} \otimes E$ using geometric Mackey functors, and Lemma~2.5 handles the ``family part'' $E\mathcal{P}_+ \otimes E$ by restriction to proper subgroups.
\end{itemize}

\subsection{Structural features of the official solution}

Several aspects of the official solution are worth highlighting:

\begin{enumerate}
\item \textbf{No family extraction.}  The solution never attempts to extract a single ``family of subgroups'' from $\mathcal{O}$.  Instead, it works with the full indexing system $\{\mathcal{O}(H)\}_{H \leq G}$ and the $T$-norms $N^T$ for admissible sets.

\item \textbf{Generators are norm spheres, not regular-representation cells.}  The generators $G_+ \otimes_H N^T S^1$ are indexed by admissible $H$-sets $T$, not by subgroups $K$ with an admissibility condition.

\item \textbf{The characteristic function $\chi^{\mathcal{O}}$ is a derived quantity.}  It emerges from the proof analysis, not as a definition imposed at the outset.

\item \textbf{Equivariant nullification, not cellularization.}  The truncation $P_{\mathcal{O}}^n$ is a nullification functor (left Bousfield localization at terminal maps), consistent with the standard HHR approach.

\item \textbf{Close adherence to Hill--Yarnall.}  The proof is explicitly modeled on~\cite{HillYarnall2018}, adapting each step to the indexed setting.
\end{enumerate}

%======================================================================
\section{The Automated \texttt{af} Attempt}
%======================================================================

\subsection{Overview}

The \texttt{af} (Adversarial Proof Framework) attempt on Problem~5 consisted of a single session producing 11 nodes in the proof tree, 5 definitions, and 6 external references.  A full breadth-first verification wave was conducted, resulting in \textbf{45 open challenges} (18 critical, 17 major, 8 minor, 2 dependency-only).  No node was validated; the proof was assessed at 0\% completion with a quality score of 30/100.

\subsection{The proof tree}

The 11-node tree was structured as follows:

\begin{center}
\begin{tabular}{llp{8cm}}
\textbf{Node} & \textbf{Status} & \textbf{Content} \\
\hline
1 & pending (root) & Root conjecture \\
1.1 & 4 challenges & Define $\mathcal{O}$-admissible family $\mathcal{F}_{\mathcal{O}}$ \\
1.2 & 3 challenges & Define $\mathcal{O}$-slice cells \\
1.3 & 6 challenges & Construct $\mathcal{O}$-slice filtration \\
1.4 & 4 challenges & Functoriality and monotonicity \\
1.5 & 4 challenges & Geometric fixed points of $\mathcal{O}$-slice cells \\
1.6 & 4 challenges & State $\mathcal{O}$-Slice Connectivity Theorem \\
1.7 & 6 challenges & Forward direction proof \\
1.8 & 5 challenges & Reverse direction proof \\
1.9 & 4 challenges & Verification/test cases \\
1.10 & 5 challenges & QED assembly \\
\end{tabular}
\end{center}

\subsection{The fatal foundational error: the $\mathcal{F}_{\mathcal{O}}$ collapse}

Node~1.1 introduced the following definition (recorded as definition \texttt{3722163dd0d58d3c} in the \texttt{af} ledger):

\begin{quote}
\textbf{af Definition} (O-admissible family): For a transfer system $\mathcal{O}$ on~$G$, define $\mathrm{Sub}_{\mathcal{O}}^{\mathrm{tr}}(G) = \{H \leq G : \mathrm{tr}_H^G \text{ is a morphism in } \mathcal{O}\}$.  The $\mathcal{O}$-admissible family $\mathcal{F}_{\mathcal{O}}$ is the family of subgroups generated by $\mathrm{Sub}_{\mathcal{O}}^{\mathrm{tr}}(G)$.
\end{quote}

This definition is \textbf{vacuous}: since $\mathrm{tr}_G^G = \mathrm{id}$ is a morphism in every transfer system, $G \in \mathrm{Sub}_{\mathcal{O}}^{\mathrm{tr}}(G)$ always.  The family generated by $\{G\}$ is $\mathrm{Sub}(G)$ itself (every subgroup is subconjugate to~$G$).  Therefore $\mathcal{F}_{\mathcal{O}} = \mathrm{Sub}(G)$ for \emph{every} transfer system $\mathcal{O}$, regardless of how incomplete it is.

This error propagates to every downstream node:
\begin{itemize}
\item Node~1.2 (slice cells): Since every subgroup is in $\mathcal{F}_{\mathcal{O}}$, the ``$\mathcal{O}$-restricted'' slice cells are exactly the standard HHR slice cells.
\item Node~1.3 (filtration): The resulting filtration is the standard regular slice filtration, with no dependence on~$\mathcal{O}$.
\item Nodes~1.6--1.8 (connectivity theorem): Any statement about $\mathcal{O}$-connectivity reduces to the standard result of Hill--Yarnall~\cite{HillYarnall2018}, making the ``generalization'' trivially equivalent to the known theorem.
\end{itemize}

\subsection{Additional errors identified by the verification wave}

Beyond the foundational collapse, the verifiers identified several independent errors:

\begin{enumerate}
\item \textbf{Wrong geometric fixed-point formula} (Node~1.5, critical):
The node claimed $\Phi^H(G_+ \wedge_K S^{m\rho_K}) = S^m$ for $H$ subconjugate to~$K$.  The correct answer, by the double coset formula, is a \emph{wedge} of spheres:
\[
\Phi^H(G_+ \wedge_K S^{m\rho_K}) = \bigvee_{[g] \in H \backslash G / K,\, H \leq gKg^{-1}} S^{m[gKg^{-1}:H]}.
\]
The single-sphere answer holds only when $H$ is conjugate to~$K$.

\item \textbf{Localizing vs.\ colocalizing confusion} (Node~1.3, critical):
The node defined $\tau_{\geq n}^{\mathcal{O}}$ as a right orthogonal complement $\{X : [\Sigma, X]^G = 0\}$, which is a \emph{colocalizing} subcategory.  The correct construction (as in the official solution) uses a \emph{localizing} subcategory generated by cells, with the truncation given by nullification.

\item \textbf{Regular vs.\ general slice cells} (Node~1.2, critical):
The node defined only cells of the form $G_+ \wedge_H S^{m\rho_H}$ (regular slice cells) but claimed to define ``standard HHR slice cells,'' which use arbitrary representations.

\item \textbf{RO($H$)-graded to integer-graded gap} (Nodes~1.7, 1.8, major):
Both proof-direction nodes required transitioning between $\mathrm{RO}(H)$-graded homotopy groups and integer-graded geometric fixed points, but neither contained any argument for this step.

\item \textbf{No proof content} (Nodes~1.3, 1.7, 1.8, major):
These leaf nodes contained only proof sketches with no actual mathematical arguments.

\item \textbf{No dependency declarations} (all nodes):
No node declared dependencies on any other, rendering taint propagation inoperative.
\end{enumerate}

%======================================================================
\section{Critical Comparison}
%======================================================================

\subsection{Why the \texttt{af} used the wrong definition}

The root cause of the \texttt{af} failure is a \textbf{conceptual misidentification}: the automated system attempted to extract a \emph{family of subgroups} from the transfer system $\mathcal{O}$, and then restrict the standard slice cells to that family.  This strategy is natural from a superficial reading of the literature---the standard slice filtration does use a family (namely, all of $\mathrm{Sub}(G)$)---but it fundamentally misunderstands what varies when the transfer system is incomplete.

In the complete case, the generating cells are $G_+ \wedge_H S^{n\rho_H}$ for \emph{all} subgroups~$H$ and all dimensions~$n$.  The \texttt{af} strategy was to restrict the ``which $H$'' part, keeping the cell shape $S^{n\rho_H}$ fixed.  But this is the wrong axis of variation:

\begin{center}
\begin{tabular}{lll}
& \textbf{\texttt{af} approach} & \textbf{Official approach} \\
\hline
\textbf{Varies:} & Which subgroups $H$ & Which $H$-sets $T$ \\
\textbf{Fixed:} & Cell shape $S^{n\rho_H}$ & Framework (norm $N^T S^1$) \\
\textbf{Input:} & Family $\mathcal{F}_{\mathcal{O}} \subseteq \mathrm{Sub}(G)$ & Indexing system $\{\mathcal{O}(H)\}$ \\
\textbf{Collapses:} & Yes ($\mathcal{F}_{\mathcal{O}} = \mathrm{Sub}(G)$ always) & No \\
\end{tabular}
\end{center}

The official solution recognizes that an incomplete transfer system does not restrict \emph{which subgroups appear} but rather \emph{which composite norms can be formed}.  The generators $N^T S^1$ for admissible $T \in \mathcal{O}(H)$ capture the correct combinatorial information.  Different transfer systems yield genuinely different collections of admissible sets, and hence different filtrations, without any collapse.

\subsection{How the official approach avoids the collapse}

The official approach avoids the collapse through three key design choices:

\begin{enumerate}
\item \textbf{Working with $H$-sets, not subgroups.}  An admissible $H$-set $T = \coprod_i H/K_i$ requires $K_i \to H$ for all~$i$.  This is a condition on each orbit individually, not on the overall family.  The identity $G \to G$ does not force anything about the admissibility of $G$-sets with orbits $G/K$ for small~$K$.

\item \textbf{Using $T$-norms $N^T$ as generators.}  The cardinality $|T|$ of an admissible set~$T$ is what determines slice level, and this depends on the full structure of the indexing system, not just the top-level subgroups.

\item \textbf{The characteristic function $\chi^{\mathcal{O}}$.}  The minimum subgroup $\chi^{\mathcal{O}}(H) = \min\{K : K \to H\}$ distills the indexing system into a single function that directly controls connectivity.  Incomplete transfer systems have $\chi^{\mathcal{O}}(H)$ strictly larger than the trivial group, yielding a genuinely weaker connectivity condition.
\end{enumerate}

\subsection{The conceptual gap: indexing systems vs.\ families}

The deepest lesson from this comparison is that transfer systems should be viewed through the lens of \emph{indexing systems} (collections of admissible finite sets), not families of subgroups.  This perspective, developed by Blumberg--Hill~\cite{BlumbergHill2015} and refined by Rubin~\cite{Rubin2021} and Balchin--Barnes--Roitzheim~\cite{BBR2021}, is essential for any generalization of equivariant constructions to the incomplete setting.

The \texttt{af} system, lacking this conceptual framework, defaulted to the most obvious parameterization (families of subgroups) and produced a definition that collapsed.  The HANDOFF document shows awareness of the problem---it lists ``use norms instead of transfers'' as a possible repair---but the system was unable to execute this repair, likely because it would require restructuring the entire proof tree.

%======================================================================
\section{Salvageability Assessment}
%======================================================================

We now assess each of the 11 \texttt{af} nodes for salvageable content.

\begin{enumerate}[label=\textbf{Node 1.\arabic*:}, leftmargin=3.5em]

\item[\textbf{Node 1 (root):}] The conjecture statement is correct (it matches the problem statement).  \textbf{Salvageable.}

\item \textbf{$\mathcal{F}_{\mathcal{O}}$ definition.}  Fundamentally broken; the entire approach of extracting a family is wrong.  The repair suggestions in the HANDOFF (use norms, exclude $G$, use Rubin's classification) were on the right track but were never validated.  \textbf{Not salvageable.}

\item \textbf{$\mathcal{O}$-slice cells.}  Doubly broken: uses the wrong parameterization (family restriction) and confuses regular vs.\ general cells.  However, the \emph{idea} that one should restrict the generating cells based on~$\mathcal{O}$ is correct.  \textbf{Partially salvageable} (the qualitative idea, not the definition).

\item \textbf{Filtration construction.}  Confuses localizing and colocalizing.  The statement that one obtains a filtration by Bousfield-type methods is correct in spirit.  \textbf{Partially salvageable} (the high-level strategy).

\item \textbf{Functoriality.}  The claim that the filtration is functorial in~$\mathcal{O}$ (with respect to refinement) is correct and appears in the official solution.  However, the proof sketch depends on the broken definition.  \textbf{Statement salvageable; proof not.}

\item \textbf{Geometric fixed points.}  Contains a wrong formula.  The correct formula involves a wedge over double cosets, which the verifier identified.  The official solution handles this differently, working with $\Phi^G(N^T S^1) = S^{|T/G|}$ directly.  \textbf{Not salvageable} (both the formula and the approach are wrong for the official framework).

\item \textbf{Connectivity theorem statement.}  The form of the statement---$\mathcal{O}$-slice connectivity characterized by geometric fixed-point connectivity---is correct.  However, the specific formula would need to involve $\chi^{\mathcal{O}}(H)$ rather than any family-based quantity.  \textbf{Partially salvageable} (the qualitative statement).

\item \textbf{Forward direction.}  No proof content; requires the RO$(H)$-to-integer transition.  The official solution handles this via a direct computation on generators.  \textbf{Not salvageable.}

\item \textbf{Reverse direction.}  No proof content.  The official solution uses isotropy separation and downward induction, which is the correct approach.  The HANDOFF mentions this strategy but provides no details.  \textbf{Not salvageable} (no content to salvage).

\item \textbf{Test cases.}  Verification against $G = C_p$ is useful and was partially done (the verifier caught the collapse).  \textbf{Partially salvageable} as a testing methodology.

\item[\textbf{Node 1.10:}] \textbf{QED assembly.}  No content.  \textbf{Not salvageable.}

\end{enumerate}

\medskip

\noindent\textbf{Summary:} Of the 11 nodes, the root statement is correct, 4 nodes contain partially salvageable qualitative ideas, and the remaining 6 nodes are not salvageable.  No node contains a correct, complete mathematical argument.

%======================================================================
\section{The Role of Domain Expertise}
%======================================================================

The HANDOFF document itself flagged the attempt as needing a ``restart,'' and the quality score of 30/100 makes this the least successful of the \texttt{af} attempts across the First Proof benchmark.  Several factors contributed to this failure:

\subsection{Specialized vocabulary with subtle distinctions}

Equivariant homotopy theory has an unusually dense web of closely related but distinct concepts:
\begin{itemize}
\item Transfer maps vs.\ norm maps vs.\ restriction maps
\item Families vs.\ indexing systems vs.\ transfer systems
\item Localizing vs.\ colocalizing subcategories
\item Regular vs.\ general slice cells
\item $\mathrm{RO}(G)$-graded vs.\ integer-graded homotopy groups
\end{itemize}
The \texttt{af} system conflated multiple pairs from this list.  An expert in equivariant homotopy theory would immediately recognize these distinctions; the automated system treated them as interchangeable.

\subsection{The indexing-system perspective is non-obvious}

The key insight of the official solution---that one should index generators by admissible $H$-sets rather than by admissible subgroups---requires familiarity with the Blumberg--Hill theory of $N_\infty$ operads and the Rubin classification of indexing systems.  This is a relatively recent development (2015--2021) in a specialized subfield.  The \texttt{af} system apparently had access to the relevant references but was unable to extract the correct conceptual framework from them.

\subsection{Verification caught the problem but could not fix it}

The adversarial verification wave performed well in \emph{detecting} the collapse: verifier agents correctly identified that $\mathcal{F}_{\mathcal{O}} = \mathrm{Sub}(G)$ for every transfer system.  However, the system was unable to \emph{repair} the error, because the fix requires not a local patch but a fundamental reconceptualization of the approach.  The HANDOFF document lists possible repairs, including ``use norms instead of transfers'' and ``use Rubin's classification,'' suggesting the system was aware of the correct direction.  But translating this awareness into a working definition and restructured proof tree required domain expertise that the automated system lacked.

%======================================================================
\section{Lessons for AI on Specialized Algebraic Topology}
%======================================================================

\subsection{Failure mode: plausible-but-wrong definitions}

The \texttt{af} definition of $\mathcal{F}_{\mathcal{O}}$ is not random garbage---it is a natural-sounding definition that someone unfamiliar with the field might propose.  The failure mode is precisely that it sounds plausible while being mathematically vacuous.  This is the most dangerous type of error for automated theorem proving: the definition compiles, the proof tree can be built on top of it, and only careful mathematical analysis reveals the collapse.

\subsection{Adversarial verification as a partial remedy}

The verification wave successfully detected the collapse within the first pass, demonstrating the value of the adversarial framework.  However, the system was unable to \emph{act} on this detection---the 45 challenges remained open, and no repair was attempted.  This suggests that adversarial verification is necessary but not sufficient: the system also needs the ability to perform fundamental reconceptualizations.

\subsection{The ``last mile'' problem in specialized mathematics}

The authors' commentary in~\cite{FirstProof2026} notes that even the best LLM solutions (by Gemini and ChatGPT~5.2~Pro, working outside the \texttt{af} framework) contained ``essentially correct'' statements of the definition and theorem but had ``sketched or slightly garbled'' details.  This suggests a hierarchy of difficulty:
\begin{enumerate}
\item \textbf{Stating the right definition:} Achievable by frontier LLMs with appropriate prompting.
\item \textbf{Outlining the proof strategy:} Achievable by frontier LLMs (following Hill--Yarnall).
\item \textbf{Getting the technical details right:} Currently beyond automated systems, even with adversarial verification.
\item \textbf{Recovering from a wrong foundation:} Fundamentally beyond the current \texttt{af} architecture.
\end{enumerate}

The \texttt{af} attempt failed at level~1 (wrong definition) and could not recover to level~2, while frontier LLMs working in a single pass achieved levels~1--2 but failed at level~3.

\subsection{Recommendations}

For future attempts at similar problems:
\begin{itemize}
\item \textbf{Definition validation before tree construction.}  The \texttt{af} should validate foundational definitions against boundary cases (e.g., complete transfer system, trivial transfer system, $G = C_p$) before building the proof tree.  Had this been done, the collapse of $\mathcal{F}_{\mathcal{O}}$ would have been detected immediately.

\item \textbf{Conceptual checkpoints.}  Before proceeding past the definition phase, the system should verify that its construction genuinely depends on the parameter it claims to generalize (here, $\mathcal{O}$).  A definition-phase sanity check ``does the construction change when $\mathcal{O}$ changes?'' would have caught the error.

\item \textbf{Expert-in-the-loop for specialized fields.}  Problems in equivariant homotopy theory, with its dense conceptual landscape, may require human expert guidance at the definition stage.  The verification agents correctly diagnosed the failure but the system lacked the domain knowledge to propose the correct fix (indexing systems).

\item \textbf{Deeper engagement with references.}  The \texttt{af} registered six external references including Blumberg--Hill and Rubin, which contain the correct framework.  More careful extraction of definitions and constructions from these references might have prevented the foundational error.
\end{itemize}

%======================================================================
\section{Conclusion}
%======================================================================

The comparison between the official solution and the \texttt{af} attempt on Problem~5 reveals a stark gap between automated and expert mathematical reasoning in specialized algebraic topology.

The official solution, by Blumberg, Hill, and Lawson, uses the elegant framework of indexed slice categories: the generators of $\tau_{\geq n}^{\mathcal{O}}$ are norm spheres $N^T S^1$ indexed by admissible sets $T \in \mathcal{O}(H)$, and the connectivity characterization involves the characteristic function $\chi^{\mathcal{O}}(H) = \min\{K : K \to H\}$.  The proof closely follows Hill--Yarnall, adapting isotropy separation and downward induction to the indexed setting.

The \texttt{af} attempt used a fundamentally wrong definition---extracting a ``family'' $\mathcal{F}_{\mathcal{O}}$ from the transfer system that collapses to $\mathrm{Sub}(G)$ for every~$\mathcal{O}$.  This rendered the entire generalization vacuous.  The adversarial verification wave correctly identified the collapse and many other errors (wrong geometric fixed-point formulas, localizing/colocalizing confusion, missing proof content), but the system was unable to repair the foundational error.

The core lesson is that in fields with dense conceptual structures, choosing the right \emph{level of abstraction}---here, indexing systems rather than families---is the critical creative step, and current automated systems are not reliably capable of making this choice.

\begin{thebibliography}{99}

\bibitem{FirstProof2026}
M.~Abouzaid, A.~Blumberg, M.~Hairer, J.~Kileel, T.~Kolda, J.~Nelson, A.~Spielman, N.~Srivastava, C.~Ward, L.~Weinberger, and S.~Williams,
\emph{First Proof: Solutions and Comments}, February 2026.

\bibitem{BBR2021}
S.~Balchin, D.~Barnes, and C.~Roitzheim,
$N_\infty$-operads and associahedra,
\emph{Pacific J.\ Math.} \textbf{315}(2):285--304, 2021.

\bibitem{BlumbergHill2015}
A.~J.~Blumberg and M.~A.~Hill,
Operadic multiplications in equivariant spectra, norms, and transfers,
\emph{Adv.\ Math.} \textbf{285}:658--708, 2015.

\bibitem{Hill2012}
M.~A.~Hill,
The equivariant slice filtration: a primer,
\emph{Homology Homotopy Appl.} \textbf{14}(2):143--166, 2012.

\bibitem{HHR2016}
M.~A.~Hill, M.~J.~Hopkins, and D.~C.~Ravenel,
On the nonexistence of elements of Kervaire invariant one,
\emph{Ann.\ of Math.} (2) \textbf{184}(1):1--262, 2016.

\bibitem{HillYarnall2018}
M.~A.~Hill and C.~Yarnall,
A new formulation of the equivariant slice filtration with applications to $C_p$-slices,
\emph{Proc.\ Amer.\ Math.\ Soc.} \textbf{146}(8):3605--3614, 2018.

\bibitem{Rubin2021}
J.~Rubin,
Detecting Steiner and linear isometries operads,
\emph{Glasg.\ Math.\ J.} \textbf{63}(2):307--342, 2021.

\bibitem{Wilson2017}
D.~Wilson,
On categories of slices,
arXiv:1711.03472, 2017.

\end{thebibliography}

\end{document}
