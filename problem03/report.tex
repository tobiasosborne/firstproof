\documentclass[11pt,a4paper]{article}

% --- Packages ---
\usepackage[utf8]{inputenc}
\usepackage[T1]{fontenc}
\usepackage{lmodern}
\usepackage[margin=2.5cm]{geometry}
\usepackage{amsmath,amssymb,amsthm}
\usepackage{mathtools}
\usepackage{enumitem}
\usepackage{booktabs}
\usepackage{array}
\usepackage{longtable}
\usepackage{xcolor}
\usepackage{hyperref}
\usepackage{tikz}
\usetikzlibrary{trees,arrows.meta,positioning}

% --- Theorem environments ---
\newtheorem{theorem}{Theorem}[section]
\newtheorem{lemma}[theorem]{Lemma}
\newtheorem{proposition}[theorem]{Proposition}
\newtheorem{corollary}[theorem]{Corollary}
\newtheorem{conjecture}[theorem]{Conjecture}
\theoremstyle{definition}
\newtheorem{definition}[theorem]{Definition}
\newtheorem{remark}[theorem]{Remark}

% --- Macros ---
\newcommand{\Sn}{S_n(\lambda)}
\newcommand{\PushTASEP}{$t$-PushTASEP}
\DeclareMathOperator{\Ind}{Ind}

% --- Colors for status ---
\definecolor{proved}{RGB}{0,128,0}
\definecolor{pending}{RGB}{200,150,0}
\definecolor{refuted}{RGB}{200,0,0}
\definecolor{archived}{RGB}{128,128,128}
\definecolor{critical}{RGB}{180,0,0}
\definecolor{resolved}{RGB}{0,100,180}

% --- Title ---
\title{\textbf{Report on Problem~3: Markov Chain with\\ASEP Polynomial Stationary Distribution}\\[6pt]
\large Adversarial Proof Framework Analysis}
\author{Generated from the \texttt{af} proof workspace\\
First Proof Project}
\date{February 2026}

\begin{document}
\maketitle

\begin{abstract}
This report documents the adversarial proof investigation of Problem~3 from the First Proof paper (posed by Lauren Williams): whether there exists a \emph{nontrivial} Markov chain on the set of compositions $\Sn$ of a restricted partition~$\lambda$ whose stationary distribution is given by the ratio of interpolation ASEP polynomials $F^*_\mu / P^*_\lambda$ at $q = 1$.
The answer is \textbf{YES}: the inhomogeneous multispecies \PushTASEP\ provides such a chain.
Over three adversarial sessions, we have constructed a 9-node proof tree.
All 97 challenges raised in Session~2 were resolved in Session~3 via complete rewrites of all 8 leaf nodes.
Currently, 0 nodes are validated, 9 are pending, and 0 challenges remain open.
The proof is ready for a second verification wave.
The main mathematical risk resides in Node~1.6 (the ratio identity), whose ``Hecke stationarity transfer'' argument is a plausible but nontrivial original claim; all other nodes are built on standard machinery and published results.
\end{abstract}

\tableofcontents
\newpage

%======================================================================
\section{Problem Statement}
\label{sec:problem}
%======================================================================

\subsection{Setup}

The problem, posed by Lauren Williams (Harvard University), lies in algebraic combinatorics at the interface of Macdonald polynomial theory and interacting particle systems.

\begin{definition}[Restricted partition]
Let $\lambda = (\lambda_1 > \lambda_2 > \cdots > \lambda_n \ge 0)$ be a partition with distinct parts.  We say $\lambda$ is \emph{restricted} if it has a unique part of size~$0$ and no part of size~$1$.
\end{definition}

\begin{definition}[State space and polynomials]
Let $\Sn$ denote the set of all compositions $\mu = (\mu_1, \ldots, \mu_n)$ that are permutations of the parts of~$\lambda$.
For $\mu \in \Sn$, define:
\begin{itemize}
\item $F^*_\mu(x_1, \ldots, x_n;\, q, t)$: the \emph{interpolation ASEP polynomial} (BDW25 Definition~1.2/1.14);
\item $P^*_\lambda(x_1, \ldots, x_n;\, q, t)$: the \emph{interpolation Macdonald polynomial} (BDW25 Proposition~2.15).
\end{itemize}
\end{definition}

\subsection{The Question}

\begin{conjecture}[Williams]
\label{conj:main}
Does there exist a \emph{nontrivial} Markov chain on $\Sn$ whose stationary distribution is
\[
  \pi(\mu) \;=\; \frac{F^*_\mu(x_1, \ldots, x_n;\, q{=}1,\, t)}{P^*_\lambda(x_1, \ldots, x_n;\, q{=}1,\, t)}
  \qquad \text{for } \mu \in \Sn,
\]
where ``nontrivial'' means the transition probabilities are \emph{not} described using the polynomials $F^*_\mu$?
\end{conjecture}

\noindent\textbf{Answer: YES.}  The inhomogeneous multispecies \PushTASEP\ on a ring with $n$ sites, site parameters $x_1, \ldots, x_n > 0$, and pushing parameter $t \in (0,1)$ provides such a chain.

\subsection{Why This Is Interesting}

Several features make this problem non-trivial:
\begin{enumerate}
\item The interpolation ASEP polynomials $F^*_\mu$ are \emph{inhomogeneous} polynomials containing lower-degree correction terms; they are \emph{not} the same as the homogeneous ASEP polynomials $f_\mu$ whose stationarity is established by Ayyer--Martin--Williams (AMW24).
\item The bridge from the homogeneous ratio $f_\mu / P_\lambda$ (known to be a stationary distribution) to the interpolation ratio $F^*_\mu / P^*_\lambda$ requires a nontrivial \emph{ratio identity}, which is the crux of the proof.
\item The ``nontriviality'' condition asks that the chain is not merely the trivial i.i.d.\ sampler that draws each state from $\pi$ independently of the current state.
\item The \PushTASEP\ is a long-range interacting particle system with geometric displacement cascades, not a nearest-neighbor chain, making its analysis intrinsically more complex than classical TASEP.
\end{enumerate}


%======================================================================
\section{Proof Strategy}
\label{sec:strategy}
%======================================================================

The proof decomposes into eight nodes (plus a root), organized in a linear dependency chain:
\[
  \underbrace{1.1}_{\text{state space}} \;\to\; \underbrace{1.2}_{\text{notation}} \;\to\; \underbrace{1.4}_{\text{chain def}} \;\to\; \underbrace{1.5}_{\text{stationarity}} \;\to\; \underbrace{1.6}_{\text{ratio id}} \;\to\; \underbrace{1.3}_{\text{positivity}} \;\to\; \underbrace{1.7}_{\text{nontriviality}} \;\to\; \underbrace{1.8}_{\text{QED}}
\]

\subsection{Node~1.1 --- State Space Setup}

Defines $\Sn$ as the set of all $n!$ permutations of $\lambda$, identified with particle configurations on the ring $\mathbb{Z}_n = \{1, \ldots, n\}$ with periodic boundary.  Species labels are $\{0\} \cup \{k \ge 2 : k \in \lambda\}$, with species~$0$ representing the unique vacancy.

\subsection{Node~1.2 --- Polynomial Identification and Notation}

Establishes a comprehensive cross-paper notation table (BDW25/AMW24/CMW22).  Key identifications:
\begin{itemize}
\item $f^*_\mu := T_{\sigma_\mu} E^*_\lambda$ (BDW25 Definition~1.2, algebraic via Hecke operators);
\item $F^*_\mu$ (BDW25 Definition~1.14, combinatorial via signed multiline queues);
\item $f^*_\mu = F^*_\mu$ (BDW25 Theorem~1.15);
\item $f_\mu$ = top-degree homogeneous component of $f^*_\mu$ (BDW25 Theorem~2.3);
\item $P^*_\lambda = \sum_{\mu \in \Sn} f^*_\mu$ (BDW25 Proposition~2.15);
\item At $q = 1$: the $f_\mu(x; 1, t) / P_\lambda(x; 1, t)$ are stationary weights of the \PushTASEP\ (AMW24 Theorem~1.1).
\end{itemize}

\subsection{Node~1.4 --- Markov Chain Construction}

Defines the inhomogeneous multispecies \PushTASEP\ following AMW24 (arXiv:2403.10485) Definition~2.1:
\begin{enumerate}
\item Each site $i$ has an independent exponential clock with rate $1/x_i$.
\item When site $i$'s clock rings, the particle there (if not a vacancy) becomes \emph{activated} and scans clockwise for weaker species.
\item With probability $(1-t)$, it displaces the first weaker species encountered; with probability $t$, it continues scanning (geometric displacement with parameter $t$).
\item Displaced particles cascade: each displaced particle repeats the scan-and-displace process.
\item The cascade terminates when a vacancy is displaced.
\end{enumerate}

Irreducibility for $t \in [0, 1)$ follows from AMW24 Proposition~2.4: the $t = 0$ PushTASEP is irreducible, and every $t = 0$ transition occurs with probability $\ge (1-t) > 0$ for $t \in (0,1)$.

A discrete-time chain $P = I + Q/R$ is obtained by uniformization at rate $R = \sum_i 1/x_i$.

\subsection{Node~1.5 --- Stationarity (AMW24 Theorem~1.1)}

Cites the main external theorem: the \PushTASEP\ has stationary distribution
\[
  \pi_\lambda(\mu) = \frac{f_\mu(x_1, \ldots, x_n;\, 1, t)}{P_\lambda(x_1, \ldots, x_n;\, 1, t)}.
\]
The AMW24 proof is \emph{algebraic/combinatorial} (via multiline diagrams and $q = 1$ properties of nonsymmetric Macdonald polynomials), not via a Ferrari--Martin multiline Markov process.

\subsection{Node~1.6 --- Ratio Identity (CRUX)}
\label{sec:crux}

\begin{proposition}[Ratio identity]
For $t \in (0,1)$ and $x_i > 0$:
\[
  \frac{f^*_\mu(x;\, 1, t)}{P^*_\lambda(x;\, 1, t)} = \frac{f_\mu(x;\, 1, t)}{P_\lambda(x;\, 1, t)}
  \qquad \forall\, \mu \in \Sn.
\]
\end{proposition}

The proof proceeds in three steps:
\begin{enumerate}
\item \textbf{Step 1 (Hecke relations):} Both $f^*_\mu$ and $f_\mu$ satisfy the same Hecke relations (BDW25 Proposition~2.10):
  \begin{itemize}
  \item $T_i f^*_\mu = f^*_{s_i\mu}$ when $\mu_i > \mu_{i+1}$;
  \item $T_i f^*_\mu = t \cdot f^*_\mu$ when $\mu_i = \mu_{i+1}$;
  \item $T_i f^*_\mu = (t-1) f^*_\mu + t\, f^*_{s_i\mu}$ when $\mu_i < \mu_{i+1}$.
  \end{itemize}
\item \textbf{Step 2 (Balance transfer):} The AMW24 proof that $(f_\mu)_\mu$ lies in the left null space of the \PushTASEP\ generator $Q$ uses \emph{only} relations (a)--(c) above.  Since $(f^*_\mu)_\mu$ satisfies the same relations, the same algebraic steps show $(f^*_\mu)_\mu$ also lies in the left null space of $Q$.
\item \textbf{Step 3 (Perron--Frobenius uniqueness):} The \PushTASEP\ is irreducible (Node~1.4), so $\ker_L(Q)$ is one-dimensional.  Both vectors are proportional: $f^*_\mu = C(x,t) \cdot f_\mu$ for all $\mu$, with $C$ independent of $\mu$.  Dividing by the respective sums gives the ratio identity.
\end{enumerate}

\textbf{Critical caveat:} Step~2 is the deepest mathematical claim.  BDW25 Remark~1.17 defers the interpolation probabilistic interpretation to a forthcoming paper [BDW], suggesting the experts consider this transfer nontrivial.  BDW25 Theorem~7.1 (factorization) provides an alternative algebraic route.

\subsection{Node~1.3 --- Positivity and Normalization (Corollary)}

Structured as a \emph{corollary} of Nodes 1.5 and 1.6 (breaking the circularity that plagued Session~2):
\begin{enumerate}
\item From Node~1.5: $f_\mu(x; 1, t) / P_\lambda(x; 1, t) \ge 0$ with full support and $P_\lambda > 0$.
\item From Node~1.6: $f^*_\mu / P^*_\lambda = f_\mu / P_\lambda$.
\item Therefore $\pi(\mu) = f^*_\mu / P^*_\lambda$ is a probability distribution with full support and $P^*_\lambda > 0$.
\end{enumerate}

\subsection{Node~1.7 --- Nontriviality}

Proves the \PushTASEP\ is nontrivial via a \emph{sparsity argument}.  The trivial i.i.d.\ sampler has $P^{\mathrm{triv}}(\mu, \nu) = \pi(\nu) > 0$ for all $\mu, \nu$.  The \PushTASEP\ transition matrix $P^{tP}$ has zero entries: for $n = 3$, $\lambda = (3, 2, 0)$, the configuration $(2, 0, 3)$ is unreachable from $(3, 2, 0)$ in a single step (verified by exhaustive cascade analysis), giving $P^{tP}((3,2,0),(2,0,3)) = 0 \ne \pi((2,0,3)) > 0$.

\subsection{Node~1.8 --- Conclusion}

Synthesizes Nodes~1.1--1.7: the \PushTASEP\ is an irreducible Markov chain on $\Sn$ with stationary distribution $f^*_\mu / P^*_\lambda = f_\mu / P_\lambda$, and it is nontrivial.


%======================================================================
\section{Session History}
\label{sec:sessions}
%======================================================================

\subsection{Session 1: Proof Tree Registration}

\begin{itemize}
\item Initialized the \texttt{af} workspace with the root conjecture (Node~1).
\item Created 8 child nodes (1.1--1.8) covering: state space setup, polynomial identification, positivity/normalization, chain construction, stationarity, ratio identity, nontriviality, and conclusion.
\item Registered external references (AMW24, BDW25, CMW22, FM07, KS96) and definitions.
\item \textbf{No verification attempted.}
\end{itemize}

\subsection{Session 2: First Verification Wave}

\begin{itemize}
\item \textbf{All 8 leaf nodes verified by adversarial verifiers.}
\item \textbf{97 challenges raised} across all 8 nodes, broken down as follows:
\end{itemize}

\begin{center}
\begin{tabular}{@{}lccccc@{}}
\toprule
\textbf{Node} & \textbf{Critical} & \textbf{Major} & \textbf{Minor} & \textbf{Total} \\
\midrule
1.1 (state space) & 0 & 4 & 3 & 7 \\
1.2 (notation) & 2 & 8 & 3 & 13 \\
1.3 (positivity) & 4 & 6 & 1 & 11 \\
1.4 (chain construction) & 4 & 10 & 3 & 17 \\
1.5 (stationarity) & 7 & 10 & 1 & 18 \\
1.6 (ratio identity) & 11 & 5 & 0 & 16 \\
1.7 (nontriviality) & 7 & 5 & 1 & 13 \\
1.8 (conclusion) & 1 & 1 & 0 & 2 \\
\midrule
\textbf{Total} & \textbf{36} & \textbf{49} & \textbf{12} & \textbf{97} \\
\bottomrule
\end{tabular}
\end{center}

\noindent The challenges revealed five systemic problems:
\begin{enumerate}
\item \textbf{Wrong AMW24 arXiv reference} --- all nodes cited a nonexistent arXiv number instead of the correct 2403.10485.
\item \textbf{Node~1.5 fictional proof method} --- incorrectly attributed the stationarity proof to a Ferrari--Martin multiline Markov process construction, which does not exist for the \PushTASEP\ at $t > 0$.
\item \textbf{Node~1.6 logical fallacy} --- the original argument claimed ``both ratios sum to~1, therefore they are equal,'' which is false.
\item \textbf{Circular dependency} --- Node~1.3 (positivity) attempted to prove $F^*_\mu \ge 0$ standalone, creating a circular dependency with the ratio identity.
\item \textbf{Notation inconsistency} --- inconsistent use of $F^*_\mu$ vs.\ $f^*_\mu$ vs.\ $F_\eta$ across nodes.
\end{enumerate}

\subsection{Session 3: Prover Wave (Complete Rewrites)}

\begin{itemize}
\item \textbf{All 8 leaf nodes rewritten by independent prover subagents.}
\item \textbf{All 97 challenges resolved.}
\item \textbf{0 open challenges remain.}
\end{itemize}

Key changes per node:

\begin{center}
\begin{tabular}{@{}lp{9cm}@{}}
\toprule
\textbf{Node} & \textbf{Key Changes in Session~3 Rewrite} \\
\midrule
1.1 & Precise definitions: ring $\mathbb{Z}_n$, species labels $\{0\} \cup \{k \ge 2\}$, $|\Sn| = n!$ \\
1.2 & Full notation table (BDW25/AMW24/CMW22), correct arXiv, six numbered claims \\
1.3 & \textbf{Restructured as corollary} of 1.5 + 1.6; circularity broken \\
1.4 & Full cascade mechanism, geometric displacement, vacancy behavior, irreducibility, CT$\to$DT bridge \\
1.5 & \textbf{Complete rewrite}: cites AMW24 Thm~1.1 correctly; algebraic proof method; no Ferrari--Martin fiction \\
1.6 & \textbf{Complete rewrite}: 3-step Hecke stationarity argument (BDW25 Prop~2.10 + AMW24 Thm~1.1 + Perron--Frobenius) \\
1.7 & Formal definition of ``nontrivial,'' rigorous sparsity proof with $n=3$ computation \\
1.8 & Full dependency chain, logical synthesis, parameter domains \\
\bottomrule
\end{tabular}
\end{center}


%======================================================================
\section{Current Status}
\label{sec:status}
%======================================================================

\subsection{Node Statistics}

\begin{center}
\begin{tabular}{@{}lcc@{}}
\toprule
\textbf{Epistemic State} & \textbf{Count} & \textbf{Meaning} \\
\midrule
\textcolor{pending}{Pending} & 9 & Awaiting verification \\
\textcolor{proved}{Validated} & 0 & --- \\
\textcolor{refuted}{Refuted} & 0 & --- \\
\textcolor{archived}{Archived} & 0 & --- \\
\midrule
\textbf{Total} & \textbf{9} & \\
\bottomrule
\end{tabular}
\end{center}

\subsection{Challenge Statistics}

\begin{center}
\begin{tabular}{@{}lcc@{}}
\toprule
\textbf{Metric} & \textbf{Count} \\
\midrule
Total challenges filed (Session~2) & 97 \\
Challenges resolved (Session~3) & 97 \\
\textbf{Open challenges} & \textbf{0} \\
\bottomrule
\end{tabular}
\end{center}

\subsection{Risk Assessment by Node}

\begin{center}
\begin{tabular}{@{}llcl@{}}
\toprule
\textbf{Node} & \textbf{Description} & \textbf{Risk} & \textbf{Rationale} \\
\midrule
1.1 & State space setup & \textcolor{proved}{Low} & Standard definitions \\
1.2 & Polynomial identification & \textcolor{proved}{Low} & Notation; cites published results \\
1.3 & Positivity (corollary) & \textcolor{proved}{Low} & Follows from 1.5 + 1.6 \\
1.4 & Chain construction & \textcolor{pending}{Medium} & Cascade details may need verification \\
1.5 & Stationarity (AMW24) & \textcolor{pending}{Medium} & External theorem citation \\
1.6 & Ratio identity (CRUX) & \textcolor{refuted}{High} & Step~2 (Hecke transfer) is nontrivial \\
1.7 & Nontriviality & \textcolor{proved}{Low} & Concrete sparsity computation \\
1.8 & Conclusion & \textcolor{proved}{Low} & Synthesis node \\
\bottomrule
\end{tabular}
\end{center}


%======================================================================
\section{Assessment of Correctness}
\label{sec:assessment}
%======================================================================

\subsection{What Is Secure}

\begin{itemize}
\item \textbf{Node~1.1 (state space):} Standard combinatorial setup.  The definition of $\Sn$ as the set of $n!$ permutations of a restricted partition is unambiguous.
\item \textbf{Node~1.2 (notation):} Cross-paper notation table is now comprehensive and internally consistent.  All six claims are direct citations from BDW25/AMW24/CMW22.
\item \textbf{Node~1.4 (chain construction):} The \PushTASEP\ dynamics follow AMW24 Definition~2.1 verbatim.  Irreducibility via the $t = 0$ reduction is standard.  The CT$\to$DT uniformization bridge is textbook.
\item \textbf{Node~1.5 (stationarity):} AMW24 Theorem~1.1 is a published result with a complete proof.  The node correctly attributes the algebraic/combinatorial proof method.
\item \textbf{Node~1.7 (nontriviality):} The explicit $n = 3$, $\lambda = (3,2,0)$ computation is verifiable by hand.  The sparsity argument is watertight.
\item \textbf{Node~1.3 (positivity):} As a corollary of 1.5 + 1.6, the logic is clean: once the ratio identity is established, positivity and normalization follow immediately.
\end{itemize}

\subsection{What Requires Scrutiny}

\begin{itemize}
\item \textbf{Node~1.6, Step~2 (Hecke stationarity transfer):} This is the \emph{deepest mathematical claim} in the proof.  It asserts that the AMW24 balance equation proof for $f_\mu$ transfers to $f^*_\mu$ because both satisfy the same Hecke relations (BDW25 Proposition~2.10).  A fully rigorous verification requires confirming that the AMW24 multiline diagram argument uses \emph{only} relations (a)--(c) and no additional properties specific to homogeneous polynomials.

BDW25 Remark~1.17 states that an interpolation analogue of the AMW24 probabilistic interpretation will appear in a forthcoming paper [BDW], indicating the experts consider this transfer worthy of dedicated treatment.

\item \textbf{Node~1.4 (cascade mechanism details):} While the overall dynamics follow AMW24, the detailed cascade termination and state-space preservation arguments should be checked against the formal definition.
\end{itemize}

\subsection{Overall Confidence}

\begin{center}
\begin{tabular}{@{}lp{9cm}@{}}
\toprule
\textbf{Component} & \textbf{Assessment} \\
\midrule
Answer (YES) & \textbf{Very high.}  The \PushTASEP\ is the natural candidate; AMW24 establishes stationarity of the homogeneous ratios. \\[4pt]
Foundations (1.1, 1.2, 1.4) & \textbf{High.}  Standard definitions and published results. \\[4pt]
Stationarity (1.5) & \textbf{High.}  Direct citation of AMW24 Theorem~1.1. \\[4pt]
Ratio identity (1.6) & \textbf{Medium--High.}  The Hecke stationarity argument is plausible and uses only published ingredients; the transfer step (Step~2) is the sole substantive risk.  BDW25 Theorem~7.1 provides an alternative route. \\[4pt]
Nontriviality (1.7) & \textbf{High.}  Explicit computation. \\[4pt]
Overall proof & \textbf{Medium--High.}  The proof is complete modulo the verification of Step~2 in Node~1.6. \\
\bottomrule
\end{tabular}
\end{center}


%======================================================================
\section{Prospects for a Successful Proof}
\label{sec:prospects}
%======================================================================

The proof is in strong shape: all 97 Session~2 challenges have been resolved, the logical structure is clean (no circular dependencies), and the only substantive risk is concentrated in a single step (Node~1.6, Step~2).  We assess the prospects as \textbf{good}.

\subsection{Why the Proof Is Likely to Succeed}

\begin{enumerate}
\item \textbf{The answer is almost certainly correct.}  AMW24 establishes stationarity for the homogeneous ASEP polynomials; the ratio identity $f^*_\mu / P^*_\lambda = f_\mu / P_\lambda$ is a natural expectation given the Hecke-algebraic relationship between interpolation and homogeneous polynomials.

\item \textbf{The Hecke stationarity argument (Node~1.6) uses only published ingredients.}  BDW25 Proposition~2.10 (Hecke relations for $f^*_\mu$), AMW24 Theorem~1.1 (stationarity of $f_\mu$), and the Perron--Frobenius theorem are all established results.  The novelty is \emph{only} in the observation that the AMW24 proof uses only the Hecke relations.

\item \textbf{An alternative algebraic route exists.}  BDW25 Theorem~7.1 (factorization of interpolation Macdonald polynomials) provides a complementary approach that may yield $f^*_\mu = C \cdot f_\mu$ directly, bypassing the stationarity argument entirely.

\item \textbf{The forthcoming [BDW] paper.}  BDW25 Remark~1.17 promises a probabilistic interpretation of the interpolation polynomials, which would likely establish the ratio identity as a corollary.
\end{enumerate}

\subsection{Strategies for Completing the Proof}

\subsubsection{Strategy A: Verify the Hecke Transfer (Node~1.6, Step~2)}

The most direct path: carefully verify that the AMW24 proof (Sections~3--6) uses \emph{only} the Hecke relations (a)--(c) of BDW25 Proposition~2.10 when establishing the balance equation $\sum_\mu f_\mu Q_{\mu,\nu} = 0$.  This requires a line-by-line audit of the AMW24 proof, checking that no homogeneity, positivity, or evaluation-at-special-points arguments are invoked.

\textbf{Difficulty:} Medium.  The AMW24 proof is algebraic and structured around Hecke operators, so this verification is feasible.

\subsubsection{Strategy B: BDW25 Factorization Route}

Use BDW25 Theorem~7.1 (factorization of $P^*_\lambda$ in terms of interpolation polynomials at different partitions) to establish the proportionality $f^*_\mu = C(x,t) \cdot f_\mu$ directly.  The factorization theorem relates $P^*_\lambda$ to products of interpolation Macdonald polynomials of smaller rank, and the ASEP polynomial decomposition $P^*_\lambda = \sum f^*_\mu$ (BDW25 Proposition~2.15) combined with $P_\lambda = \sum f_\mu$ may yield the proportionality.

\textbf{Difficulty:} Medium--Hard.  Requires working with the factorization machinery of BDW25 Section~7.

\subsubsection{Strategy C: Await [BDW] Paper}

BDW25 Remark~1.17 announces a forthcoming paper by Ben Dali and Williams establishing the probabilistic interpretation of interpolation ASEP polynomials.  If published, this would likely resolve Node~1.6 directly.

\textbf{Difficulty:} None (but timeline uncertain).

\subsubsection{Strategy D: Direct $q = 1$ Specialization}

At $q = 1$, the interpolation Macdonald polynomials $E^*_\mu$ develop special properties (AMW24 Proposition~4.1, Corollary~4.6) that simplify the Hecke algebra action.  One could attempt to show directly that at $q = 1$, the lower-degree terms of $f^*_\mu$ are proportional to those of $f_\mu$, establishing $f^*_\mu = C \cdot f_\mu$ without the stationarity detour.

\textbf{Difficulty:} Medium.

\subsection{Assessment of Strategy Viability}

\begin{center}
\begin{tabular}{@{}lp{5.5cm}cc@{}}
\toprule
\textbf{Strategy} & \textbf{What It Resolves} & \textbf{Difficulty} & \textbf{Viability} \\
\midrule
A (Hecke transfer) & Node~1.6 Step~2 directly & Medium & High \\
B (BDW25 factorization) & Node~1.6 via alternative route & Medium--Hard & Medium \\
C (Await [BDW]) & Node~1.6 via external result & None & Uncertain \\
D ($q=1$ specialization) & Node~1.6 via direct argument & Medium & Medium--High \\
\bottomrule
\end{tabular}
\end{center}

\noindent Strategy~A is the recommended primary approach, with Strategy~D as a backup.

%======================================================================
\section{Recommended Next Steps}
\label{sec:next}
%======================================================================

\begin{enumerate}
\item \textbf{Priority 1: Verification wave~2.}  Run adversarial verifiers on all 9 nodes (all are pending with 0 open challenges).  Priority order:
  \begin{enumerate}
  \item Node~1.6 (highest risk --- scrutinize the Hecke stationarity transfer, Step~2);
  \item Node~1.4 (verify cascade mechanism matches AMW24 precisely);
  \item Node~1.5 (verify AMW24 Theorem~1.1 is cited faithfully);
  \item Nodes~1.1, 1.2, 1.3, 1.7, 1.8 (should be quick to verify).
  \end{enumerate}

\item \textbf{Priority 2: Address any new challenges from verification wave~2.}  Based on Session~2 experience, expect Node~1.6 to receive the most scrutiny.  Have the Hecke transfer verification (Strategy~A) ready as a repair.

\item \textbf{Priority 3: Validate leaf nodes and propagate upward.}  Once leaf nodes pass verification, validate them and work upward to the root.

\item \textbf{Priority 4: If Node~1.6 Step~2 is challenged again}, implement Strategy~D ($q = 1$ specialization) as an alternative route to the ratio identity.
\end{enumerate}


%======================================================================
\section{Key References}
\label{sec:refs}
%======================================================================

\begin{thebibliography}{99}

\bibitem{AMW24}
A.~Ayyer, J.~B.~Martin, and L.~K.~Williams,
\emph{The inhomogeneous multispecies PushTASEP and the $t$-PushTASEP},
arXiv:2403.10485 (2024).
Theorem~1.1 (stationarity), Proposition~2.4 (irreducibility), Definition~2.1 (dynamics).

\bibitem{BDW25}
L.~Ben Dali and L.~K.~Williams,
\emph{Interpolation ASEP polynomials},
arXiv:2510.02587 (2025).
Definition~1.2 ($f^*_\mu$), Definition~1.14 ($F^*_\mu$), Theorem~1.15 ($f^*_\mu = F^*_\mu$), Proposition~2.10 (Hecke relations), Proposition~2.15 ($P^*_\lambda$), Theorem~2.3 (top-degree), Theorem~7.1 (factorization), Remark~1.17 (forthcoming [BDW]).

\bibitem{CMW22}
S.~Corteel, O.~Mandelshtam, and L.~K.~Williams,
\emph{From multiline queues to Macdonald polynomials via the exclusion process},
arXiv:1811.01024 (2018).
ASEP polynomials via multiline queues.

\bibitem{AM23}
A.~Ayyer and J.~B.~Martin,
\emph{The multispecies PushTASEP on a ring},
arXiv:2310.09740 (2023).
PushTASEP at $t = 0$; irreducibility.

\bibitem{FM07}
P.~A.~Ferrari and J.~B.~Martin,
\emph{Stationary distributions of multi-type totally asymmetric exclusion processes},
Ann.\ Probab.\ \textbf{35}(3) (2007), 807--832.
\textbf{Ordinary TASEP only} --- does NOT apply to the \PushTASEP.

\bibitem{KS96}
F.~Knop (1997) and S.~Sahi (1996),
Interpolation Macdonald polynomials.

\end{thebibliography}


\newpage
%======================================================================
\appendix
\section{Full Proof Tree}
\label{app:tree}
%======================================================================

The complete proof tree as exported from the adversarial proof framework (\texttt{af status}).
All 9 nodes are \textcolor{pending}{\textbf{pending}}.
All 97 challenges from Session~2 have been \textcolor{resolved}{\textbf{resolved}}.

{\small\begin{verbatim}
1 [pending] Root: For a restricted partition lambda with distinct parts,
  |  there exists a nontrivial Markov chain on S_n(lambda) whose
  |  stationary distribution is F*_mu(x;q=1,t) / P*_lambda(x;q=1,t),
  |  where transition probabilities are not described using F*_mu.
  |  Challenges: 0 open (0 total on root).
  |
  +-- 1.1 [pending] STATE SPACE SETUP
  |   Defines S_n(lambda) as n! permutations of lambda on ring Z_n.
  |   Species labels: {0} ∪ {k >= 2 : k in lambda}. One vacancy.
  |   Inference: by_definition.
  |   Challenges: 7 filed, 7 resolved, 0 open.
  |
  +-- 1.2 [pending] POLYNOMIAL IDENTIFICATION AND NOTATION
  |   Cross-paper notation table (BDW25/AMW24/CMW22).
  |   Key: f*_mu = F*_mu (BDW25 Thm 1.15), P*_lambda = sum f*_mu.
  |   At q=1: f_mu/P_lambda = stationary weights (AMW24 Thm 1.1).
  |   Depends on: 1.1.
  |   Inference: by_definition.
  |   Challenges: 13 filed, 13 resolved, 0 open.
  |
  +-- 1.3 [pending] POSITIVITY AND NORMALIZATION (COROLLARY)
  |   pi(mu) = f*_mu/P*_lambda = f_mu/P_lambda >= 0, sums to 1.
  |   Derives from: Node 1.5 (stationarity) + Node 1.6 (ratio id).
  |   NOT standalone: circularity broken by making this a corollary.
  |   Depends on: 1.5, 1.6.
  |   Inference: modus_ponens.
  |   Challenges: 11 filed, 11 resolved, 0 open.
  |
  +-- 1.4 [pending] MARKOV CHAIN CONSTRUCTION
  |   Defines the inhomogeneous multispecies t-PushTASEP (AMW24 Def 2.1).
  |   Exponential clocks rate 1/x_i, geometric displacement (param t),
  |   cascade mechanism, vacancy termination.
  |   Irreducibility via t=0 reduction (AMW24 Prop 2.4).
  |   CT->DT uniformization: P = I + Q/R.
  |   Depends on: 1.1.
  |   Inference: by_definition.
  |   Challenges: 17 filed, 17 resolved, 0 open.
  |
  +-- 1.5 [pending] STATIONARITY (AMW24 THEOREM 1.1)
  |   The t-PushTASEP has stationary distribution
  |   pi_lambda(mu) = f_mu(x;1,t) / P_lambda(x;1,t).
  |   Proof: algebraic/combinatorial (multiline diagrams + q=1
  |   properties of nonsymmetric Macdonald polynomials).
  |   NOT Ferrari-Martin multiline Markov process.
  |   Depends on: 1.2, 1.4.
  |   Inference: external theorem (AMW24 Thm 1.1).
  |   Challenges: 18 filed, 18 resolved, 0 open.
  |
  +-- 1.6 [pending] RATIO IDENTITY (CRUX)
  |   f*_mu(x;1,t)/P*_lambda(x;1,t) = f_mu(x;1,t)/P_lambda(x;1,t).
  |   Step 1: Both f*_mu and f_mu satisfy Hecke relations (BDW25 2.10).
  |   Step 2: AMW24 balance proof transfers from f_mu to f*_mu.
  |   Step 3: Perron-Frobenius uniqueness => proportionality.
  |   CAVEAT: Step 2 is nontrivial (BDW25 Remark 1.17).
  |   Alternative: BDW25 Thm 7.1 (factorization route).
  |   Depends on: 1.2, 1.4, 1.5.
  |   Inference: deduction (modus ponens).
  |   Challenges: 16 filed, 16 resolved, 0 open.
  |
  +-- 1.7 [pending] NONTRIVIALITY
  |   P^{tP} != P^{triv} (the i.i.d. sampler).
  |   Proof: sparsity. For n=3, lambda=(3,2,0):
  |   P^{tP}((3,2,0),(2,0,3)) = 0 (unreachable in one step),
  |   but P^{triv}((3,2,0),(2,0,3)) = pi((2,0,3)) > 0.
  |   Explicit cascade analysis for all 3 clock events given.
  |   Depends on: 1.1, 1.2, 1.4, 1.5.
  |   Inference: deduction.
  |   Challenges: 13 filed, 13 resolved, 0 open.
  |
  +-- 1.8 [pending] CONCLUSION
      Synthesizes Nodes 1.1-1.7.
      Answer: YES. The t-PushTASEP is the required Markov chain.
      W = W^circ universal; V case-dependent.
      Depends on: 1.1, 1.2, 1.3, 1.4, 1.5, 1.6, 1.7.
      Inference: deduction.
      Challenges: 2 filed, 2 resolved, 0 open.

--- Statistics ---
Nodes: 9 total (9 pending, 0 validated, 0 refuted, 0 archived)
Challenges: 97 total (97 resolved, 0 open)
\end{verbatim}}


\newpage
%======================================================================
\section{Complete Challenge Summary}
\label{app:challenges}
%======================================================================

All 97 challenges from Session~2 have been resolved in Session~3.
The following table summarizes them by node and severity.

{\footnotesize
\begin{longtable}{@{}p{2.0cm}p{1.0cm}p{1.2cm}p{8.5cm}@{}}
\toprule
\textbf{Challenge ID} & \textbf{Node} & \textbf{Severity} & \textbf{Summary (abbreviated)} \\
\midrule
\endhead

\multicolumn{4}{@{}l}{\textbf{Node 1.1 --- State Space Setup (7 challenges, all resolved)}} \\
\midrule
ch-035eb1556b1 & 1.1 & major & Missing specification that every site has exactly one occupant \\
ch-0516d10647b & 1.1 & major & Ring topology not stated to imply periodic boundary \\
ch-9ef4da0119d & 1.1 & major & ``Vacancies and particles'' phrasing imprecise \\
ch-e30fa7e54f4 & 1.1 & major & ``Well-separated'' is mathematically undefined \\
ch-26d7d0aee8e & 1.1 & minor & Does not state $|S_n(\lambda)| = n!$ \\
ch-9f66d8b2bf4 & 1.1 & minor & Inference type should be by\_definition \\
ch-f87bbe81877 & 1.1 & minor & ``$n$ sites of a ring'' phrasing ambiguous \\
\midrule

\multicolumn{4}{@{}l}{\textbf{Node 1.2 --- Polynomial Identification (13 challenges, all resolved)}} \\
\midrule
ch-3281658e3b5 & 1.2 & critical & Wrong arXiv for AMW24 \\
ch-6bf8ddf12b6 & 1.2 & critical & Wrong arXiv for AMW24 (duplicate) \\
ch-331c0d2575b & 1.2 & major & $P_\lambda$ as partition function not established \\
ch-9b47dae9fa2 & 1.2 & major & Interpolation vs.\ homogeneous conflation \\
ch-a7d6a8c27f2 & 1.2 & major & $P_\lambda$ partition function claim \\
ch-bbd61e7e621 & 1.2 & major & Notation mismatch $F^*_\mu$ vs $f^*_\mu$ \\
ch-bd836338c75 & 1.2 & major & Notation confusion AMW24 vs BDW25 \\
ch-d37cd6df09e & 1.2 & major & Notation mismatch in identities \\
ch-daea0680bca & 1.2 & major & AMW24 uses $F_\eta$ (no asterisk) \\
ch-df660cc9766 & 1.2 & major & Conflation interpolation/homogeneous \\
ch-34f72af7e1d & 1.2 & minor & Missing domain of validity \\
ch-9dc89a4a93d & 1.2 & minor & Imprecise $E^*_\lambda$ description \\
ch-f566b5b1fc6 & 1.2 & minor & Missing dependency on Node~1.1 \\
\midrule

\multicolumn{4}{@{}l}{\textbf{Node 1.3 --- Positivity and Normalization (11 challenges, all resolved)}} \\
\midrule
ch-25c6c5abfb4 & 1.3 & critical & Circular positivity argument \\
ch-35912e69e9b & 1.3 & critical & Undeclared circular dependency \\
ch-51f3984ac6d & 1.3 & critical & Circular positivity: $F^*_\mu \ge 0$ standalone \\
ch-8f12f73248b & 1.3 & critical & Missing dependency on Node~1.6 \\
ch-1e86d0c492e & 1.3 & major & Hecke symmetrization identity source \\
ch-27ac074eac1 & 1.3 & major & Division by $P^*_\lambda$ requires $P^*_\lambda > 0$ \\
ch-32e4972f5d3 & 1.3 & major & Conclusion presented as assumption \\
ch-3abeea04155 & 1.3 & major & Hecke symmetrization identity at $q=1$ \\
ch-c45e58fbc9a & 1.3 & major & No proof strategy for nonnegativity \\
ch-d3617c80310 & 1.3 & major & $P^*_\lambda > 0$ not established \\
ch-c7c400820a7 & 1.3 & minor & Edge cases not addressed \\
\midrule

\multicolumn{4}{@{}l}{\textbf{Node 1.4 --- Chain Construction (17 challenges, all resolved)}} \\
\midrule
ch-14311e6596c & 1.4 & critical & Ambiguous ``next weaker particle'' \\
ch-1ec04fc2163 & 1.4 & critical & Fundamentally incomplete dynamics \\
ch-41d4aa1ee0c & 1.4 & critical & Cascade mechanism omitted \\
ch-f020555a39e & 1.4 & critical & Ambiguous ``next weaker particle'' \\
ch-171bbc0fce3 & 1.4 & major & Vacancy behavior unspecified \\
ch-39bf563cf04 & 1.4 & major & State space preservation not verified \\
ch-4653a7ed47b & 1.4 & major & Irreducibility not addressed \\
ch-4cf17322d27 & 1.4 & major & Missing parameter domains \\
ch-50b37b860d3 & 1.4 & major & Irreducibility not addressed \\
ch-74a7074b0a7 & 1.4 & major & CT vs DT not addressed \\
ch-a1e37bfece4 & 1.4 & major & CT vs DT bridge missing \\
ch-cb75c248d3c & 1.4 & major & State space preservation \\
ch-da6b8df72d3 & 1.4 & major & Vacancy behavior unspecified \\
ch-e172d2fb63a & 1.4 & major & No theorem citation \\
ch-31ae7af3f1e & 1.4 & minor & Misleading phrase \\
ch-6ead9e2ddfb & 1.4 & minor & Missing dependency on 1.1 \\
ch-7274c0c9ec2 & 1.4 & minor & Wrong inference type \\
\midrule

\multicolumn{4}{@{}l}{\textbf{Node 1.5 --- Stationarity (18 challenges, all resolved)}} \\
\midrule
ch-1c4cf108384 & 1.5 & critical & Misattribution to Ferrari--Martin \\
ch-19bf58e0220 & 1.5 & critical & Marginal-to-polynomial gap \\
ch-6670d121be2 & 1.5 & critical & No actual proof structure \\
ch-76d4c8c4a62 & 1.5 & critical & Misattribution of proof method \\
ch-934661602e9 & 1.5 & critical & False product-form claim \\
ch-9658da4177f & 1.5 & critical & False product-form claim \\
ch-e017b73e220 & 1.5 & critical & No actual proof structure \\
ch-053def23083 & 1.5 & major & Uniqueness not established \\
ch-26806410fb8 & 1.5 & major & Notation inconsistency \\
ch-3e9c0d7d87e & 1.5 & major & Uniqueness of stationary distrib.\ \\
ch-4139786e2b5 & 1.5 & major & $q=1$ specialization not justified \\
ch-4aad6c675a9 & 1.5 & major & Wrong inference type \\
ch-6dbd5eeae65 & 1.5 & major & No citation to specific theorem \\
ch-974ea3da1b7 & 1.5 & major & Missing parameter domain \\
ch-c7fac73aacf & 1.5 & major & Missing dependency on 1.4 \\
ch-d3321b98205 & 1.5 & major & No specific theorem citation \\
ch-fa80af229fc & 1.5 & major & Missing dependency on 1.4 \\
ch-ae87a001369 & 1.5 & minor & Ambiguous phrase \\
\midrule

\multicolumn{4}{@{}l}{\textbf{Node 1.6 --- Ratio Identity (16 challenges, all resolved)}} \\
\midrule
ch-0f3e80c57dd & 1.6 & critical & Wrong characterization of lower-order terms \\
ch-1b65900e006 & 1.6 & critical & Missing stationarity proof \\
ch-3cd11cc979f & 1.6 & critical & Fatal logical fallacy (``both sum to 1'') \\
ch-3dbba9297a9 & 1.6 & critical & Open research problem \\
ch-430bb8a8557 & 1.6 & critical & Vague Hecke symmetrization \\
ch-4b9c28e7647 & 1.6 & critical & Vague Hecke symmetrization \\
ch-710f6d7316d & 1.6 & critical & BDW25 \S7 factorization relevance \\
ch-8f48027c9c1 & 1.6 & critical & Wrong inference type \\
ch-9505e439a36 & 1.6 & critical & Fatal logical fallacy \\
ch-d137782ed8d & 1.6 & critical & Ratio identity is open problem \\
ch-e00bb70559e & 1.6 & critical & Unproven prerequisite \\
ch-54dae5b9c11 & 1.6 & major & Conflation of notation \\
ch-7fd14d62c5c & 1.6 & major & Alternative strategies identified but not pursued \\
ch-e4f2f63360a & 1.6 & major & No concrete verification for $n=2$ \\
ch-fef9b62e8eb & 1.6 & major & Undeclared dependency on 1.4, 1.5 \\
ch-2d5406cea99 & 1.6 & major & Missing parameter domain \\
\midrule

\multicolumn{4}{@{}l}{\textbf{Node 1.7 --- Nontriviality (13 challenges, all resolved)}} \\
\midrule
ch-0781b5a14b8 & 1.7 & critical & Argument is not a proof \\
ch-0e46adbdfbd & 1.7 & critical & Transition probs may be described using $F^*_\mu$ \\
ch-0fe08ac59a5 & 1.7 & critical & ``Described using'' is undefined \\
ch-59a456f3dad & 1.7 & critical & Not a proof: ``local rates'' non-sequitur \\
ch-6f040c21803 & 1.7 & critical & False claim of locality \\
ch-dda1f9555ce & 1.7 & critical & What does ``described using'' mean? \\
ch-f044bfe282d & 1.7 & critical & False claim of locality \\
ch-38532088086 & 1.7 & major & Undeclared dependency on 1.4 \\
ch-446df132c99 & 1.7 & major & $F^*_\mu$ are not symmetric-function stationary weights \\
ch-53a810054e3 & 1.7 & major & No engagement with Hecke connection \\
ch-a4ba65ae810 & 1.7 & major & $F^*_\mu$ are not symmetric-function stationary weights \\
ch-ae20e2d87b0 & 1.7 & major & Wrong inference type \\
ch-4068a87ead2 & 1.7 & minor & Missing parameter domain \\
\midrule

\multicolumn{4}{@{}l}{\textbf{Node 1.8 --- Conclusion (2 challenges, all resolved)}} \\
\midrule
ch-0d50b43ab6a & 1.8 & critical & Conclusion depends on all siblings \\
ch-8ffcc34da1f & 1.8 & major & Missing parameter domains \\
\bottomrule
\end{longtable}
}


\newpage
%======================================================================
\section{Lessons Learned}
\label{app:lessons}
%======================================================================

\begin{enumerate}
\item \textbf{The Hecke stationarity argument (Node~1.6) is a plausible original mathematical argument} --- not just a citation of existing work.  It uses BDW25 Proposition~2.10 + AMW24 Theorem~1.1 + Perron--Frobenius in a novel combination.  This makes it the most vulnerable node to further challenges.

\item \textbf{Restructuring Node~1.3 as a corollary of 1.5 + 1.6 elegantly breaks the circularity} without needing to prove the open problem of $F^*_\mu \ge 0$ directly.

\item \textbf{The nontriviality argument via sparsity (Node~1.7) is clean and concrete} --- exhibiting a zero in the transition matrix is much stronger than informal ``local vs global'' heuristics.

\item \textbf{Notation discipline pays off} --- the cross-paper notation table in Node~1.2 makes all subsequent nodes readable and prevents notation-related challenges.

\item \textbf{Provers should always resolve ALL challenges}, not just critical ones.  Leaving minor challenges open accumulates technical debt.

\item \textbf{Session~2 revealed that LLM provers have a tendency to fabricate proof methods} (e.g., attributing the stationarity proof to Ferrari--Martin).  Adversarial verification is essential for catching such errors.

\item \textbf{The most common challenge type was ``notation inconsistency''} (across nodes 1.2, 1.3, 1.5, 1.6).  Establishing a canonical notation table early (Node~1.2) is critical.
\end{enumerate}


\end{document}
