\documentclass[11pt,a4paper]{article}

% --- Packages ---
\usepackage[utf8]{inputenc}
\usepackage[T1]{fontenc}
\usepackage{lmodern}
\usepackage[margin=2.5cm]{geometry}
\usepackage{amsmath,amssymb,amsthm}
\usepackage{mathtools}
\usepackage{enumitem}
\usepackage{booktabs}
\usepackage{hyperref}

% --- Theorem environments ---
\newtheorem{theorem}{Theorem}[section]
\newtheorem{lemma}[theorem]{Lemma}
\newtheorem{proposition}[theorem]{Proposition}
\newtheorem{corollary}[theorem]{Corollary}
\newtheorem{conjecture}[theorem]{Conjecture}
\theoremstyle{definition}
\newtheorem{definition}[theorem]{Definition}
\newtheorem{remark}[theorem]{Remark}

% --- Macros ---
\newcommand{\Sn}{S_n(\lambda)}
\newcommand{\PushTASEP}{\mbox{$t$-PushTASEP}}
\DeclareMathOperator{\wt}{wt}

% --- Title ---
\title{\textbf{Critical Comparison: Automated \texttt{af} Proof Attempt\\
vs.\ Official Solution for Problem~3}\\[6pt]
\large First Proof Benchmark --- Markov Chain with\\
ASEP Polynomial Stationary Distribution}
\author{Comparative Analysis Report\\
First Proof Project}
\date{February 2026}

\begin{document}
\maketitle

\begin{abstract}
Problem~3 of the First Proof benchmark (posed by Lauren Williams) asks
whether there exists a \emph{nontrivial} Markov chain on the set of
compositions $\Sn$ of a restricted partition $\lambda$ whose stationary
distribution is the ratio $F^*_\mu / P^*_\lambda$ of interpolation ASEP
to interpolation Macdonald polynomials at $q=1$. The official solution,
by Ben Dali and Williams, constructs the \emph{interpolation}
\PushTASEP{} --- a modification of the ordinary \PushTASEP{} with a new
``Step~2'' reinsertion phase governed by site-dependent probabilities
$\mathfrak{p}_k$ and $\mathfrak{q}_k$. The automated \texttt{af} proof
attempt also answered YES, but proposed the \emph{ordinary}
(inhomogeneous multispecies) \PushTASEP{} of Ayyer--Martin--Williams as
the Markov chain, and attempted to bridge the gap via a ``ratio
identity'' $f^*_\mu / P^*_\lambda = f_\mu / P_\lambda$. This report
critically compares the two approaches, evaluating what the \texttt{af}
attempt got right and what it missed.
\end{abstract}

\tableofcontents
\newpage

%======================================================================
\section{Problem Statement}
\label{sec:problem}
%======================================================================

Let $\lambda = (\lambda_1 > \cdots > \lambda_n \ge 0)$ be a partition
with distinct parts that is \emph{restricted} (unique part 0, no part
1). Let $\Sn$ be the set of compositions that are permutations of the
parts of $\lambda$. For $\mu \in \Sn$, let $F^*_\mu(x_1,\ldots,x_n;
q,t)$ be the interpolation ASEP polynomial and
$P^*_\lambda(x_1,\ldots,x_n; q,t)$ the interpolation Macdonald
polynomial.

\begin{conjecture}[Williams]
Does there exist a nontrivial Markov chain on $\Sn$ whose stationary
distribution is
\[
  \pi(\mu) = \frac{F^*_\mu(x_1,\ldots,x_n;\, q{=}1,\, t)}
                   {P^*_\lambda(x_1,\ldots,x_n;\, q{=}1,\, t)}
  \qquad \text{for } \mu \in \Sn,
\]
where ``nontrivial'' means the transition probabilities are not
described using the polynomials $F^*_\mu$?
\end{conjecture}

The answer is YES. Both the official solution and the \texttt{af}
attempt agree on this. They disagree, however, on which Markov chain
achieves this stationary distribution and how to prove it.


%======================================================================
\section{The Official Solution (Ben Dali--Williams)}
\label{sec:official}
%======================================================================

\subsection{The Interpolation \PushTASEP{}}

The official solution constructs a \emph{new} Markov chain: the
\textbf{interpolation \PushTASEP{}} with content $\lambda$. This is
\emph{not} the same as the ordinary \PushTASEP{} of
Ayyer--Martin--Williams~\cite{AMW25}. It is a three-step process on
configurations $\mu \in \Sn$, viewed as particles on a ring:

\begin{enumerate}[label=\textbf{(Step \arabic*)}]
\item[\textbf{(Step 0)}] A position $j$ is selected with probability
  $P_j$ proportional to a product involving
  $(x_k - t^{-(n-2)})$ and $(x_k - t^{-(n-1)})$ factors.

\item[\textbf{(Step 1)}] The particle at position $j$ (label $a$) is
  activated and travels clockwise, displacing weaker particles via the
  standard \PushTASEP{} geometric rule: with probability
  $t^{k-1}/[m]_t$ it moves to the $k$-th of the $m$ weaker particles.
  The cascade continues until a vacancy is reached. Position $j$
  becomes vacant.

\item[\textbf{(Step 2)}] The vacancy at position $j$ (now labeled
  $a := 0$) travels to position~1 and proceeds clockwise through
  positions $1, \ldots, j-1$, interacting with each particle $b$ at
  site $k$ via site-dependent probabilities:
  \[
    \mathfrak{p}_k = \frac{t^{-n+1}(1-t)}{x_k - t^{-n+2}}, \qquad
    \mathfrak{q}_k = \frac{(1-t)\,x_k}{x_k - t^{-n+2}}.
  \]
  The vacancy either skips or displaces each particle it encounters,
  with probabilities depending on the relative labels.
\end{enumerate}

The key innovation is Step~2: a reinsertion phase that has no analogue
in the ordinary \PushTASEP. This step is what makes the chain's
stationary distribution involve \emph{interpolation} polynomials rather
than the homogeneous ones.

\subsection{Key Ideas in the Proof}

The proof of Theorem~2.2 (that the stationary probability of $\mu$ is
$F^*_\mu / P^*_\lambda$) proceeds through several stages:

\begin{enumerate}
\item \textbf{Two-line queue encoding (Proposition~3.6).} Step~2 of the
  interpolation \PushTASEP{} is encoded by \emph{unsigned two-line
  queues} $\bar{Q} \in \bar{\mathcal{G}}^\rho_\nu$. The weight of
  $\bar{Q}$ (defined via ball weights and pairing weights) divided by a
  normalizing product equals the transition probability
  $\mathbb{P}_j^{(2)}(\rho, \nu)$.

\item \textbf{Signed-to-unsigned reduction (Lemma~3.2).} Signed
  two-line queues from~\cite{BDW25} are related to the unsigned versions
  by summing over all sign assignments. The sign cancellations produce
  the ball weights in Equation~(6) of the official solution.

\item \textbf{Factorization via Corollary~3.5.} Using the strong
  factorization property of interpolation Macdonald polynomials at
  $q=1$ (from Do\l{}{\k{e}}ga~\cite{Dol17} and~\cite{BDW25}), the
  authors establish:
  \[
    F^*_\rho(\mathbf{x}; 1, t) = F^*_\eta(\mathbf{x}; 1, t) \cdot
    e^*_k(\mathbf{x}; t),
  \]
  where $\eta = \rho^-$ (parts decremented by 1) and $k$ is the number
  of nonzero parts. This uses Theorem~3.4, an interpolation analogue of
  \cite[Theorem~4.18]{AMW25}.

\item \textbf{Stationarity via balance equation (Proposition~3.7).}
  Combining \cite[Lemma~5.4]{AMW25} (for Step~1 transition
  probabilities) with Proposition~3.6 (for Step~2), the full transition
  probability decomposes as
  \[
    \mathbb{P}(\mu, \nu) = \sum_{\rho \in \Sn}
    \frac{a^\mu_\rho \, c^\rho_\nu}{e^*_{n-1}}.
  \]
  Then from the expansion $F^*_\nu = \sum_\eta F^{*\eta}_\nu \cdot
  F^*_{\eta^-}$ (from \cite[Theorem~1.15 and Lemma~5.6]{BDW25}) and
  Corollary~3.5, one obtains:
  \[
    F^*_\nu(\mathbf{x}; 1, t) = \sum_{\eta} F^*_\eta(\mathbf{x}; 1, t)
    \cdot \mathbb{P}(\eta, \nu),
  \]
  proving that $(F^*_\mu)_\mu$ is a left eigenvector of the transition
  matrix with eigenvalue~1.
\end{enumerate}


%======================================================================
\section{The \texttt{af} Automated Proof Attempt}
\label{sec:af}
%======================================================================

\subsection{Overview}

The \texttt{af} attempt over three adversarial sessions produced a
9-node proof tree (root plus 8 children). It answered YES and proposed
the \textbf{ordinary} (inhomogeneous multispecies) \PushTASEP{} of
Ayyer--Martin--Williams~\cite{AMW25} as the Markov chain.

The proof strategy was:
\begin{enumerate}
\item \textbf{Node~1.1:} Define the state space $\Sn$.
\item \textbf{Node~1.2:} Establish notation and the relationship between
  interpolation polynomials $f^*_\mu$ and homogeneous polynomials
  $f_\mu$.
\item \textbf{Node~1.4:} Construct the ordinary \PushTASEP.
\item \textbf{Node~1.5:} Cite AMW24 Theorem~1.1 that the ordinary
  \PushTASEP{} has stationary distribution $f_\mu / P_\lambda$.
\item \textbf{Node~1.6 (Crux):} Establish the ``ratio identity''
  $f^*_\mu / P^*_\lambda = f_\mu / P_\lambda$.
\item \textbf{Node~1.3:} Derive positivity and normalization as a
  corollary.
\item \textbf{Node~1.7:} Prove nontriviality via a sparsity argument.
\item \textbf{Node~1.8:} Conclude.
\end{enumerate}

\subsection{The Ratio Identity (Node~1.6)}

The crux of the \texttt{af} attempt is the claim that for all $\mu \in
\Sn$:
\begin{equation}
\label{eq:ratio}
  \frac{f^*_\mu(\mathbf{x}; 1, t)}{P^*_\lambda(\mathbf{x}; 1, t)}
  = \frac{f_\mu(\mathbf{x}; 1, t)}{P_\lambda(\mathbf{x}; 1, t)}.
\end{equation}
The proposed proof had three steps:
\begin{enumerate}
\item Both $f^*_\mu$ and $f_\mu$ satisfy the same Hecke relations
  (BDW25 Proposition~2.10).
\item The AMW24 balance equation proof for $f_\mu$ ``transfers'' to
  $f^*_\mu$ because it uses only these Hecke relations.
\item Perron--Frobenius uniqueness forces proportionality:
  $f^*_\mu = C \cdot f_\mu$ for a constant $C$ independent of $\mu$.
\end{enumerate}

\subsection{Session History}

The \texttt{af} attempt went through three sessions:
\begin{itemize}
\item \textbf{Session~1:} Proof tree registration (no verification).
\item \textbf{Session~2:} Adversarial verification raised 97 challenges
  (36 critical, 49 major, 12 minor). Five systemic problems were
  identified: wrong arXiv reference, fabricated Ferrari--Martin proof
  method, logical fallacy in Node~1.6, circular dependency, and notation
  inconsistency.
\item \textbf{Session~3:} All 8 leaf nodes were rewritten, resolving all
  97 challenges. Node~1.6 was completely rewritten with the three-step
  Hecke argument. Node~1.3 was restructured as a corollary. No further
  verification was performed.
\end{itemize}

The proof tree ended with 9 pending nodes, 0 validated, 0 refuted, and 0
open challenges.


%======================================================================
\section{Critical Comparison}
\label{sec:comparison}
%======================================================================

\subsection{The Fundamental Difference: Which Markov Chain?}

The most significant difference between the two approaches is the choice
of Markov chain:

\begin{center}
\begin{tabular}{@{}p{3cm}p{5cm}p{5cm}@{}}
\toprule
& \textbf{Official Solution} & \textbf{\texttt{af} Attempt} \\
\midrule
Markov chain & Interpolation \PushTASEP{} & Ordinary \PushTASEP{} \\
Dynamics & Steps 0 + 1 + 2 (novel Step~2) & Standard \PushTASEP{} \\
Stationary dist.\ & $F^*_\mu / P^*_\lambda$ directly & $f_\mu / P_\lambda$ (then ratio identity) \\
Key novelty & New chain with reinsertion & Ratio identity \\
\bottomrule
\end{tabular}
\end{center}

The official solution constructs a \emph{genuinely different} Markov
chain whose stationary distribution is \emph{directly} $F^*_\mu /
P^*_\lambda$, without needing to pass through the homogeneous
polynomials. The \texttt{af} attempt, by contrast, tried to reuse the
\emph{same} chain (the ordinary \PushTASEP) and bridge the gap
algebraically.

This is a crucial distinction. The authors' commentary on AI-generated
solutions explicitly notes that a common LLM failure mode was ``to
change the problem to a related but different, and already-solved
problem, namely, to replace interpolation ASEP and interpolation
Macdonald polynomials by ASEP and Macdonald polynomials. In this case
the solution to this problem is the \PushTASEP{} and was given in a
paper by Ayyer, Martin, and Williams.'' The \texttt{af} attempt falls
squarely into this category.

\subsection{Is the Ratio Identity True?}

The \texttt{af} attempt's entire strategy depends on the ratio identity
\eqref{eq:ratio}. Does the official solution validate it?

\textbf{Yes, but indirectly.} The official solution's Theorem~3.4
establishes that
\[
  \frac{G^*_\eta(\mathbf{x}; t)}{P^*_\lambda(\mathbf{x}; 1, t)}
  = \frac{F^*_\eta(\mathbf{x}; 1, t)}{P^*_\kappa(\mathbf{x}; 1, t)},
\]
where $G^*_\eta = \sum_{\rho: \phi(\rho) = \eta} F^*_\rho$. In the
special case where $\lambda$ is restricted and $\phi : i \mapsto
\max(i-1, 0)$, the map $\phi$ is injective on $\Sn$, so $G^*_\eta =
F^*_\rho$ for a unique $\rho$. This, combined with the factorization
$P^*_\lambda / P^*_\kappa = e^*_k$ (Corollary~3.5), gives relationships
between interpolation polynomials at \emph{different} partitions.

However, the official solution does \emph{not} establish the ratio
identity \eqref{eq:ratio} in the form claimed by the \texttt{af}
attempt (i.e., $f^*_\mu / P^*_\lambda = f_\mu / P_\lambda$ at the
\emph{same} partition $\lambda$). Instead, the official solution avoids
this identity entirely by constructing a different Markov chain that
directly has $F^*_\mu / P^*_\lambda$ as its stationary distribution.

That said, the ratio identity \eqref{eq:ratio} is in fact a
\emph{consequence} of both solutions being correct: if the ordinary
\PushTASEP{} has stationary distribution $f_\mu / P_\lambda$ (AMW24
Theorem~1.1) and the interpolation \PushTASEP{} has stationary
distribution $F^*_\mu / P^*_\lambda$ (Ben Dali--Williams Theorem~2.2),
and if the two chains happen to have the \emph{same} stationary
distribution, then the ratio identity holds. But this would require
showing that the ordinary \PushTASEP{} and the interpolation
\PushTASEP{} have the same stationary distribution, which is by no means
obvious given their different dynamics.

\textbf{In summary:} the ratio identity is a nontrivial claim that the
official solution neither proves nor needs. The \texttt{af} attempt's
strategy of reducing to this identity was the wrong approach.


\subsection{Did the \texttt{af} Attempt Identify the Right Techniques?}

\begin{center}
\begin{tabular}{@{}p{5cm}cc@{}}
\toprule
\textbf{Technique} & \textbf{Official} & \textbf{\texttt{af}} \\
\midrule
Multiline queues & Central & Mentioned \\
Signed two-line queues & Central & Not used \\
Unsigned two-line queues & Central & Not used \\
Hecke relations (BDW25) & Not used & Central \\
AMW24 Thm~1.1 & Used (Step~1) & Central \\
Factorization ($P^*_\lambda$) & Central & Not used \\
Perron--Frobenius & Not used & Central \\
BDW25 Thm~1.15 ($f^*=F^*$) & Used & Used \\
\bottomrule
\end{tabular}
\end{center}

The official solution relies heavily on the \emph{combinatorial}
machinery of two-line queues (both signed and unsigned) and the
factorization of interpolation Macdonald polynomials. The \texttt{af}
attempt instead relied on the \emph{algebraic} machinery of Hecke
operators and Perron--Frobenius uniqueness. These are fundamentally
different proof strategies.

Notably, the official solution does use AMW24, but only for the Step~1
transition probabilities (\cite[Lemma~5.4]{AMW25}), not for its
stationarity theorem. The \texttt{af} attempt treated AMW24 Theorem~1.1
as the centerpiece, which reflects the misidentification of the Markov
chain.


%======================================================================
\section{Evaluation of the \texttt{af} Ratio Identity (Node~1.6)}
\label{sec:node16}
%======================================================================

The \texttt{af} attempt's Node~1.6 claimed the ratio identity
\eqref{eq:ratio} via three steps. We evaluate each in light of the
official solution.

\subsection{Step~1: Hecke Relations}

The claim that both $f^*_\mu$ and $f_\mu$ satisfy the same Hecke
relations (BDW25 Proposition~2.10) is \textbf{correct}. This is a
published result. However, the official solution does not use Hecke
relations at all, suggesting they are not the right tool for this
problem.

\subsection{Step~2: Balance Transfer}

The claim that the AMW24 balance equation proof ``transfers'' from
$f_\mu$ to $f^*_\mu$ because both satisfy the same Hecke relations is
\textbf{unverified and likely false as stated}. The argument asserts
that the AMW24 proof uses \emph{only} the Hecke relations (a)--(c), and
therefore applies verbatim to $f^*_\mu$.

There are several problems:
\begin{enumerate}
\item The AMW24 proof of stationarity is specific to the \emph{ordinary}
  \PushTASEP. Even if $f^*_\mu$ satisfies the same Hecke relations, the
  relevant balance equation involves the \emph{transition matrix} of the
  chain. If the chain is the ordinary \PushTASEP, then
  $\sum_\mu f^*_\mu \cdot Q_{\mu,\nu} = 0$ would mean that $f^*_\mu$
  is \emph{also} a stationary measure of the ordinary \PushTASEP. By
  Perron--Frobenius uniqueness, this would force
  $f^*_\mu = C \cdot f_\mu$, i.e., the interpolation polynomials are
  proportional to the homogeneous ones.

\item But $f^*_\mu$ and $f_\mu$ are \emph{not} proportional in general:
  $f_\mu$ is the top-degree homogeneous component of $f^*_\mu$ (BDW25
  Theorem~2.3), and $f^*_\mu$ contains lower-degree correction terms.
  Hence $f^*_\mu \neq C \cdot f_\mu$ for any constant $C$ (unless the
  lower-degree terms vanish, which they do not in general).

\item This means Step~2 of the \texttt{af} argument is \textbf{false}:
  the AMW24 proof does \emph{not} transfer from $f_\mu$ to $f^*_\mu$
  for the ordinary \PushTASEP, because $f^*_\mu$ is not in the left
  null space of the ordinary \PushTASEP's generator $Q$.
\end{enumerate}

The official solution confirms this diagnosis: a \emph{different} Markov
chain (the interpolation \PushTASEP, with its additional Step~2) is
needed to have $F^*_\mu$ as stationary weights. The ordinary
\PushTASEP{} only gives $f_\mu$ as stationary weights.

\subsection{Step~3: Perron--Frobenius Uniqueness}

The Perron--Frobenius argument is \textbf{correct in principle} but is
applied to a false premise (Step~2). If $f^*_\mu$ were indeed in the
left null space of the ordinary \PushTASEP's generator, then
Perron--Frobenius would force proportionality. But since $f^*_\mu$ is
\emph{not} in that null space, the argument is vacuous.

\subsection{Verdict on Node~1.6}

The ratio identity as formulated by the \texttt{af} attempt is
\textbf{false}. The interpolation polynomials $f^*_\mu$ are not
proportional to the homogeneous polynomials $f_\mu$ (they differ by
lower-degree terms), and the ordinary \PushTASEP{} does not have
$f^*_\mu / P^*_\lambda$ as its stationary distribution. The \texttt{af}
attempt's HANDOFF.md itself flagged this as a ``high risk'' node, and
this assessment was correct: the node contains a fundamental error.


%======================================================================
\section{What the \texttt{af} Attempt Got Right}
\label{sec:right}
%======================================================================

Despite the fundamental error in the choice of Markov chain, the
\texttt{af} attempt made several correct identifications:

\begin{enumerate}
\item \textbf{The answer is YES.} The \texttt{af} attempt correctly
  identified that a nontrivial Markov chain exists.

\item \textbf{The \PushTASEP{} family is the right family of chains.}
  The official solution's interpolation \PushTASEP{} is a modification
  of the ordinary \PushTASEP. The \texttt{af} attempt correctly
  identified the \PushTASEP{} as the relevant dynamical framework.

\item \textbf{Correct use of published literature.} After Session~3
  corrections, the \texttt{af} attempt correctly cited AMW24
  Theorem~1.1, BDW25 Proposition~2.10, and other results. The notation
  table (Node~1.2) was comprehensive.

\item \textbf{The nontriviality argument (Node~1.7) is sound.} The
  sparsity argument --- exhibiting a zero in the transition matrix ---
  is a valid proof of nontriviality. It would work for the interpolation
  \PushTASEP{} as well, since the interpolation chain also has sparse
  transition matrices (Step~1 only moves particles clockwise).

\item \textbf{Clean logical structure after Session~3.} The
  restructuring of Node~1.3 as a corollary of 1.5 + 1.6 was a good
  logical move that broke the circularity. The dependency chain
  $1.1 \to 1.2 \to 1.4 \to 1.5 \to 1.6 \to 1.3 \to 1.7 \to 1.8$ is
  clean.

\item \textbf{The adversarial process caught real errors.} Session~2
  correctly identified the logical fallacy in the original Node~1.6
  (``both sum to 1, therefore equal''), the fabricated Ferrari--Martin
  proof method, and the circular dependency. The adversarial framework
  worked as intended.
\end{enumerate}


%======================================================================
\section{What the \texttt{af} Attempt Missed}
\label{sec:missed}
%======================================================================

\begin{enumerate}
\item \textbf{The need for a new Markov chain.} The most critical miss.
  The problem asks for a Markov chain whose stationary distribution
  involves \emph{interpolation} polynomials, not homogeneous ones. The
  ordinary \PushTASEP{} gives the homogeneous ratio $f_\mu / P_\lambda$,
  not the interpolation ratio $F^*_\mu / P^*_\lambda$. A new chain ---
  the interpolation \PushTASEP{} with its Step~2 reinsertion --- is
  required.

\item \textbf{The combinatorial machinery of two-line queues.} The
  official solution is fundamentally combinatorial, encoding the Step~2
  transitions via signed and unsigned two-line queues. The \texttt{af}
  attempt did not engage with this machinery at all, instead pursuing an
  algebraic approach via Hecke operators.

\item \textbf{The factorization of interpolation Macdonald polynomials.}
  The official solution crucially uses $P^*_\lambda(\mathbf{x}; 1, t) =
  \prod_i e^*_{\lambda'_i}(\mathbf{x}; t)$ (from Do\l{}{\k{e}}ga and
  BDW25), which provides the key Corollary~3.5 relating $F^*_\rho$ to
  $F^*_{\rho^-}$. The \texttt{af} attempt mentioned BDW25 Theorem~7.1
  (factorization) as an ``alternative route'' but never pursued it.

\item \textbf{The distinction between interpolation and homogeneous
  polynomials.} While the \texttt{af} attempt acknowledged that $f_\mu$
  is the top-degree component of $f^*_\mu$ (BDW25 Theorem~2.3), it did
  not fully reckon with the implications: $f^*_\mu \neq C \cdot f_\mu$,
  so the ordinary \PushTASEP{} cannot have $f^*_\mu$ as stationary
  weights.

\item \textbf{The role of the BDW25 expansion.} The official proof uses
  the expansion $F^*_\nu = \sum_\eta F^{*\eta}_\nu \cdot F^*_{\eta^-}$
  from \cite[Theorem~1.15 and Lemma~5.6]{BDW25} as the starting point
  for the stationarity proof. The \texttt{af} attempt did not use this
  result.

\item \textbf{The inherent difficulty flag in BDW25 Remark~1.17.} The
  \texttt{af} attempt correctly noted that BDW25 Remark~1.17 defers the
  interpolation probabilistic interpretation to a forthcoming paper, but
  drew the wrong conclusion: it treated this as suggesting the Hecke
  transfer argument merely needs careful verification, rather than as a
  signal that a genuinely new construction is needed.
\end{enumerate}


%======================================================================
\section{Alignment with Common LLM Failure Modes}
\label{sec:failure}
%======================================================================

The authors' commentary identifies two common LLM failure modes for
Problem~3:

\begin{enumerate}
\item \textbf{Metropolis--Hastings (trivial solution).} Using the
  desired distribution formula to define transition rates. The
  \texttt{af} attempt avoided this failure mode --- it correctly
  recognized that this would be trivial.

\item \textbf{Replacing interpolation polynomials by homogeneous ones.}
  Solving the already-solved problem for $f_\mu / P_\lambda$ (ordinary
  ASEP/Macdonald polynomials) instead of the interpolation versions
  $F^*_\mu / P^*_\lambda$. The \texttt{af} attempt \emph{fell into this
  exact failure mode}, proposing the ordinary \PushTASEP{} and
  attempting to bridge the gap with the ratio identity.
\end{enumerate}

The \texttt{af} attempt's approach is essentially a more sophisticated
version of failure mode~2: rather than simply ignoring the
``interpolation'' qualifier, it acknowledged the distinction and
attempted an algebraic bridge. But the bridge (the ratio identity) is
built on the false premise that $f^*_\mu$ is proportional to $f_\mu$.


%======================================================================
\section{Lessons Learned}
\label{sec:lessons}
%======================================================================

\begin{enumerate}
\item \textbf{LLMs tend to reduce novel problems to known ones.} The
  \texttt{af} attempt replaced the open problem (interpolation
  \PushTASEP{} stationarity) with the solved problem (ordinary
  \PushTASEP{} stationarity) and tried to bridge the gap algebraically.
  This is a pervasive LLM failure mode: rather than constructing a
  genuinely new object, the system reuses a known object and claims the
  difference does not matter.

\item \textbf{The adversarial framework caught errors but not the
  fundamental one.} Sessions~2 and~3 caught fabricated proofs (Ferrari--Martin),
  logical fallacies (``both sum to 1''), circular dependencies, and
  notation errors. But they did not catch the deepest error: that the
  \emph{wrong Markov chain} was proposed. The adversarial verifiers
  correctly flagged Node~1.6 as high-risk but could not definitively
  refute it.

\item \textbf{Risk assessment was accurate.} The HANDOFF.md correctly
  identified Node~1.6 as the highest-risk node and acknowledged the
  critical caveat about Step~2. The system ``knew'' where the proof was
  weakest but could not resolve the issue.

\item \textbf{The ratio identity approach is seductive but wrong.} The
  identity $f^*_\mu / P^*_\lambda = f_\mu / P_\lambda$ looks natural
  and would elegantly reduce the problem to known results. But it is
  false: $f^*_\mu$ contains lower-degree terms that prevent
  proportionality to $f_\mu$.

\item \textbf{Combinatorial constructions may be harder for LLMs than
  algebraic manipulations.} The official solution requires inventing a
  new dynamical system (Step~2 of the interpolation \PushTASEP) and
  encoding its transitions via two-line queues. This is a creative
  combinatorial construction that may be inherently harder for LLMs to
  generate than algebraic manipulations of known identities.

\item \textbf{The ``forthcoming paper'' signal was misread.} BDW25
  Remark~1.17's deferral to a forthcoming paper was a strong signal that
  the interpolation probabilistic interpretation requires new ideas
  beyond the ordinary \PushTASEP. The \texttt{af} attempt interpreted
  this as ``the transfer is nontrivial but doable''; the correct
  interpretation was ``a new construction is needed.''

\item \textbf{Nontriviality arguments transfer well.} The \texttt{af}
  attempt's sparsity argument for nontriviality (Node~1.7) is clean,
  correct, and would apply to the interpolation \PushTASEP{} as well.
  This is a reusable technique.
\end{enumerate}


%======================================================================
\section{Conclusion}
\label{sec:conclusion}
%======================================================================

The \texttt{af} automated proof attempt for Problem~3 correctly
identified the answer (YES) and the right family of Markov chains (the
\PushTASEP{} family), but proposed the wrong specific chain (the
ordinary \PushTASEP{} instead of the interpolation \PushTASEP) and
attempted to bridge the gap with a ratio identity that is false. The
official solution by Ben Dali and Williams constructs a genuinely new
Markov chain --- the interpolation \PushTASEP{} with its novel Step~2
reinsertion phase --- and proves stationarity directly via two-line
queue combinatorics and the factorization of interpolation Macdonald
polynomials.

The \texttt{af} attempt's failure aligns precisely with the second
common LLM failure mode identified by the problem authors: replacing
interpolation polynomials by homogeneous ones and solving the
already-solved problem. The adversarial framework successfully caught
many errors (97 challenges in Session~2) but could not catch the
fundamental misidentification of the Markov chain. The system's own risk
assessment correctly flagged Node~1.6 (the ratio identity) as the
highest-risk node, demonstrating that the \texttt{af} framework has some
capacity for self-diagnosis even when it cannot resolve the underlying
issue.


%======================================================================
% References
%======================================================================

\begin{thebibliography}{99}

\bibitem{AMW25}
A.~Ayyer, J.~B.~Martin, and L.~K.~Williams,
\emph{The inhomogeneous $t$-PushTASEP and Macdonald polynomials at
$q=1$},
Annales de l'Institut Henri Poincar\'e D (2025).
arXiv:2403.10485.

\bibitem{BDW25}
H.~Ben Dali and L.~K.~Williams,
\emph{A combinatorial formula for interpolation Macdonald polynomials},
Preprint arXiv:2510.02587 (2025).

\bibitem{CMW22}
S.~Corteel, O.~Mandelshtam, and L.~K.~Williams,
\emph{From multiline queues to Macdonald polynomials via the exclusion
process},
Amer.\ J.\ Math.\ \textbf{144}(2) (2022), 395--436.

\bibitem{Dol17}
M.~Do\l{}{\k{e}}ga,
\emph{Strong factorization property of Macdonald polynomials and
higher-order Macdonald's positivity conjecture},
J.\ Algebraic Combin.\ \textbf{46}(1) (2017), 135--163.

\bibitem{AS19}
P.~Alexandersson and M.~Sawhney,
\emph{Properties of non-symmetric Macdonald polynomials at $q=1$ and
$q=0$},
Ann.\ Comb.\ \textbf{23}(2) (2019), 219--239.

\end{thebibliography}


\end{document}
